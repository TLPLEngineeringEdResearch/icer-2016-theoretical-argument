\subsection{Introduction}\label{introduction-2}

Although imaginative and carefully designed, the Harvard course taught
that there is only one right way to approach the computer, a way that
emphasizes control through structure and planning. There are many
virtues to this computational approach (it certainly makes sense when
dividing the labor on a large programming project), but Lisa and Robin
have intellectual styles at war with it. Lisa says she has ``turned
herself into a different kind of person'' in order to perform, and Robin
says she has learned to ``fake it.'' Although both women are able to get
good grades in their programming course, they represent casualties of
this war. Both deny who they are in order to succeed. (Turkle \& Papert,
1991, p. 165)

Over 20 years ago Sherry Turkle and Seymour Papert called for
epistemological pluralism in computing education. Their argument --- in
my view ---had multiple parts:

\begin{enumerate}
\def\labelenumi{\arabic{enumi}.}
\item
  Programming a computer involved a shift in how one thought about the
  nature knowledge from a propositional \emph{what is} to an imperative
  \emph{how to} (Abelson \& Sussman, 1996; Papert, 1980).
\item
  Historically, there was a tendency to treat computers and the act of
  programming them as an extension of the same Western logico-deductive
  knowledge tradition that fueled science.
\item
  Recent scholarship, often at the intersection of post-structuralism
  and social studies of science, had been troubling the idea that
  knowledge-work in science proceeded solely by way of logical
  deductions (Keller, 1983; Latour, 1987; Traweek, 1988)
\item
  Data from professionals suggested programming, too, didn't proceed
  solely by way of the structured-planning approach emergent from
  Western logico-deductive knowledge traditions. There were many ways of
  knowing and constructing knowledge in programming.
\item
  A viewpoint of \emph{epistemological pluralism} --- embracing multiple
  kinds and ways of knowing --- was informing the field of science
  studies but not, to a large degree, how education researchers thought
  about or taught computing.
\item
  The lack of support for diverse ways of knowing in computing
  classrooms demonstrably hurt students.
\end{enumerate}

To Turkle and Papert (1991), the result of this culturally-rooted
epistemological tension was a war. More specifically, it was a war where
students of computing were the casualties, and the aggressor was a
distributed failure to recognize, embrace, and support diverse ways of
knowing. While I think their strong-form characterization of a ``war''
no longer applies, their consideration of epistemological issues is
still relevant two decades later. An understanding that computing does
in fact involve diverse ways of knowing should continue to inform
research and discussions on learning

This paper tries to model what directs and sustains students'
in-the-moment activity when they program. Its focus is on early-stage
program design; analyzing how what students say, do, write, and gesture
\emph{even before they type code} can help us improve theories of
cognition and activity in learning programming. Using a ``revelatory
case study'' (Yin, 2009) of two students in an introductory programming
course, I argue the following:

\begin{enumerate}
\def\labelenumi{\arabic{enumi}.}
\item
  Students' early-stage design activity reveals patterns outside the
  explanatory scope of misconception-based accounts of cognition.
\item
  For a subset of phenomena, we can recast students' productive
  capacities and their difficulties in terms of epistemological stances.
\item
  Such recastings are analytically powerful. Beyond Rebecca and Lionel,
  they could potentially explain the diversity of practices I saw other
  students take up.
\item
  As evidenced by work in science and math education, dynamic
  epistemological models can offer a lens for reforming assessment and
  instruction.
\end{enumerate}

\subsection{Literature Review}\label{literature-review}

I begin by exploring the historical legacy of Turkle and Papert's (1991)
remarks on epistemological pluralism in computing. I'll then describe
what I see as a well-intentioned obstacle: the research-based
preoccupation with students' misconceptions in computing education.
These two stra.

\subsubsection{Computing culture should support a diversity of ways of
knowing}\label{computing-culture-should-support-a-diversity-of-ways-of-knowing}

Turkle \& Papert (1991) describe a tension between individuals and a
kind of cultural collective. In this tension, individual students and
their personally-identified ways of knowing were in conflict with the
larger culture of computing and its manifested ways of knowing. Lisa,
for instance, was a Harvard student who identified as a poet:

Lisa experiences language as transparent, she knows where all the
elements are at every point in the development of her ideas. She wants
her relationship to computer language to be similarly transparent. When
she builds large programs she prefers to write her own smaller
``building block'' procedures even though she could use prepackaged ones
from a program library; she resents the latter's opacity. (Turkle \&
Papert, 1991, p. 164)

Turkle and Papert connect Lisa's programming practices to Lisa's
personal view of knowledge. She writes building block procedures because
of a continuity between herself as knower-of-her-own-poetry and that of
knower-of-her-own-programs. Across both contexts, Lisa \emph{resents
opacity} because it is an obstacle to her knowing. But, that resentment
slows her progress on projects in the class.

In part because of her commitments to transparency and the difficulty
they created, Lisa changed her approach to programming. She ultimately
had to, in her words, ``be a different kind of person with the machine''
(Turkle \& Papert, 1991, p. 164). Turkle and Papert explain:

She had been told that the ``right way'' to do things was to control a
program through planning and black­boxing, the technique that lets you
exploit opacity to plan something large without knowing in advance how
the details will be managed. Lisa recognized the value of these
techniques --- for someone else. She struggled against using them as the
starting points for her learning. Lisa ended up abandoning the fight,
doing things ``their way,'' and accepting the inevitable alienation from
her work. (Turkle \& Papert, 1991, p. 164)

The phrase ``inevitable alienation'' is particularly damning. One might
accuse Turkle and Papert (1991) of hyperbole, but research from other
disciplines suggests they're not overstating the case. Identity
alienation for epistemological reasons --- or some variation of them ---
is playing out in other subjects as well.

Connections and tensions between identities, learning, and personal ways
of knowing are not unique to computing education. Wortham (2006), for
example, detailed instances of social identification and academic
identity as \emph{jointly emerging} in a combined high school English
and History class. Nasir and colleagues explored how different learning
environments afford access to identity, domain practices, and
self-expression (Nasir \& Cooks, 2009; Nasir \& Hand, 2008). To single
out one study, Nasir \& Hand (2008) followed players ``from the
{[}basketball{]} court to the classroom.'' They found the court afforded
players a sense of role on the team, chances to be expressive through
play, and ``access to the domain {[}of basketball{]} as a whole'' (Nasir
\& Hand, 2006, p. 147). Comparatively, the mathematics classroom offered
little (if any) sense that students were on a team, a much narrower role
for them as having either right or wrong answers, and no strong sense of
self-expression or creativity through activity. These markedly different
contexts led not only to differences in kinds of activity but, Nasir and
Hand (2006) argue, divergent \emph{practice-linked identities} for the
players. In a sense, players had a different identity on the court than
they did in the mathematics classroom because who they were was strongly
coupled to a sense of their relationship with a practice.

Boaler's work on learning and identity in mathematics classrooms (Boaler
\& Greeno, 2000; Boaler, 1998, 2000, 2002) offers resonant findings. In
particular, Boaler (2000) strongly echoes the student voices in Turkle
and Papert (1991). Where Turkle and Papert's aggressor is the dominant
computing culture, Boaler describes a beast with similar effects in the
form of ``school mathematics'' --- the \emph{monotonous},
\emph{meaningless} (Boaler, 2002, pp. 383--384) imposition of routinized
mathematical procedures in ways that seem far removed from the real
world. Of school mathematics, Boaler writes:

School mathematics, for many of them, was \emph{of another world} and to
fully engage in that world, students needed to suspend their knowledge
of the real world, suppress their desire to interact with others, and
strive to reproduce standard procedures that held little meaning for
them. (Boaler, 2002, p. 392 emphasis in original)

Schoenfeld (1988, 1991) had already argued school mathematics could and
did depart substantially from the core of mathematical thinking. Boaler
established that such school mathematics could alienate learners
(Boaler, 1998, 2000) and influence their feelings about pursuing
mathematics after secondary school (Boaler \& Greeno, 2000).

Most recently, Danielak, Gupta, and Elby (in press) extended those
findings in engineering education. In their 3 ½ year case study of an
electrical engineering undergraduate named ``Michael,'' those authors
argued Michael's practice of sense-making in engineering coursework was
strongly coupled to his identity. Michael's passion for deeply
understanding concepts, much like Lisa's (Turkle \& Papert, 1991) desire
to understand every element of her program, did not mesh neatly with the
larger culture of his courses. But, because Michael's practice of
sense-making entwined with his identity, forces that pushed against
sense-making alienated him. That tension was particularly pointed when
Michael described his father's resistance to Michael's way of thinking:

{[}My dad said{]} ``you're just an undergraduate. Nobody expects
undergraduates to understand how anything works. That's why you go to
graduate school.'' I was like ``look, you know. I'm gonna be an unhappy
person if I have to....I have life goals other than to just get a good
grade on the exam. Other things are important to me.'' (Danielak et al.,
in press)

And, much like Turkle and Papert's (1991) Lisa, Michael learned to
curtail parts of his identity to get by in class:

I think the reason {[}pursuing deep learning{]} actually hasn't affected
my GPA is because I view learning as a hobby. So, as with any hobby, you
shouldn't let it interfere your GPA. But it \emph{is} one of my hobbies,
and I \emph{do} enjoy learning, I just---up to the point where I get my
grades done \{raises eyebrow\}. (Danielak et al., in press)

Ultimately, research suggests the identity-epistemology tensions Turkle
and Papert (1991) described in computing resonates with those found in
related disciplines. And, what such accounts --- particularly those of
Boaler and others (Boaler \& Greeno, 2000; Boaler, 1998, 2000, 2002;
Danielak et al., in press) --- have in common is a culturally-sanctioned
``right way'' (Turkle \& Papert, 1991, p. 164) of thinking or knowing
that alienates some students. Avowedly such ways of knowing have use and
value. Turkle \& Papert themselves acknowledge ``there are many
virtues'' to black-boxing and top-down design in programming (Turkle \&
Papert, 1991, p. 165). But,

\begin{enumerate}
\def\labelenumi{\arabic{enumi}.}
\item
  When a preponderance of results from science studies suggests a
  plurality of ways of knowing among practicing scientists and
  programmers, and
\item
  When research from education shows that the inflexible imposition of
  one way of knowing above all else is alienating students, then
\item
  It's worth reconsidering whether a narrow sense of what counts as
  knowing might still be hindering improvements in computing education.
\end{enumerate}

In the next section, I turn to misconceptions research in computing
education. In keeping with observations 1--3, I argue research
approaches that inflexibly privilege canonical knowledge do so at the
expense of other productive knowledge and ways of knowing that students
have.

\subsubsection{Misconceptions research in computing education tends to
ignore students' productive
knowledge}\label{misconceptions-research-in-computing-education-tends-to-ignore-students-productive-knowledge}

In the past three decades, educational research has had a marked focus
on students' misconceptions in programming. It's a focus with a sensible
origin. Students get things wrong in programming --- often
systematically so --- and in ways that seem resistant to instruction.
The cause of those errors is theorized to be something cognitive,
whether it's a ``bug'' (Pea et al., 1987; Pea, 1986; VanLehn, 1990) a
``misconception'' (Bayman \& Mayer, 1983; Bonar \& Soloway, 1985;
Clancy, 2004; Gal-Ezer \& Zur, 2004; Herman et al., 2008; Kaczmarczyk et
al., 2010), a ``belief'' (Fleury, 1993) or a ``student-constructed
rule'' (Fleury, 1991, 2000). Instruction should try to identify,
address, and correct these misconceptions (Clancy, 2004) because they
can represent barriers to learning.

How that line of thinking and research becomes problematic is two-fold.
First, when taken in total the alleged brokenness of student knowledge
begins eclipsing all else in studying the cognition of learning to
program. In other words, most cognitively-focused educational research
in computer science treats students as having varied degrees of
deficiency with respect to canonical knowledge. Below is an unordered,
partial sampling of topics about which researchers have documented
students' misconceptions. Note across the list the variation in both the
grain sizes of students' misconceptions and the programming languages in
which they manifest:

\begin{itemize}
\item
  Objects in object-oriented programming (Holland, Griffiths, \&
  Woodman, 1997)
\item
  Algorithms and data structures (Danielsiek, Paul, \& Vahrenhold, 2012;
  Paul \& Vahrenhold, 2013)
\item
  Programming statements in BASIC (Bayman \& Mayer, 1983)
\item
  Programming in Java (Fleury, 2000)
\item
  Programming in Pascal (Fleury, 1993)
\item
  Parameter-passing (Fleury, 1991)
\item
  Arrays in Java (Kaczmarczyk et al., 2010)
\item
  Objects in Java (Kaczmarczyk et al., 2010)
\item
  Algorithms and computational complexity (Trakhtenbrot, 2013)
\item
  Boolean logic (Herman et al., 2008)
\item
  The efficiency of algorithms (Gal-Ezer \& Zur, 2004)
\item
  The Build-Heap algorithm (Seppälä, Malmi, \& Korhonen, 2006)
\item
  Hashtables (Patitsas, Craig, \& Easterbrook, 2013)
\item
  The correctness of programs (Kolikant \& Mussai, 2008)
\end{itemize}

Clancy (2004) provides a comprehensive overview of this line of
research, though in the past decade it has only grown. Indeed, roughly
half the articles above were published in the ten years since Clancy's
overview.

Identifying and removing barriers to student learning seems like a good
thing. So, it should follow that cataloging student misconceptions and
developing remedies for them should also be a good thing. But, the
logical implication isn't that clean. In some cases students display
productive, useful knowledge that's either ignored or outright
criticized by researchers. Aligning students toward canonical knowledge
makes sense, but doing so at the expense of---or in direct contradiction
to---useful ways of knowing seems undesirable at best. Next, I expand on
two examples from misconceptions research in programming. Specifically,
I show how and why I think a misconceptions focus in programming casts
aside students' useful intuitions and understandings.

The first example comes from Kaczmarczyk et al. (2010). Part of that
study involved giving students snippets of Java code and asking students
to diagram (or pseudo-code) how the information would be stored in
memory. Below I have reproduced the code for Problem 2 (Kaczmarczyk et
al., 2010, p. 110):

Cheese{[}{]} cheeses = new Cheese{[}4{]};

Meat{[}{]} meats = new Meat{[}2{]};

Turkey turkey;

Ham ham;

RoastBeef roastBeef;

boolean lettuce = true;

boolean tomato = true;

SauceType sauceType = new SauceType();

int numMeat;

int numCheese;

In diagramming this information, a student in the study makes a mistake:

Student3 makes incorrect assumptions about connections between variables
to the extent that the student makes a mistake concerning the types of
the variables. As a result, the student places Objects of different
types in an array whose type matches none of them: ``And so because
there's two arrays, cheese and meats, uh, all those turkey and ham and
roast beef are gonna be sorted into the meats array.'' (Kaczmarczyk et
al., 2010, p. 110)

The researchers are correct in the sense that \textbf{turkey} and
\textbf{ham} and \textbf{roastBeef} will \emph{not} be sorted into the
\textbf{meats} array. First, there is no code here that places
\textbf{turkey} and \textbf{ham} in the array; there is only code that
declares them as variables. Moreover, as written, an attempt to place
\textbf{turkey} and \textbf{ham} and \textbf{roastBeef} into the array
would fail. Because of type restrictions in Java, only objects of class
Meat (or objects that inherit from class Meat) can go in the array.
\textbf{turkey} and \textbf{ham} and \textbf{roastBeef} are, perhaps
confusingly, references to object instances of classes \textbf{Turkey}
and \textbf{Ham} and \textbf{RoastBeef}, so in the current snippet they
cannot enter an array of type \textbf{Meat} because (1) they don't yet
exist as objects and (2) even if they did exist, their types don't match
the array's type. The authors call this misconception \emph{semantics to
semantics}, which occurs ``when the student inappropriately
assume{[}s{]} details about the relationship and operation of code
samples, although such information was neither given nor implied''
(Kaczmarczyk et al., 2010, p. 110).

Again, the researchers are right that the student is failing to describe
the code in a way consistent with canon.\footnote{In this case, canon is
  the specifications and operations of the Java language and its
  compilers.} But, in their non-canonical thinking Student3 evidences
potentially productive insights about design. Precisely \emph{because}
there is no code stating that \textbf{turkey} and \textbf{roastBeef} and
ham are sorted into the array, the student is \emph{inferring} that to
be true. And, while that behavior is not what's happening, it
\emph{would} be sensible to design a program where specific instances of
classes \textbf{Turkey} and \textbf{Ham} and \textbf{RoastBeef} could go
into an array of type \textbf{Meat}. To do so, a designer could define
\textbf{Turkey} and \textbf{Ham} and \textbf{RoastBeef} as inheriting
from \textbf{Meat}.

Student3 has an idea about a relationship between entities where that
relationship is not specified in the code. Kaczmarczyk et al.
(Kaczmarczyk et al., 2010) focus only on the downside of the idea: the
student fails to display a proper understanding of how arrays work in
Java. Moreover, the student might be prone to similar mistakes of
inferring information that does not actually exist in code. But, there
is also an upside of this idea. Because Student3 is thinking about
real-world propositions like turkey and ham being kinds of meat, they
might be prepared to appreciate and discuss an object-oriented way to
put turkey in a \textbf{Meat} array. But, that possibility is
speculative conjecture. We can't know for certain whether Student3 could
be tipped into a productive object-oriented design activity around the
meats example because that question was not a focus of the research.

My second example of research that criticizes students' non-canonical
understandings comes from Bonar and Soloway's (1983) study of Pascal
programmers, part of which is discussed in Pea (1986). A student in
Bonar \& Soloway's study was asked to ``Write a program which reads in
ten integers and prints the average of those integers'' (Bonar \&
Soloway, 1983, p. 12). In pseudo-code, she wrote:

Repeat

(1) Read a number (Num)

(1a) Count := Count + 1

(2) Add the number to Sum

(2a) Sum := Sum + Num

(3) until Count :=10

(4) Average := Sum div Num

(5) writeln (`average = `,Average)

The interviewer then asked whether (1a) and (2a) were ``the same kinds
of statements.'' That interchange is reproduced here:

Interviewer: Steps 1a and 2a: are those the same kinds of statements?

Subject: How's that, are they the same \emph{kind}. Ahhh, ummm, not
exactly, because with this {[}1a{]} you are adding - you initialize it
at zero and you're adding one to it {[}points to the right side of 1a{]}
which is just a constant kind of thing.

Interviewer: Yes

Subject: {[}points to 2a{]} Sum, initialized, to, uhh Sum to Sum plus
Num, ahh - thats {[}points to left side of 2a{]} storing two values in
one, two variables {[}points to Sum and Num on the right side of 2a{]}.
That's {[}now points to 1a{]} a counter, that's what keeps the whole
loop under control. Whereas, this thing {[}points to 2a{]} was probably
the most interesting thing\ldots{}about Pascal when I hit it. That you
could have the same, you sorta have the same thing here {[}points to
1a{]}, it was interesting that you cold have, you could save space by
having the Sum re-storing information on the left with two different
things there {[}points to right side of 2a{]}, so I didn't need to have
two. No, they're different to me.

Interviewer: So - in summary, how do you think of 1a?

Subject: I think of this {[}point to 1a{]} as just a constant, something
that keeps the loop under control. And this {[}points to 2a{]} has
something to do with something that you are gonna, that stores more
kinds of information that you are going to take out of the loop with
you. (Bonar \& Soloway, 1983, p. 12)

Pea's (1986) interpretation? ``Here, again, we see the student believing
that the programming language knows more about her intentions than it
possibly can'' (p. 32).

As before, this student has an idea about relationships in code. Pea
(1986) see the downside of her idea: believing PASCAL can understand
shades of programmer intent when, in fact, it cannot. And again, that
downside is real. It could cause trouble for this programmer later on if
she expects PASCAL to interpret her intent and it cannot.

In defense of the student, the question --- as asked --- is vague. Are
those statements the same \emph{to whom and in what way?} Pea (1986)
treats the data as though she meant ``the same to PASCAL.'' Indeed,
maybe she did, in which case his interpretation has traction. But,
another interpretation is that she meant to herself, or to someone else
reading the code. Those statements might not be the same \emph{to her}
because she treats (1a) as having a function of controlling iteration
while (2a)'s job is to combine two numbers into a new sum.

These two purposes, which for the sake of description I'll call
\emph{keeping control} and \emph{totaling up} are, in a sense,
different. The PASCAL compiler (and runtime) does not differentiate
them, but humans can. And, humans may well \emph{want} to differentiate
them. diSessa (1986) describes exactly this kind of differentiation as a
consequence of separating the structural understanding of a programming
language from a functional understanding of a language. As an example,
he discusses the structure/function difference with respect to
variables:

The structural aspects of a variable in a computer language are given
primarily by the rules for setting their values and for getting access
to their values. These rules apply in all contexts. In contrast, a
variable's functions might vary. Sometimes they might be described as
``a flag'' or more generally, as ``a communications device.'' At other
times a variable might function as ``a counter,'' ``data,'' or
``input.'' (diSessa, 1986, p. 202)

The student in Bonar \& Soloway's (1983) study did not show evidence of
understanding the structural similarities between (1a) and (2a) in her
pseudo-code. And, those authors as well as Pea (1986) justly insist that
similarity is important for students to understand. From a conceptual
standpoint, seeing the structural similarity constitutes a part of
``knowing'' PASCAL. But, even if knowing PASCAL were not the goal,
seeing the similarity helps one to take the perspective of a computing
agent that has no means for discerning programmer intent. Such
perspective-taking may help students avoid mistakes that arise from
over-assuming what a computer ``understands.''

The student did show evidence of understanding a functional difference
between (1a) and (2a), but Pea (1986) does not remark on that kind of
understanding at all.\footnote{Also glossed over is, to me, another
  important difference: a programmer might not know in advance which
  numbers are being passed in to the sum statement. So, in advance the
  programmer can say nothing about how the value of \textbf{Sum} will
  change as the loop iterates. In contrast, the programmer knows exactly
  how the value of Count will change with each loop iteration.} Again, I
claim this oversight is part of a subtle but observable trend in
programming misconceptions literature. While, or perhaps because
research has been so preoccupied warring with students' problematic
knowledge, it has sometimes failed to recover the productive knowledge
(or resources for building it) students have. In this example, the
student already has a grasp that syntactically similar statements could
serve different conceptual purposes. Hypothetically, the student might
use that information in design by calling up the ``◻=︎ + 1'' syntactical
template when the situation seems to demand \emph{keeping control},
while calling up ``◻= ︎ + number'' when \emph{totaling up} is the goal.
And, the idea that structurally identical symbol templates can serve
different functional and conceptual purposes fits precisely in line with
Sherin's (2001) theory of symbolic forms. For example, the symbolic
forms \emph{parts-of-a-whole} and \emph{base+change} have different
conceptual schemata. Parts-of-a-whole refers to the contributions of
component entities while base±change describes a kind of accumulation.
Specifically, the terms in base±change ``play different roles'' (Sherin,
2001, p. 534) But, the two distinct conceptual schemata share what I
would argue is the same concrete symbol template of how to write an
expression: ◻=︎ + ◻.

The problem, for learning to program, comes in needing to fluidly
interpret and write code in languages that may demand incommensurable,
or at least distinct, conceptual schemata. As I show later in Table 4,
three current programming languages make remarkably different use of the
plus sign (+) as an operator. Crucially, some of the entailing ways to
make sense of how + works in those languages don't exist in Sherin's
(2001) catalog of conceptual schemata. In other words, I would argue
there are conceptual ways a programmer may need to think about
interpreting or writing a ◻=︎ +◻ symbol template that Sherin doesn't
enumerate.\footnote{One of the most obvious is the movement from seeing
  ``◻=︎ +◻'' as a statement of equality to seeing it as the storage of a
  sum to a variable.} To follow that implication, the canonical body of
knowledge about which programmers must reason is itself fractured,
because different languages design their operations around different
symbolic and conceptual metaphors.

To return to misconceptions, what drives research on students'
misconceptions is largely a need to get students to program computers
and reason about computation in ways that are canonically correct. And,
those are assuredly worthwhile goals. But, as I've argued, we can
already identify cases where a narrow misconceptions focus is silent
about or dismissive of students' useful intuitions. We can also identify
the further problems whereby unilateral emphasis on one language's canon
opposes the vocabulary of symbolic forms (and diverse conceptual
schemata) expert programmers ultimately need (Sherin, 2001) Taken in
total, that silence, dismissiveness, and narrow view of refining
knowledge perpetuates a deficit-focused discourse about student's
knowledge in computing.\footnote{Worse still, in my view, is that there
  is a pattern of researchers failing to discuss or acknowledge
  students' productive knowledge \emph{even when their own data supports
  it}.} Such a perspective also fails to address what aspects of
students reasoning might get broken by attempts to ``fix'' such
``misconceptions''.

It's important to note that misconceptions research in computing didn't
always treat students' non-canonical knowledge this way. I begin the
next section by backtracing to some of the earliest work on students'
cognitive ``bugs''. There, we find researchers talking more explicitly
about what's useful in students' non-canonical knowledge---a stance
largely absent in modern computing misconceptions research.

\subsubsection{Not all cognitive programming bugs imply a problem with
the
student}\label{not-all-cognitive-programming-bugs-imply-a-problem-with-the-student}

What's curious about Pea's (1986) comments on Bonar \& Soloway (1983) is
that Pea drew different conclusions from the data than Bonar and Soloway
did. Pea (1986) emphasized that when the student thought two
semantically-equivalent assignment statements in PASCAL were different,
it was a problem of egocentrism: ``students assume that there is
\emph{more} of their meaning for what they want to accomplish in the
program than is actually present in the code they have written'' (p.
30). In other words, Pea treated the data as a fairly clear example of a
class of bugs. Specifically, he saw a bug class in which students simply
assumed the interpreter or runtime could infer the code author's shades
of intent.

Bonar and Soloway (1983), by contrast, were less quick to make
inferences either about the nature of the bug or about the intervention
it entailed. Rather than jump to definitive conclusions, they were
circumspect:

It is not clear exactly how to react to the bugs we have uncovered in
novice understanding of programming. In some cases it may be appropriate
to design new languages or constructs. Often, better instruction would
take care of the problem. The intent of our studies is to better
understand the source of the mismatches and misconceptions that cause
novice bugs. Only once a bug is uncovered and understood are we ready to
create a remedy for that bug. (Bonar \& Soloway, 1983, p. 12)

That the authors would even consider developing new constructs or
languages is a noteworthy distinction. Rather than assume in toto that
students with ``bugs'' had wrong knowledge, the authors instead suppose
cognitive bugs have plausible origins worth designing around. Moreover,
they treat students' divergence from canon as \emph{an opportunity for
research to learn from students}. Precisely because students saw
functional differences (cf., diSessa, 1986) in semantically-equivalent
PASCAL statements, Bonar and Soloway (1983) reflect that perhaps
programming languages should be more expressive:

We find it quite interesting that novices seem to understand the role or
strategy of statements more clearly than the standard semantics. Such
roles discussed here include ``counter variable,'' ``running total
variable,'' ``running total loop,'' and ``first, then rest loop''. (See
Soloway et al {[}1982b{]} for a detailed discussion of novice looping
strategies.) Much work in programming languages is concerned with
allowing a programmer to more accurately express his or her intentions
in the program. Perhaps we can learn something from novices here - our
programming systems should support recording the roles the programmer
intends for various statements and variables. (Bonar \& Soloway, 1983,
p. 12)

Again, what's noteworthy here is that rather than treating students'
non-canonical views as a burden for instruction, Bonar and Soloway
instead see them as an opportunity for programming language designers to
make languages better.

That insight---that instruction and design can meet novices where they
are---carries through to their final remarks about studying and
analyzing novice programming knowledge:

The experience and understanding of a novice are available for analysis.
In particular, our results suggest that the knowledge people bring from
natural language has a key effect on their early programming efforts.
Our work suggests that we need serious study of the knowledge novices
bring to a computing system. For most computerized tasks there is some
model that a novice will use in his or her first attempts. We need to
understand when it is appropriate to appeal to this model, and, when
necessary, how to move a novice to some more appropriate model. (Bonar
\& Soloway, 1983, p. 13)

I assume Bonar and Soloway structured their final line deliberately. If
so, their phrasing has three consequences:

\begin{enumerate}
\def\labelenumi{\arabic{enumi}.}
\item
  Appealing to novice's existing models gets precedence. That is,
  understanding how to leverage novices' existing knowledge comes first
  in research.
\item
  Changing the models novices have comes next, and ``when necessary.''
\item
  They speak of ``how to move a novice to some more appropriate model,''
  which does not necessarily entail the removal or destruction of
  novice's existing models.
\end{enumerate}

Taken together, these points convey a sense of how Bonar and Soloway
view learning and instruction in computing. Instruction explicitly
includes appeals to prior models and knowledge students might already
have. Learning, meanwhile, involves the movement \emph{when necessary}
to more appropriate models of computation. As I explain in the next
section, such a view exactly aligns with a particular branch of
constructivism, where cognition is viewed as the complex activation of
manifold resources for thinking and knowing.

\subsubsection{Examples motivate the need for contextual-sensitivity in
modeling programming
cognition}\label{examples-motivate-the-need-for-contextual-sensitivity-in-modeling-programming-cognition}

Let's begin with two motivating examples. My aim with these examples is
to show how a practicing programmer might employ specific, distinct
conceptual models to reason locally about a piece of code. First,
consider variants of the PASCAL statements from Bonar and Soloway
(1983), where an assignment statement worked to increment a value and
store the result back to that value. In Table 2 below, I imagine five
different ways of writing a programming statement. Each of the five ways
is semantically equivalent to the others (or near enough for explanatory
purposes). And, in each example I add a layer of specificity to the
syntax. I also propose a corresponding interpretation of how I might
apply interpret and think about a given statement.\footnote{I assume for
  simplicity's sake that all variables and functions in my code samples
  evaluate to floating-point numbers that can be combined under
  addition. I also assume, crucially, that all statements are in the
  same hypothetical language and that the variable names connote nothing
  to the computational interpreter.}

\protect\hypertarget{ux5fToc252445956}{}{}Table -- Five different
interpretations of semantically equivalent programming statements in the
same language

% \begin{center}
%
%
% \begin{supertabular}[]{@{}ll@{}}
% \toprule
% Statement & Syntax\tabularnewline
% \midrule
% \endhead
% 1 & \texttt{x\ =\ x\ +\ d}\tabularnewline
% 2 & \texttt{x\ =\ x\ +\ update}\tabularnewline
% 3 & \texttt{x\_position\ =\ x\_position\ +\ update}\tabularnewline
% 4 & \texttt{x\_position\ =\ x\_position\ +\ update(t)}\tabularnewline
% 5 & \texttt{x\_position\ =\ x\_position\ +\ update(...)}\tabularnewline
% \bottomrule
% \end{supertabular}
% \end{center}
\begin{center}
  \begin{tabular}{ | l | l |}
    1 & \texttt{x\ =\ x\ +\ d} \\
    2 & \texttt{x\ =\ x\ +\ update} \\
    3 & \texttt{x\_position\ =\ x\_position\ +\ update} \\
    4 & \texttt{x\_position\ =\ x\_position\ +\ update(t)} \\
    5 & \texttt{x\_position\ =\ x\_position\ +\ update(...)} \\
  \end{tabular}
\end{center}


% I stress that the third column is about \emph{how I might think about}
% each programming statement\emph{.} By no means am I making normative
% claims about how one \emph{ought} to think about it, or whether the
% statement actually does what its names might suggest it does. Rather,
% ``How I might think about it'' reflects the kind of local meaning or
% interpretation I might attach to such a statement when I work with it,
% given my understanding of its role and context.
%
% Consider specifically statements 1 and 4. Statement 1 could very well be
% an incrementer in some kind of iterative code. If I had to debug code
% that employs statement 1, what I might do is exploit my knowledge of
% what d is (does it hold a constant value? Is it 1?) and try inserting
% intermediate print statements into the iterative code. Doing so, I can
% inspect textually how values change with each iteration. In statement
% 4's case, it could very well be that the position-update code animates
% images on a screen. If there's a problem with that code, one of the
% easiest ways I might notice is that the acceleration seems off in the
% graphics. Consequently, my debugging might call upon knowledge I have
% from kinematics. I might try inserting code that draws a dot at the
% object's on-screen position with each iteration, leaving a trail I can
% visually inspect. I could then look at the path of the object's
% trajectory and the spacing patterns between successive dots as a
% first-pass test of whether my code achieves the motion I want.
%
% To be clear, it's not just that graphics might improve my efficiency in
% debugging a statement. Rather, applying an interpretive frame that
% treats an assignment statement as a kinematic position update lets me
% \emph{use conceptual knowledge from physics} to diagnose and fix
% problems in my code. If I view the assignment statement as saying
% something about the motion of an object (cf., Hammer, 1994, p. 165), a
% field of knowledge and concomitant techniques from physics becomes
% available to me to think with. But, if I focus only on the
% semantic-level equivalence of statements 1 through 5, there would be no
% obvious reason for me to access what I know about physics in order to
% reason about the code.
%
% My second example concerns statements that are syntactically-similar,
% rather than semantically-similar. I proposed that the statements in
% Table 2 all came from the same language. But, another phenomenon comes
% into play when different languages use the same symbology for
% conceptually different operations. Table 3 shows examples of what look
% like semantically-equivalent operations, but in fact are not.
%
% \protect\hypertarget{ux5fToc252445957}{}{}Table -- Three
% syntactically-similar statements with very different semantics
%
% \begin{supertabular}[]{@{}llll@{}}
% \toprule
% \textbf{Statement } & Programming Syntax & Language & How I think about
% it\tabularnewline
% \midrule
% \endhead
% \textbf{1} & i = i + 1 & C & \textbf{Increments} i by 1\tabularnewline
% \textbf{2} & w = w + ``ly'' & JavaScript & \textbf{Appends} ``ly'' to
% the string w\tabularnewline
% \textbf{3} & p = p + geom\_point() & R (ggplot2) & \textbf{Composes} a
% layer of points onto a plot p\tabularnewline
% \bottomrule
% \end{supertabular}
%
% The catch here is that the plus operator takes on different roles in
% different languages because of how those languages define its use.
% Statement 1 increments a number in C; Statement 2 appends the letters
% ``ly'' to a string; Statement 3 adds a layer of points to a statistical
% graphics plot. These different kinds of operations become even more
% apparent and consequential when, for example, such statements are
% repeated. In the R/ggplot2 code below,\footnote{In R, the assignment
%   operator can be written as a directional arrow. The symbol
%   \textless{}- (``less-than, hyphen'') indicates the value on the right
%   side of the symbol is being assigned to the variable on the left side
%   of the symbol. As used, the symbol itself is semantically equivalent
%   to having written an equals sign (``=``).} I'm using multiple
% reassignment statements to compose a statistical graphics plot. The
% layered creation of a plot invites a very different kind of conceptual
% interpretation than, say, repeatedly accumulating numbers into a running
% sum:
%
% p \textless{}- InitializeGgplot\_1w()
%
% p \textless{}- p + GrandMeanLine(owp)
%
% p \textless{}- p + GrandMeanPoint(owp)
%
% p \textless{}- p + ScaleX\_1w(owp)
%
% p \textless{}- p + ScaleY\_1w(owp)
%
% p \textless{}- p + JitteredScoresByGroupContrast(owp, jj)
%
% One obvious reason for thinking about this code with a different
% interpretive frame is that numeric addition is commutative; composing a
% plot is not necessarily commutative.\footnote{To convince yourself that
%   plot composition isn't commutative, imagine a scale function that
%   squares up the aspect ratio of a plot and another scale function that
%   sets the aspect ratio to 1.5:1. Applying the square function last
%   produces a square plot; applying it first produces a rectangular plot.
%   String concatenation is also not necessarily commutative. ``cat'' +
%   ``dog'' evaluates to ``catdog,'' while ``dog'' + ``cat'' evaluates to
%   ``dogcat.''} So, despite the syntactic similarity, reassignments that
% compose a plot using reassignment (as above) does not obey the same
% rules as reassignments for a running total. But, even within this code
% block, statements that look alike perform operations of a different
% nature. While some expressions (e.g, ``+ GrandMeanLine(owp)'') add a
% visual layer to a plot, others modify features of the plot (e.g, ``+
% Scale\_X(owp)'', which adjusts the scales on the plot's x-axis to fit
% the numeric range of the data).
%
% Given these motivating examples, it seems sensible to think there's
% utility in a programmer having different conceptual metaphors available
% to think about and work with code. Example 1 shows that choosing to
% apply knowledge from physics to a piece of code can change the cognitive
% nature of debugging. There, debugging a position-update statement
% becomes, in part, reasoning kinematically about the properties of motion
% trails.\footnote{For comparison, consider the argument that ringing an
%   aircraft speedometer with physical markers changes the nature of
%   cognition when pilots work to land a plane (Hutchins, 1995b).} Example
% 2 shows that across languages, programmers might have to deploy
% different conceptual metaphors to reason about statements in a
% locally-consistent way. Knowing that plots in ggplot2 can be composed
% layer-by-layer with reassignment is crucial if you're trying to write or
% understand code that creates statistical graphics. But, I would argue
% that thinking about ◻ = ◻ + ◻ as ``compose new layer onto plot'' can and
% does appeal to different kinds of knowledge when compared to thinking
% about ◻ = ◻ + ◻ as ``include this addend in the sum,'' which itself can
% and does appeal to different kinds of concepts when compared to thinking
% about ◻ = ◻ + ◻ as ``increment the counter.''
%
% Stepping back, I can build the following argument
%
% \begin{enumerate}
% \def\labelenumi{\arabic{enumi}.}
% \item
%   Programming can be helped by applying conceptual models to code,
%   particularly when relevant domain-knowledge structures can
%   advantageously transform a problem (example 1)
% \item
%   But, conceptual models don't work all the time for all statements.
%   Because languages are designed differently, the same syntax can
%   actually correspond to very different operations in code (example 2).
%   And, that's true both within and across languages.
% \item
%   Consequently, it makes less sense to treat conceptual models as right
%   or wrong, and more sense to treat them as differentially advantageous
%   for thinking about what a piece of code does. (Thinking a ``+''
%   implies numerical addition isn't \emph{globally wrong} in JavaScript,
%   but it won't explain why 1 + ``1'' yields ``11'' as a result.)
% \item
%   It seems plausible that successful programmers, when reasoning about
%   or writing code, are able to dynamically access or deploy conceptual
%   models that are advantageous given the context (language, syntax,
%   surrounding code). Certainly such a supposition is in line with work
%   suggesting students have resources for thinking conceptually and
%   dynamically about how mathematics models real-world situations (Izsák,
%   2004; Sherin, 2001)
% \item
%   To model how programmers think with conceptual models, a suitable
%   framework should be able to account for the dynamic, context-sensitive
%   deployment of conceptual knowledge.
% \item
%   To model how programmers develop expertise, a suitable framework
%   should be able to describe higher-order phenomena. Such phenomena
%   include explaining how programmers come to have conceptual models or
%   generate new ones, why they decide to deploy them, and how programmers
%   consider which conceptual model (i.e., which way to think about code)
%   is appropriate.
% \end{enumerate}
%
% Taken together, these assertions propose criteria for how we might
% strive to model cognition in programming. Our modeling frameworks should
% be context-dependent, dynamic, and capable of explaining where
% conceptual models come from. They should also be able to account for
% phenomena that are not themselves conceptual, including what directs the
% use of certain kinds of conceptual knowledge. In the learning sciences,
% such models already exist and have proven useful and productive for
% thinking about thinking.
%
% \subsubsection{Manifold models of cognition explain context-dependence
% and the growth of
% expertise}\label{manifold-models-of-cognition-explain-context-dependence-and-the-growth-of-expertise}
%
% In 1993, a pair of articles in the learning sciences staked a strong
% claim for viewing knowledge as a network of pieces, isolated enough to
% be locally triggered but trainable enough to fire in larger concerted
% patterns (diSessa, 1993; Smith, diSessa, \& Roschelle, 1993). Informed
% in part by agent-based accounts of cognition (Minsky, 1986) and complex
% systems models (diSessa, 2002), the central tenets of an ``in-pieces''
% approach hold that knowledge is emergent from interacting primitives,
% rather than unitary and monolithic. An example from Smith, diSessa, and
% Roschelle helps illustrate the point.
%
% The authors show that we might think of a rubber band as a different
% conceptual entity depending on context. In several different
% situations---wrapped around a newspaper, pulled taut as a string, spun
% to store energy in a toy plane propeller---we intuitively think about
% the rubber band's physical behavior differently: one as a negligible
% part of the newspaper's point mass, another as a transverse pendulum
% (likely) obeying Hooke's Law, and the third as a torsional spring. Those
% differences in intuitive thinking reflect the contextual dependency of
% what we know about the physical world:
%
% In each of the rubberband examples, various pieces of intuitive physical
% knowledge describe the mechanism at work: the rubber band binds the
% newspaper, grips the jar lid, and acts a source of springiness for the
% bobbing object. Although a mapping cannot be made from the rubberband to
% scientific entities, it is quite easy to map these qualitatively
% distinct physical processes to scientific entities and laws. For
% example, instances of binding almost always map to a practically rigid
% body. Likewise, gripping maps to friction forces, and springiness maps
% to Hooke's law. This suggests that applicability can depend directly on
% our intuitive knowledge---knowledge that exists prior to any formal
% scientific training (Smith et al., 1993, p. 144).
%
% The in-pieces approach to modeling cognition has been used, among other
% things, to explain how experts reason about fractions and decimals
% (Smith et al., 1993), how students reason about forces in physics
% (diSessa \& Sherin, 1998; diSessa, 1993; Hammer, 1996; Sherin, 2001),
% how students construct and evaluate algebraic representations of
% physical situations (Izsák, 2004), and how knowledge transfers across
% contexts (Hammer et al., 2005; Wagner, 2006). Because its starting
% assumption is that knowledge is fragmented, knowledge-in-pieces can
% account for wide variations of how people---particularly novices---use
% knowledge on a moment-to-moment basis. In other words, because it
% assumes knowledge is local, it can still explain the kinds of
% globally-inconsistent ways people might reason about physical situations
% (diSessa, 1993). As a framework, an in-pieces approach ultimately argues
% that models of concept replacement and good/bad criteria for knowledge
% should be supplanted by a learning model of alignment/refinement of
% prior knowledge and the consideration of knowledge as
% productive/unproductive.
%
% Hammer and colleagues have worked to extend the in-pieces approach to
% explain how students' epistemological activity---how they orient toward
% knowledge and knowing in a context (Hammer et al., 2005; Hammer \& Elby,
% 2002, 2003). Specifically, those authors use two core theoretical
% constructs to explain students' stances toward knowledge and knowing:
%
% \begin{itemize}
% \item
%   \emph{Epistemological Resources} are the epistemological equivalent of
%   diSessa's phenomenological primitives (p-prims). Resources, the
%   authors propose, are the atomic units involved in how people cognize
%   about the source of knowledge, the nature of knowledge, and
%   epistemological activities (Hammer \& Elby, 2003; Louca, Elby, Hammer,
%   \& Kagey, 2004).
% \item
%   \emph{Epistemological Frames} are the emergent result of subsets of
%   resources acting in concert. Drawing from both Goffman's (1974)
%   sociological notion of frame as structures of expectations and
%   subsequent work on framing in discourse (Tannen, 1993),
%   epistemological frames are a participant's local answer to the
%   question ``what is it {[}specifically, what knowledge activity{]}
%   that's going on here'' (Goffman, 1974, p. 8).
% \end{itemize}
%
% Resources can frames can interact in activity settings to produce
% larger-scale patterns called ``epistemological coherences'' (Rosenberg,
% Hammer, \& Phelan, 2006) where evidence from data suggests that a
% network of discrete cognitive units can nonetheless give rise to stable
% cognition.
%
% A useful conceptual metaphor for resources and framing is to think about
% a lecture hall with different sets of lights: a spotlight in the back to
% highlight a lecturer, chalkboard lights, house lights, and a projector.
% Each light has an individual brightness, but the lights are only
% controllable at the per-bulb level. In that sense, they are atomic. But,
% light patterns interact with one another to create a field of lighting
% for the room. Thus, lights are a bit like resources, and different light
% configurations can correspond to different, sociologically-stable uses
% of the room.
%
% \protect\hypertarget{ux5fToc252445958}{}{}Table -- Using room lighting
% configurations to think about epistemological framing
%
% \begin{supertabular}[]{@{}lllll@{}}
% \toprule
% \textbf{What is it that's going on?} & Houselights & Spotlight &
% Chalkboard & Projector\tabularnewline
% \midrule
% \endhead
% \textbf{On-stage monologue} & Low & ON & Low & OFF\tabularnewline
% \textbf{Presenting slides} & Low & ON & Low & ON\tabularnewline
% \textbf{Working through equations} & Low & ON & High &
% OFF\tabularnewline
% \textbf{Students discussing with each other} & High & OFF & Low &
% OFF\tabularnewline
% \textbf{Watching a movie} & Low & OFF & Low & ON\tabularnewline
% \textbf{Cleaning the room after a movie} & High & OFF & High &
% OFF\tabularnewline
% \bottomrule
% \end{supertabular}
%
% To convince yourself of the sociological stability of these
% configurations, imagine you were watching a movie when suddenly the
% house and chalkboard lights came up to full intensity. It would jar you
% out of the experience. You might wonder whether something was wrong: is
% there an emergency? Should you evacuate?
%
% The metaphor isn't perfect. Ontologically, for example, resource and
% framing theory propose these constructs exist not in the world but in
% the minds of individuals. But, the metaphor is quite useful for
% understanding how individual elements---in this case lights---can work
% in concert to create and sustain a stable frame. And, crucially for a
% cognitive system, the metaphor lets us account for variation in
% activity. The same lecture hall can become a window onto one person's
% thoughts (monologue), a site for instruction (working through equations
% at the board), a shared space for collaboration (student discussion), or
% a place in need of repair (cleaning up) simply by varying the
% intensities of lighting banks in particular ways.
%
% An example helps ground this in-pieces approach to epistemology. In
% Russ, Coffey, Hammer, and Hutchison (2008), the authors describe the
% situation where an elementary student reasons about why an empty juice
% box collapses when you suck on the straw. One student gives what the
% authors deem to be an excellent mechanistic account of why the juice box
% collapses. But, as the teacher seems to steer the discussion toward
% vocabulary---in this case, ``pressure''---the student clearly pulls back
% from her mechanistic reasoning, seems much more diffident, and claims
% that pressure is hard to explain. That example highlights the disconnect
% between doing science as knowing vocabulary and doing science as
% reasoning mechanistically. Moreover, it strikingly highlights that a
% student who by all accounts produced an excellent explanation of how
% pressure works was left nonetheless with the impression that pressure
% was hard to explain.\footnote{In fairness to the teacher, I'm trying to
%   focus on the result of the interaction and far less on the intent the
%   teacher might have had. The student still left with a sense that
%   science might be about vocabulary, even if that's not the view the
%   teacher would espouse or was trying to enact in the moment.}
%
% \subsection{Methods and Theoretical
% Commitments}\label{methods-and-theoretical-commitments}
%
% \subsubsection{Student population, course background, and selection
% }\label{student-population-course-background-and-selection}
%
% This paper focuses on Electrical Engineering (EE) students from Flagship
% State, a large public research institution in the mid-Atlantic. The
% students I studied were taking Intermediate Programming, the second
% semester of an introductory programming course taught in C. The course
% was exclusively for EE majors and taught by EE faculty, and its
% enrollees were typically first-year or second-year EE majors. Its model
% was two 75-minute lectures and a 1-hour teaching assistant-led
% discussion section for students each week. Course topics included:
%
% \begin{itemize}
% \item
%   Strings
% \item
%   Pointers
% \item
%   Dynamic Memory Allocation
% \item
%   Testing
% \item
%   Debugging
% \item
%   Hash tables
% \item
%   Trees
% \item
%   Linked Lists
% \item
%   Abstract Data Types
% \item
%   Functional Decomposition
% \end{itemize}
%
% Students in the course had varying degrees of experience with
% programming. Partly, that variation is because students with Advanced
% Placement (AP) Computer Science credit could place out of Basic
% Programming, the first semester C course. So, some students in
% Intermediate Programming came from technical magnet schools where they
% may have already had one or more years of programming in multiple
% languages, while other students may have been first-time programmers
% with only one semester of programming experience: Basic Programming.
%
% I studied the same course, Intermediate Programming, during the fall
% 2011 semester and the spring 2012 semester. During fall 2011 I
% ethnographically observed over 50\% of the course lectures, and during
% spring 2012 I sat in on several discussion sections. At the beginning of
% each semester I solicited student participants for semi-structured
% clinical interviews. Of the students who responded to the solicitation,
% I scheduled interviews opportunistically with students given my
% resources as the sole researcher on the project. In total, I worked with
% a cohort of 6 students in fall 2011 and 4 students in spring 2012.
%
% My specific interest was in how students design computer programs. By
% that, I mean I wanted to know not just how students program, but whether
% and how they structured programs, how they did (or did not) try to
% incorporate modularity into their programs, and how the final form of a
% program reflected what they had learned about how to manage complexity
% through software. In the sections that follow I outline the palette of
% methods I used to pursue those questions.\footnote{The language in
%   sections 3.3.2 and 3.3.3 is taken with slight modification, from
%   ``Studying students' early-stage software design practices,'' a paper
%   I authored, along with William Doane, and submitted to the 2014
%   International Conference of the Learning Sciences (ICLS).}
%
% \subsubsection{Studying design as a complex phenomenon of disciplinary
% practice}\label{studying-design-as-a-complex-phenomenon-of-disciplinary-practice}
%
% Practicing professionals make complex use of talk, gesture, and
% representational artifacts in physics (Gupta, Hammer, \& Redish, 2010;
% Kaiser, 2005; Ochs, Gonzales, \& Jacoby, 1996); chemistry (Stieff,
% 2007); field biology (Hall, Stevens, \& Torralba, 2002); civil
% engineering (Stevens \& Hall, 1998); structural engineering (Gainsburg,
% 2006); mechanical and electrical engineering (Bucciarelli, 1994;
% Henderson, 1999); and architectural design (Hall et al., 2002; Hall,
% 1999). Given that:
%
% \begin{enumerate}
% \def\labelenumi{\arabic{enumi}.}
% \item
%   Science and engineering education research has made progress by
%   looking for continuities in how novice learners develop disciplinary
%   practices (Gupta et al., 2010; Hall \& Stevens, 1995; Smith et al.,
%   1993; Stevens \& Hall, 1998), and
% \item
%   Emerging research on software engineering reveals that early-stage
%   software design involves complex inscriptional, discursively, and
%   epistemic practices,
% \end{enumerate}
%
% it seems striking that there is no contemporary body of research,
% comparable to studies of expert practice, that looks at students'
% software design practices. In other words, we have every reason to
% believe that expertise in software design involves complex practices,
% but little (if any) research on what productive capacities students have
% for them. Finding those productive design capacities requires a shift
% from questions such as \emph{how can we assess and mitigate students'
% difficulty in programming?} toward questions such as \emph{how do
% students learn and display evidence of design thinking in programming?}
%
% Rephrasing research questions asked of professional software engineers
% (Petre et al., 2010, p. 533) and instead treating \emph{students} as
% designers, I ask:
%
% \begin{itemize}
% \item
%   What do students actually do during early stage software design work?
% \item
%   What does students' exploratory design thinking look like?
% \item
%   How do students communicate?
% \item
%   What sorts of drawings do students create when they design software?
% \item
%   What kinds of strategies do students apply in exploring the vast space
%   of possible software designs?
% \end{itemize}
%
% \subsubsection{Deploying methods to capture the complexity of
% early-stage design
% work}\label{deploying-methods-to-capture-the-complexity-of-early-stage-design-work}
%
% The methods below form the core of my developing program to study
% students as software designers. None of the methods below are new; all
% have been used in prior educational research. What \emph{is} new, I
% believe, is the opportunity to combine them all under the umbrella of
% understanding what happens when students design software. For each of
% the four students in my spring 2012 cohort I captured multiple streams
% of data.
%
% \begin{itemize}
% \item
%   I collected their \textbf{code history}. By this, I mean I preserved
%   frequent snapshots in time of what students' code looks like. Research
%   has already shown code snapshotting to be a useful method for
%   understanding large-scale patterns of student error (Jadud, 2006;
%   Rodrigo et al., 2009; Spacco, Hovemeyer, et al., 2006). And, the
%   resolution of snapshots is extremely fine: Spacco et al. are able to
%   capture the contents of a file each and every time a student saves it
%   to disk. But, that research takes an aggregate view: it identifies
%   large-scale error patterns at the expense of detailed naturalistic
%   understandings of why students make those errors. Moreover, it's
%   primarily used to identify what \emph{mistakes} students make, which
%   is distinctly different from a research orientation that considers the
%   negative \emph{and} positive consequences of students' design choices.
%   A currently untapped advantage of collecting code history data, then,
%   is that while we historically use it to aggregate \emph{across
%   programmers} it nevertheless also gives us fine-grained individual or
%   team-based histories of how designs evolve.
% \item
%   I conducted \textbf{clinical interviews} with them. Clinical
%   interviews have proven historically useful in understanding the
%   substance of students' knowledge and the nature of conceptual change
%   (diSessa \& Sherin, 1998; Duckworth, 2006; Ginsburg \& Opper, 1988;
%   Ginsburg, 1997). Crucially for Computer Science Education (CSEd)
%   research, clinical interviews can tell us about students'
%   epistemologies---how they view knowledge and knowing in a discipline
%   (Hofer \& Pintrich, 1997)---which can affect how they approach and
%   adopt that discipline's practices (Hammer, 1989, 1994; Lising \& Elby,
%   2005). Moreover, because my interviews were videotaped and analyzed
%   from perspectives of interaction analysis (Goodwin, 2000; Jordan \&
%   Henderson, 1995), they offer rich evidence of the substance of
%   students' design practices
% \item
%   I analyzed their in-interview \textbf{inscriptions} when they designed
%   --- what they wrote, how they wrote it, and how those writings got
%   used. Evidence from both science studies (Hall et al., 2002;
%   Henderson, 1999; Hutchins, 1995a; Kaiser, 2005; Latour, 1990; Ochs et
%   al., 1996) and educational research (Hall \& Stevens, 1995; Lehrer,
%   Schauble, Carpenter, \& Penner, 2000; Stevens \& Hall, 1998)
%   highlights the centrality of inscriptions to disciplinary practice in
%   science and mathematics. Ethnomethdological data from studies of
%   professional software engineers supports the same result:
%   inscriptional practice is central to how professional engineers design
%   software (Rooksby \& Ikeya, 2012; Rooksby, 2010). And, since part of
%   the inscriptional environment when designing software is the computer
%   itself, I captured and analyzed what happened on-screen as students
%   design programs.
% \item
%   I paid close attention to students' in-interview \textbf{gestures}.
%   Perspectives of gestural analysis hold that gestures can not only
%   support or extend thinking, they can also communicate entirely
%   different information than what's being said (Goldin-Meadow, 2003).
%   Moreover, perspectives from cognitive anthropology and embodied
%   cognition studies argue that bodily motion \emph{is itself} cognition
%   (Hall \& Nemirovsky, 2012; Hutchins, 1995a; Nemirovsky, Rasmussen,
%   Sweeney, \& Wawro, 2012). For example, when students describe a part
%   of code by moving their hand across an imaginary row of items and
%   tapping each item, their bodies convey information we can interpret
%   about how they understand iteration.
% \end{itemize}
%
% Figure 3-1 presents a visual overview of some of these methods and modes
% of analysis. In particular, the first (top-left) panel depicts how we
% deploy these data collection methods during a clinical interview:
%
% \begin{itemize}
% \item
%   a voice recorder captures speech (often a redundancy in case another
%   recording device fails)
% \item
%   a LiveScribe Pulse pen digitizes written inscriptions, allowing us to
%   play back what was written in time
% \item
%   a videocamera records data for knowledge analysis, interaction
%   analysis, and gestural analysis,
% \item
%   an in-interview computer tracks code history and its screen-capturing
%   software records real-time activity.
% \end{itemize}
%
% \includegraphics[width=6.25556in,height=4.20000in]{media/image4.jpeg}
%
% \protect\hypertarget{ux5fToc252445967}{}{}Figure ‑ -- An comic-based
% overview of my methods for capturing students' design practices
%
% \subsubsection{Developing a revelatory case of students' early-stage
% design
% work}\label{developing-a-revelatory-case-of-students-early-stage-design-work}
%
% In the empirical work that follows I explore what productive knowledge
% and resources students have for structuring programs. I begin with
% Lionel, a male student in Intermediate Programming, fall 2011. First, I
% propose the idea that Lionel connects ``designerly'' (Archer, 1979)
% stances across contexts. I analyze his retrospective account of
% modifying a bike and compare it to his retrospective accounts of how he
% programs in Intermediate Programming. Then I detail Lionel's multi-stage
% cascading approach to structuring a program solution in an interview.
% The argument throughout is that from bike design to code design Lionel
% evinces knowledge and stances that productively help him solve problems.
%
% Next I transition to Rebecca, first explaining the epistemological
% difficulty she felt when coding, then showing how she circumvented that
% block by structuring a program piece using gesture and talk. I argue
% that what Rebecca and Lionel have in common are resources for handling
% design obstacles as they emerge. Where the two students differ is in
% their in-the-moment judgments about what sorts of knowledge and
% activities are appropriate to work through a design problem.
%
% I argue that taken together, Rebecca and Lionel constitute what Yin
% (2009) calls a ``revelatory case,'' by which I mean my investigation
% arises from ``an opportunity to observe and analyze a phenomenon
% previously inaccessible to social science inquiry.'' In my case, I argue
% the phenomenon I'm observing is what engineering students do in the
% early stages of design work on a program. Its inaccessibility is
% demonstrated by the almost complete paucity of studies that draw on
% real-time (e.g., video records) of how students begin work on a complex
% program. Contrary to being ``representative'' or ``typical,'' my
% revelatory case does not aim to generalize. Rather, its value is in
% phenomenological richness. As Yin (2009) writes, and I think others
% including Erickson (1986) would agree, such case studies are worth doing
% ``because the descriptive information alone will be revelatory.''
%
% \subsection{\texorpdfstring{\protect\hypertarget{ux5fToc247188553}{}{\protect\hypertarget{ux5fToc252445926}{}{}}Lionel's
% approach to design and to
% programming}{Lionel's approach to design and to programming}}\label{lionels-approach-to-design-and-to-programming}
%
% \subsubsection{While modifying a bike Lionel decomposed tasks, explained
% solutions to himself, and created intermediate design
% representations}\label{while-modifying-a-bike-lionel-decomposed-tasks-explained-solutions-to-himself-and-created-intermediate-design-representations}
%
% Prior to entering my study, Lionel had already earned a Bachelor's
% degree in Business Information Technology at a different university. We
% began the interview by discussing why he re-matriculated to earn an
% electrical engineering degree.
%
% Interviewer: How did you, um end up coming out of your business program
% and your business experience and deciding that electrical engineering
% was what you wanted?
%
% Lionel: Um, well, since I was Business Information Technology, basically
% all my job was doing was writing SQL code and, you know, short
% statements like that. I also worked at a help desk for---I guess I
% worked at Department of Homeland Security. And, pretty much besides the
% SQL statements I worked with this one program and whenever somebody had
% a problem with it they'd come to me. And basically that---that's what my
% whole degree was---was going down that track and then that's really not
% what I wanted to do: just sitting at a desk, you know people coming to
% me being like ``oh, you gotta do this for me.'' So, I was like, alright.
% Or, I have to fix a problem for them, pretty much. And that's really
% what I didn't want to do. I wanted to do something more hands-on where I
% could, you know, I'd be given a project where I'd do something on my
% own, create something on my own. And, um, also my Dad and my uncle were
% both electrical engineers. And my brother became a computer scientist.
% Cu---so I've always been around, you know, that type of environment,
% where my dad's always building something and, you know, I always see it
% and I'm like, you know, that's very interesting. I'd like to do
% something like that too. So it's really just uh, a mixture of bad
% experience with my previous job and being around company that likes to
% do that sort of stuff. (Interview 1 of 1, October 17, 2011)
%
% Lionel apprenticed building things with his father first, taking on
% small, highly-directed roles in projects.
%
% Lionel: Be-before I started my degree, you know, I really didn't know
% how to do any of this stuff. So, I would just kinda help my dad out a
% little bit. He'd be like ``Oh, here,'' you know, ``to, to do this you
% have to do exactly this.'' You know, ``solder this here,'' or, you know,
% ``drill this,'' blah blah blah. He'd tell me exactly what to do. So, I
% just kind of followed along. (Interview 1 of 1, October 17, 2011)
%
% Presumably in those projects Lionel was at a periphery and direction
% came from his father. More recently, though, Lionel had struck out on
% projects of his own. As Lionel said, ``now that uh, my dad kinda showed
% me how to do certain typical types of things, I've tried to become
% \textbar{}\textbar{}more adventurous\textbar{}\textbar{} \textbar{}\{air
% quotes\}\textbar{} /Sure/ and do things on my own'' (Interview 1 of 1,
% October 17, 2011).
%
% One of Lionel's recent adventurous projects was adding a motor to a
% beach cruiser bicycle. The gist of the project was simple: using a
% low-priced motor he found online, Lionel realized he could mount the
% motor to an inexpensive bicycle and make a motorized bike. With one
% exception, Lionel was solely responsible for the project from start to
% finish.\footnote{Lionel told me that the only part his father helped him
%   with was showing him how to use a Dremel power tool to smooth, sand,
%   and cut edges (Interview 1 of 1, October 17, 2011).} He bought the
% parts, he mounted the motor, and he maintained it as a working vehicle.
% But, in practice the project was not straightforward. In the next
% subsections I show at length how Lionel addressed the emergent
% challenges of modifying\footnote{Often called ``modding'', for short.}
% his bike with a motor. My analysis focuses on three features of Lionel's
% story that I argue carry over to his programming approach:
%
% \begin{enumerate}
% \def\labelenumi{\arabic{enumi}.}
% \item
%   When faced with ambiguity, Lionel worked to decompose large tasks into
%   smaller subtasks he could understand.
% \item
%   Throughout the task, Lionel was metacognitive about the state of his
%   design. His constant refrain was ``\emph{how} am I gonna make this
%   work?''
% \item
%   Before committing work and materials to a solution, Lionel created
%   intermediate designs whose suitability he could evaluate.
% \end{enumerate}
%
% I argue in later sections that these three features of Lionel's
% approach---decomposing subtasks, explaining procedures to himself, and
% explicitly planning prior to committing time and resources to a final
% product---have strong resonances with the approach Lionel took to while
% programming.
%
% \paragraph{Lionel had to decompose Step 1, which simply said ``Install
% the
% Engine''}\label{lionel-had-to-decompose-step-1-which-simply-said-install-the-engine}
%
% Lionel bought an inexpensive motor online with the plan of attaching it
% to a cheap bike. As he discovered, though, frugality came with a hidden
% cost: poor instructions. Attaching the motor to the bike was not at all
% a straightforward process.
%
% Lionel: it was funny. When I, when I got the motor /Mmmhmm/ Um, oh,
% it---like I said it was like cheaply made and whatnot and it came with
% instructions, but um, no exaggeration at all, step 1 said ``install the
% engine''. And that was it. \{Laughs\}
%
% Interviewer: Huh
%
% Lionel: And then step 2 was, um, I guess it was somethin like ``put the
% gas tank on.'' It really didn't tell me anything. It was kinda like
% ``OK, those are the obvious steps, but *how* do I do that? So I kinda
% had to figure stuff out on my own. /So/ yup
%
% Interviewer: Did it---how did it end up going? Was it a lot of trial and
% error? Like, did they even have pictures of how it was supposed to look
% when it /There, there/ was installed?
%
% Lionel: There are maybe two or three pictures, but they were black and
% white pictures printed out on like a, I guess like a crappy printer. So
% they were kinda hard to see. Um, so I guess my first step was to go look
% online at better quality pictures. Unfortunately nobody really---I was
% hopin to find you know, like a YouTube video: this is how I did it.
%
% Interviewer: Mmmhmm.
%
% Lionel: I couldn't really find somethin like that, so I just looked at
% pictures. (Interview 1 of 1, October 17, 2011).
%
% Lionel recognized that having an attached motor is an obvious goal
% state, but that goal state was poorly-specified by ``crappy pictures''
% that were hard to interpret. Moreover, the states in between start and
% goal were undocumented and not obvious. Lionel's subsequent search for
% materials that showed a step-by-step progression came up empty. Faced
% with a poorly-specified goal-state and no intermediate guidance,
% Lionel's response was to force himself to think carefully about how he
% would proceed.
%
% \paragraph{In making modifications Lionel frequently asked himself ``how
% am I gonna make this
% work?''}\label{in-making-modifications-lionel-frequently-asked-himself-how-am-i-gonna-make-this-work}
%
% Faced with the challenge of interpolating between ``crappy pictures,''
% Lionel thought hard about what to do.
%
% Lionel: I spent, I guess the, I guess the first five hours I just spent
% kinda staring at it like, ``hmm, *what* am I gonna do?'' Like, ``*how*
% am I gonna make this work?'' (Interview 1 of 1, October 17, 2011)
%
% Ultimately, he was struggling with more than just poor instructions.
% Even if he did figure out how the motor was supposed to be mounted, he
% still faced another problem: the motor was designed for a mountain bike.
%
% Lionel: I had to make a few modifications on my own. Because I guess
% it's built for uh, a mountain bike. And the tubes on mountain bikes are,
% y'know like *that* thin \{makes circle with right thumb and index
% finger\}, but on my beach cruiser there's one tube \{widens thumb and
% forefinger\} that's pretty thick where I had to mount the engine.
%
% Lionel: And since the engine mount was built for a motorized bike it was
% really small. So I kinda had to like devise my own plan on *how* to,
% y'know, add a bigger mount.
%
% Interviewer: Mmmhmm
%
% Lionel: So, that was, that was my first problem. That was, y'know the
% first step was to put the motor on I'm thinkin, well, how am I gonna put
% the motor on if, y'know the, the specs don't even align. So, I guess
% spent y'know five hours just thinkin', like, ``what am I gonna do?''
%
% Lionel: Finally I came up with a plan that seems to be working. So. And
% then, I just kinda went from there. I'd put it together, look at the
% picture and say ``OK, this piece looks like that piece, and y'know I
% guess it looks like it attaches that way, so I'm gonna try that.'' And
% just put it together and, y'know if it fit, it fit. If it didn't, then
% I'd, y'know I'd be like ``OK, maybe that was wrong. Lemme, lemme work on
% somethin' else on it.'' So. (Interview 1 of 1, October 17, 2011)
%
% Asking himself \emph{how} to do something helped bring sub-problems into
% focus. Within a larger mod task (viz., adding a motor to a bike), Lionel
% in effect entered a nested task: modding a mountain bike engine mount to
% fit a beach cruiser. Modding the mount itself was entirely undocumented,
% so Lionel's solution in part was to look for affordances in the physical
% structure of parts to see what he could attach. Crucially, the
% environment afforded Lionel rapid, iterative feedback. It wasn't hard to
% tell when a design choice worked because ``if it fit, it fit.''
%
% \paragraph{Lionel used intermediate representations to consider
% alternative
% designs}\label{lionel-used-intermediate-representations-to-consider-alternative-designs}
%
% Lionel grew increasingly frustrated as he saw how difficult it would be
% to mount the motor. Crucially, part of his tactic around that
% frustration was to resolve to keep trying and thinking of possible
% solutions.
%
% Lionel: So, I guess I had to give myself, like a few hours just to cool
% down and then actually think ``OK, well maybe, you know I've already
% bought it, so, I'm gonna continue on with it. So I had to think, you
% know, OK, what's my next step. *How* could I make this work? You know,
% I---I thought of a few different ideas. You know, I'd think of one idea
% and think ``OK, would this work? It might, but let me think of another
% idea.'' So I'd come up with another idea and say, ``would this work?''
% You know, so once I got a few ideas I would just kind of, in my head
% picture it. Say, ``OK, this is how it would mount, or this is how it
% would mount. Which one would be safer? Or will one of these not even
% work?''
%
% Interviewer: Right.
%
% Lionel: So, I guess after, you know, awhile of deciding which one was
% better, you know, which one would be easier and safe at the same time
% /mmhmm/ so then, that's when I made my decision, and just kinda, went
% for it to see if it would work. And luckily it did \{laughs\}.
% (Interview 1 of 1, October 17, 2011)
%
% Before committing to a design, Lionel forced himself to evaluate
% alternatives first. Naively, he could have gone with the first idea that
% came to him. Instead, even if he thought an idea might work he pushed
% himself to consider others first. Then, with a collection of possible
% designs he evaluated what he saw as their suitability in his head. He
% applied criteria as he evaluated, trying to balance the ease with which
% a mounting scheme might work with a concern for the safety of the
% user---in this case, himself. Again, at least in Lionel's telling, those
% steps happened \emph{before} committing to a final implementation. In
% other words, Lionel generated and evaluated multiple configurations with
% an eye toward both the builder of and user of those configurations
% before settling on a path to take.
%
% In sum, I note three key features of Lionel's approach to the design
% challenges of modifying his bike. First, he repeatedly asked himself
% ``how am I gonna do this,'' forcing himself to think in terms of a plan
% or procedure to fill in the gaps between the instructional pictures.
% Second, he proposed ideas, tried to evaluate their feasibility, and
% continued pressing for alternate solutions (``would this work? It might,
% but let me think of another solution''). Third, so much of his planning
% took place before he ever mounted the motor. By his account, he fiddled
% with pieces, envisioned multiple ways the engine could mount, and
% evaluated their feasibility before ever taking concrete steps to mount
% the motor.
%
% \subsubsection{Lionel's approach to programming resonates strongly with
% the design stance he held when working on his
% bike}\label{lionels-approach-to-programming-resonates-strongly-with-the-design-stance-he-held-when-working-on-his-bike}
%
% \paragraph{Lionel thinks about steps at a
% chalkboard}\label{lionel-thinks-about-steps-at-a-chalkboard}
%
% Later in the interview I asked Lionel to compare his experiences in
% Basic Programming with those he was having in Intermediate Programming.
% His response revealed a great deal about how he starts projects in
% Intermediate Programming: he starts at a chalkboard.
%
% Lionel: Well, {[}Basic Programming{]} I already learned the basics of
% the class so I know the basics of where to start my code. And then,
% obviously the projects in this class {[}Intermediate Programming{]} are
% harder than last class, but I have the basics of, ``OK, well, I know how
% a program runs. So how can I translate that into my program, for, for
% {[}Intermediate Programming{]}?'' And, yeah, so I guess I'll sit down,
% like, yeah, and I'll have like a chalkboard and I'll write things out,
% and say OK well first, you know, \{sweeps hands{]} not even think about
% the code. Just, in \{air quotes\} English words, so you know, my
% project's gonna do this. You know, first I need to make this
% calculation, then I'm gonna print this. And then I'm gonna make this
% calculation. Et cetera. And I'll sit down and write that out, in the
% order that it needs to be done. And then from there I'll go back and
% think, ``OK, what's the code, to, you know, make this calculation or
% print that?''
%
% Interviewer: So, when you start a project, you don't start at a
% computer, it sounds like you kinda start at your chalkboard.
%
% Lionel: Right. Exactly. Yeah, cuz I've tried to start a project at a
% computer, and you know I'll write a---I'll start writing a few lines of
% code, but then I just really zone in on what I'm doing and kinda lose
% sight of the whole picture.
%
% Lionel: So then I'll write this code for, you know, this one little
% section of the project, and then after I've done that I'll look back and
% say ``Oh, yeah, I need to write the whole project,'' but then this
% little code doesn't actually fit into my project, so it's kinda, you
% know I guess I get tunnel vision when I start at a computer. (Interview
% 1 of 1, October 17, 2011)
%
% For Lionel, the chalkboard is a pivotal object in his design activity.
% It holds the top-down ``English words'' description of his program. In
% so doing, it serves as the earliest durable representation of his
% program's intended hierarchy and relationships. That function is
% important, because without it Lionel is apt to ``lose sight of the whole
% picture'' of what his code is supposed to do. Moreover, as the next
% section reveals, the board continues to help Lionel orient his activity
% as he digs into the specifics of how his program will work.
%
% \paragraph{Lionel copies pseudo-code into the computer even though it
% won't
% compile}\label{lionel-copies-pseudo-code-into-the-computer-even-though-it-wont-compile}
%
% As Lionel described the role of the chalkboard, I initially thought
% there was a distinct separation between the chalkboard and the computer
% as activity objects. The chalkboard, I assumed, was where Lionel
% sketched out design plans while the computer was where he wrote code.
% When I probed Lionel based on my hunch, I discovered Lionel's activity
% wasn't cleanly separable into design on the board / code on the
% computer.
%
% Interviewer: Did you, when you start thinking about the code, do you end
% up usually filling it in on your board, or will you actually go straight
% over to the computer and do it?
%
% Lionel: Um, on my board, I'll do, I guess like
% \textbar{}\textbar{}pseudo\textbar{}\textbar{} \textbar{}\{air
% quotes\}\textbar{} code /mmhmm/ uh, you know, first on the board it
% would only say ``this function calls this function, this function calls
% this function.''
%
% Interviewer: Mmmhmm
%
% Lionel: So then on the board I'd write down, ``alright, well this
% function'' um, I dunno, for instance I'll, I'll say you know, I'm workin
% with a function that checks to make sure that it's on the---that I'm
% moving *on* the board between spaces 1 and 24.\footnote{At that point in
%   the semester, Lionel's class was working through a Backgammon project,
%   so the spaces and moves he's referring to pertain to programming a
%   basic Backgammon game.} So then, you know I'll write down like,
% pseudocode, write ``OK, well I'm gonna start. I'm gonna have a for-loop
% that will go from 1 to 24, and then here I'll, you know I'll have an
% if() statement that says if it's between you know, 1 and 24 it's valid,
% if not it's invalid.'' So, that'd be my pseudocode, and then when I
% actually---I'd, I'd try to write that out for the entire program, or as
% much as I could.
%
% Interviewer: Filling out the tree, you mean.
%
% Lionel: Right, fillin out the tree. And then I'd go to my computer, so
% that at least I already had a baseline of what to work with. And then,
% on my computer I'd, you know I'd write out the pseudocode, I'd ac---I'd
% literally write out the pseudocode, even though obviously it wouldn't
% compile and actually work.
%
% Interviewer: Mmmhmm
%
% Lionel: So, then, from the pseudocode, as, as I read down on my computer
% what the pseudocode said, I'd actually try to write the actual code that
% would work. (Interview 1 of 1, October 17, 2011)
%
% Lionel blurred my initial chalkboard / computer distinction in two ways.
% First, his overall ``English words'' plan of functions calling functions
% evolved on the chalkboard to include computational idioms, including
% if-statements and for-loops. But, then, according to Lionel, he would
% copy that pseudo-code as-was into the computer, knowing full well it
% wouldn't compile. These observations---that Lionel writes computational
% pseudo-code on the board, then copies that same code into the
% computer---may seem trivial, but I contend they aren't.
%
% Because of the choices Lionel makes, neither chalkboard nor computer has
% a distinct, privileged kind of syntax. Rather, ``English words'' and
% pseudo-code can peacefully co-exist on the chalkboard, and non-working
% pseudo-code gets deliberately typed into the computer before it's
% expanded out into ``actual code.'' Put another way, Lionel is capable of
% distinguishing between ``English,'' ``pseudo-code,'' and ``actual
% code,'' but all three forms of expression are a part of his design
% process. It's perfectly acceptable for him to have something other than
% ``actual code'' exist in intermediate representations of his design.
%
% \subsubsection{Gestural, inscriptional, and verbal in-interview evidence
% reveals Lionel's resources for structuring a
% program}\label{gestural-inscriptional-and-verbal-in-interview-evidence-reveals-lionels-resources-for-structuring-a-program}
%
% To probe Lionel's retrospective account of how he programs, I gave him
% an in-interview task to work on. In this section, I argue Lionel's
% development process on that task moved rather linearly through what I
% call a ``representational cascade.''\footnote{As I'll explain later, I
%   think it would be a mistake to assume that all development activity
%   proceeds linearly and top-down in the manner Lionel's does. Not only
%   do I think Lionel's clean linearity is rare, I also think it's not
%   normative. So, my point here isn't to stress that Lionel moved
%   linearly, monotonically through the cascade. Rather, my point is to
%   show that he moved between different representational levels \emph{at
%   all}.} My schematic for that cascade --- along with the features and
% affordances in each layer of the cascade --- is depicted in Figure 3.
% What I show here is that actually, the final code Lionel produces is in
% fact one representation of a procedure that actually existed in
% different forms and modalities before it became the final code in Figure
% 3.
%
% In the analysis that follows, I explore two of those modalities---verbal
% description and written pseudo-code---and explain why each is crucial to
% understanding the final produced code. Ultimately, I argue there is a
% need to explain \emph{how} Lionel creates and moves fluidly between
% these markedly different cascade layers. Specifically, I attempt to
% answer the question of \emph{Why does he see the upper layers (verbal
% code, hand-written pseudocode, and handwritten source code) as
% legitimate activity?} My candidate answer is that Lionel's approach to
% programming is in part stabilized by epistemological resources that
% allow him to treat the upper layers of the cascade as valid, productive
% knowledge expressions in programming. Such an answer is strongly in line
% with the findings from sections 3.4.1, where we see Lionel deliberately
% making use of other intermediate design representations.
%
% \includegraphics[width=5.74444in,height=4.31111in]{media/image5.jpeg}
%
% \protect\hypertarget{ux5fToc252445968}{}{}Figure ‑ -- An overview of
% Lionel's representational cascade and the features and affordances of
% each representational layer.
%
% \paragraph{The range-finding prompt and Lionel's final source
% code}\label{the-range-finding-prompt-and-lionels-final-source-code}
%
% About 35 minutes into our interview, I gave Lionel the programming task
% reproduced below:
%
% Suppose you want to write a program that finds the range of a set of
% numbers. So, the program would take 10 numbers as input, then it would
% compute the range as the difference between the largest number and the
% smallest number.
%
% How would you write such a program? (Interview prompt, October 10, 2011)
%
% The prompt was deliberately vague about certain constraints. It did not
% prescribe \emph{how} the program should get input (Should the programmer
% expect the user might type it in, or feed in a formatted file?). It also
% did not prescribe how the program should provide the result (should the
% programmer print the result on screen? Store it to a file? Return the
% result to be used by another function?) The reasons for vagueness were
% in part because I wanted to see what assumptions students might make
% during their design process. Some solutions, I reasoned, might decouple
% the procedures of getting input, computing range, and returning output.
% Others might simply hard-code a procedure for getting keyboard-based
% input. I wanted to prescribe no specific approach up front if I didn't
% have to. I have reconstructed his original source code from
% screen-capture footage of his in-interview programming:
%
% \includegraphics[width=5.74444in,height=7.11111in]{media/image6.emf}
%
% \protect\hypertarget{ux5fToc252445969}{}{}Figure ‑ -- Lionel's final
% source code for the range-finding prompt
%
% Let's explore some top-level features of Lionel's code. First, Lionel
% chose to have a user enter text directly using the keyboard (lines
% 9--12), and his program displays the range result as printed text (lines
% 28--29). His strategy for finding the maximum and minimum is to
% implement a procedure that iterates through the array of numbers and
% compares the current number to the globally stored max and
% min.\footnote{I think of this as a ``king of the court'' procedure. The
%   current value stays on as an extreme if and only if it beats the
%   current ``king'' for that extreme. Otherwise, the challenger is
%   replaced and the contest resumes with a new challenger until there are
%   no more challengers.} If the current number (array{[}i{]}) ``beats''
% the global min (by being lower than min), its value replaces that of the
% global min. Similarly, if the current number exceeds the global max, its
% value replaces that of max. Before the procedure starts, Lionel assigns
% the first number of the array as the value for both extrema.
%
% \paragraph{Lionel's verbal description of the
% program}\label{lionels-verbal-description-of-the-program}
%
% Lionel chose to start the task on pencil and paper. After scribbling a
% line about scanning input into an array (which I discuss in a later
% section), Lionel stopped and began explaining out loud to me.
%
% Lionel: so I'm thinkin, my general concept is I'm going to parse through
% this array, check each number /Mmmhmm/ and I'll have two variables, min
% and max, and, first I'll set both min and max to the first variable in
% array, er, yeah first number in array /Mmmhmm/ and, I'll parse through
% the rest of the array, checking to see if,
%
% Lionel: Umm, you know if---I'll s---I'll start with max, for example.
% /Mmmhmm/ And I'll check to see if the number in that section of the
% array is bigger than max. /uh-huh/ And if it is, I'll set max to that
% new number. /OK/ And then I'll do the same thing for minimum. And then
% at the very end, I'll have my minimum and max and then I'll just
% subtract `em.
%
% Interviewer: OK
%
% Lionel: So that's my main concept. (Interview, October 17, 2011)
%
% This verbal description is in some ways bound by the least stringent
% rules of expression. If we're hard-nosed about syntax, Lionel's speech
% stops and starts (line 5 into line 6), which leans on the unspoken
% understanding that ideas in talk are under negotiation. Stopping
% mid-sentence and leaving that sentence unfinished is a valid and
% acceptable move in conversational space. Moreover, there is a strong
% mapping between Lionel's temporal language and the intended
% flow-of-control for the program. ``First I'll set,'' ``then I'll do,''
% ``and then at the very end'' all describe when instructions will be run.
%
% Finally, Lionel's instructions begin at the high level, and in some
% cases implementation details are completely left out. For example,
% Lionel does not specify precisely how he'll ``parse through the array.''
% At this stage of the cascade, it seems sufficient for him to explain
% \emph{that} he plans to parse through, but now how. What Lionel
% \emph{does} do is gesture while describing ``parsing''. This gesturing,
% I argue, creates a concrete representation of iteration and
% incrementation --- the gestural marker of a loop.
%
% \includegraphics[width=5.74444in,height=3.23333in]{media/image7.png}
%
% \protect\hypertarget{ux5fToc252445970}{}{}Figure ‑ -- Lionel makes a
% cycloid/helix gesture as he says ``parse through this array''
%
% Contrast how Lionel specifies ``parsing'' through the array with how he
% describes ``checking'' its contents. First, Lionel explains that he'll
% ``check each number'' (line 2), decides that he has underspecified how
% ``checking'' happens (line 5--6), and finally makes ``checking''
% relatively more explicit (lines 7--8). Once checking for the maximum has
% been made explicit (lines 7--8), Lionel says ``and then I'll do the same
% thing for minimum'' (line 9). Line 9 thus becomes the verbal equivalent
% of a repeated, parameterized procedure. It is, in a sense, the
% talk-space seed of a function. Conversationally, both participants
% understand that Lionel saying, ``then I'll do the same thing for
% minimum'' (line 9) refers to a procedure defined elsewhere in the
% conversation, not a new set of instructions.
%
% \paragraph{Lionel's pseudo-code}\label{lionels-pseudo-code}
%
% Immediately following the verbal description above, Lionel begins
% creating what he calls his ``pseudo-code,'' narrating it as he writes.
% There are four key features in Lionel's talk (below) that I discuss in
% my analysis. Below is an overview of each feature and what importance I
% think it has:
%
% \begin{enumerate}
% \def\labelenumi{\arabic{enumi}.}
% \item
%   Lionel iterates over his initial verbal description by fleshing out
%   some implementation specifics --- structures that further specify
%   \emph{how} something in a program is to be done.\footnote{It's a
%     limitation of my writing style that in this document I use
%     ``iteration'' both to describe aspects of Lionel's design process
%     and as a formal term of art in computer science. In both cases,
%     though, the conceptual meaning is the same: the incremental,
%     rule-based set of transitions to the state of an object.} This
%   iteration is important because it suggests that his subgoals --- and
%   criteria for satisficing them --- change from one micro-coherence to
%   the next.
% \item
%   Reconciling what Lionel \emph{narrates} with what Lionel \emph{writes}
%   reveals peculiar mismatches. The mismatches suggests a productive
%   capacity: that Lionel can \emph{background} certain details within a
%   micro-coherence by omitting them from the written code while still
%   maintaining an understanding of how his program will work. Put another
%   way: the inscriptional object in a microcoherence captures what Lionel
%   thinks is important within that microcoherence, but the inscription
%   does not necessarily represent the totality of his conceptual
%   understanding.
% \item
%   At one point in his code, Lionel writes ``same for min,'' which we can
%   reasonably infer means he intends to repeat a procedure for
%   calculating the minimum. That line encapsulates the precursors of
%   functional abstraction: Lionel not only recognizes procedures that
%   share the same structure, he can background the specifics (point 2) of
%   code he repeats.
% \end{enumerate}
%
% Lionel: So it'll be scanf, um, and I'll have my for-loop, and I'll just
% have my int-i, and I'll go from zero uh, zero to, four actually cuz
% it'll be zero, one, two, three, four, yeah, zero to four. /Mmmhmm/
%
% Lionel: Umm, and I'll have, oh, up top here, to set the, right---right
% after the scanf /Mmmhmm/, I'll just say min and max equal the
% first---mm, hiccup, sorry
%
% Interviewer: It's OK.
%
% Lionel: \{laughs\} the first number in the array. So I'll say min and
% max equal array of zero. And that's just setting them so that I have
% something to compare to, I guess. /uh-huh/ I think that will work. Yeah.
%
% Lionel: So, so then, as I parse through my array of five variables, I'll
% say ``if min,'' er no, I'll say ``if array of i, you know of, of that,
% if, uh, the number in the array that I'm looking at is greater than max,
% then max equals that number, array of i'' /Mmmhmm/ and then I'll do the
% same thing, same for min. And then at the very bottom, I'll have, uh,
% range equals max minus min.
%
% Lionel: So that's my pseudo-code, just so that *I* understand what's
% goin' on in my head.
%
% \includegraphics[width=5.75000in,height=3.00208in]{media/image8.emf}
%
% First, Lionel's talk iterates over his initial verbal description by
% fleshing out \emph{implementation details} --- programming structures
% that further specify \emph{how} something is to be done. Compare here
% how Lionel instantiates ideas of iteration in Table 1 below.
%
% \protect\hypertarget{ux5fToc252445959}{}{}Table -- Comparing Lionel's
% words and actions across different stages of ``programming'' over the
% same putative section of his program. I have bolded some
% language-specific syntax.
%
% \begin{supertabular}[]{@{}lll@{}}
% \toprule
% \textbf{Design Coherence} & What he says & What he does\tabularnewline
% \midrule
% \endhead
% \textbf{Verbal Description} & ``I'm going to parse through this array,
% check each number'' &
% \includegraphics[width=2.22222in,height=1.48889in]{media/image9.jpeg}\tabularnewline
% \textbf{Narrated pseudocode} & ``I'll have my \textbf{for-loop}, and
% I'll just have my \textbf{int-i}, and I'll go from zero uh, zero to,
% four actually cuz it'll be zero, one, two, three, four, yeah, zero to
% four. &
% \includegraphics[width=2.23333in,height=0.35556in]{media/image8.emf}\tabularnewline
% \bottomrule
% \end{supertabular}
%
% In his narrated pseudo-code, he introduces the C-specific syntax of
% ``for-loop'' and ``int-i,'' which handle iteration and indexing,
% respectively. Moreover, he counts in order to specify the index bounds
% of the loop (zero to four).
%
% Second, it's actually somewhat complicated to address the question ``how
% do Lionel's inscriptions align with what he says?'' He \emph{says}
% ``I'll have a for-loop'', but the visual evidence shows that he hasn't
% bounded his for-loop body in curly-braces.\footnote{As a language, C
%   typically uses curly braces to bound instructions that span multiple
%   lines. So, a typical for-loop would be written like this, where the
%   loop preamble is bounded by parenthesis and the loop body is bounded
%   by curly braces (bolded):
%
%   for(int i=0; i \textless{} maximum\_value; i++) \textbf{\{}
%
%   // Do something
%
%   \textbf{\}}} If he had done so, his for-loop preamble would end with
% an opening curly brace; it doesn't (Table 1 graphic above). Also, for
% matching, his for-loop should have concluded with a closing curly brace;
% it doesn't. Nevertheless, his verbal evidence --- particularly him
% saying ``int-i'' --- suggests that he understands details that variables
% need to be declared. Together, these mismatches suggest that features
% (viz., denoting iteration and setting its bounds) are important enough
% to be included in this representation. Other specifics, including proper
% declaration of variables, are things Lionel understands but does not
% insist on writing.\footnote{One might argue that Lionel is simply being
%   sloppy and forgetting to include details, even though he said them.
%   The point I'd make here is that even if he is being sloppy, I would
%   think it unfair to say, for example, that Lionel has conceptual issues
%   with declaring variables. His talk (and many later iterations of his
%   design) suggests he understands that declarations are necessary for
%   working with variables in C. Moreover, even if he is being sloppy on
%   the particular point of variable declarations, the overall character
%   of this pseudo-code is still starkly different from that of the full,
%   working source code solution he'll ultimately produce. So, even if we
%   propose that Lionel has some unintended tolerance --- such as
%   sometimes overlooking missing variable declarations --- for assessing
%   whether his pseudo-code satisfices, there is still strong evidence
%   that his criteria for evaluating his work in pseudo-code stage are
%   markedly different from the criteria he'll apply to later iterations
%   of design.
%
%   no one would confuse the pseudo-code skeleton here for Lionel's
%   fleshed-out, fully-working code later. So, even if Lionel \emph{meant}
%   to put them in and forgot, he still judged this pseudocode as
%   satisficing his subgoal} Together, and combined with later evidence of
% Lionel's successful working program, these observations suggest that
% ignoring the details is actually productive for Lionel.
%
% Third, Lionel's talk reveals productive capacities for dealing with
% design work. For example, he describes the guts of his program through
% brief perspective-taking (lines 14 -- 18 above). As a reminder, Lionel's
% procedure is what I refer to as a ``king of the court'' approach: if the
% number currently-being-inspected exceeds the reigning champ, the champ
% is dethroned and replaced by the number currently-being-inspected. But,
% the way Lionel articulates that process actually uses multiple senses of
% ``I,'' where it's at times actually not clear whether ``I'' means ``I
% the programmer'' or ``I the program.''\footnote{To keep your head from
%   exploding, I remind you that the lower-case version of ``i'' in
%   Lionel's talk actually refers to the subscript for indexing an array.}
% For example, the I in ``I'll say `if the array of i'' could very well be
% the programmer, where by ``I'll say'' he means ``I the programmer will
% write a conditional test.'' But, that phrase is immediately followed by
% ``if, uh, the number in the array that I'm looking at is greater than
% max'', where I believe we may be seeing Lionel taking the perspective of
% the program itself.\footnote{My point here is that technically, Lionel
%   can never inspect an array element himself; he can only write code
%   that does that. You might disagree, in which case we get to have a
%   really fun discussion over beer about how we think programming is a
%   case of ``distributing cognition'' (Hall, Wright, \& Wieckert, 2007).}
% If anything, the seemingly untroubled, interchangeable use of ``I'' here
% might suggest that to do the work of programming one can occupy a hybrid
% space of both being the program and being the programmer at the same
% time.
%
% Lionel concludes this procedure with the invocation ``same for min.''
% Picking up the thread of hybridity, this phrase could mean, ``I the
% programmer will write a similar procedure for min'' or ``I the program
% will perform similar operations to determine min.'' That Lionel leaves
% his pseudocode there (``so that's my pseudocode'', line 20) suggests
% that whatever perspective it might be written from, this designed
% artifact satisfices his pseudo-code production subgoal. And, that
% satisfaction is remarkable because if ``same for min'' is an appropriate
% place-holder, it suggests Lionel has ways of seeing that procedure as
% \emph{similar enough} in structure to the procedure already laid out as
% to not require further specification. This is an important point: if a
% precursor to functional abstraction is recognizing repeated code, Lionel
% has actually recognized repeated code \emph{before ever writing it}. As
% we'll see later, he ends up writing this procedure out explicitly, which
% means that whatever answer we might come up with as to why Lionel
% doesn't abstract this to a function \emph{can't} be that he doesn't see
% the repeated computation; to the contrary, repeating core parts of the
% max computation was built into his design from the start.
%
% \subsubsection{\texorpdfstring{\protect\hypertarget{ux5fToc247188557}{}{\protect\hypertarget{ux5fToc252445930}{}{}}We
% can trace continuities in Lionel across different kinds of design
% activities}{We can trace continuities in Lionel across different kinds of design activities}}\label{we-can-trace-continuities-in-lionel-across-different-kinds-of-design-activities}
%
% We can trace patterns in Lionel's design activity by asking questions
% about the intermediate design artifacts he creates. Specifically, we can
% pay attention to \emph{what} representations he creates, \emph{where}
% they live, and \emph{how} he uses them. Table 3 outlines those
% relationships for the three contexts discussed: Lionel's bike modding
% project, Lionel's retrospective account of his programming practices,
% and Lionel's in-interview work on the range-finding task. Table 3 helps
% support three key points:
%
% \begin{enumerate}
% \def\labelenumi{\arabic{enumi}.}
% \item
%   There is a specific consistency to Lionel's design activity across
%   contexts. All activities---given the limits of Lionel's
%   self-reports---\emph{involved the generation and use of intermediate
%   design representations}. Those representations may at times have lived
%   in Lionel's head, on paper, or in electronic format.
% \item
%   A thorough account of Lionel's in-interview programming activity
%   cannot and should not ignore those intermediate representations and
%   how he used them.
% \item
%   Because intermediate representations are a feature of all three design
%   contexts, it seems likely that what stabilizes their creation and use
%   is tied to something deeper or more global than just conceptual
%   knowledge in a domain such as computer programming.
% \end{enumerate}
%
% \protect\hypertarget{ux5fToc252445960}{}{}Table -- Comparing Lionel's
% design activity across contexts
%
% \begin{supertabular}[]{@{}llll@{}}
% \toprule
% \textbf{Context / Question} & Modding the bike & Retrospective
% programming account & In-interview task\tabularnewline
% \midrule
% \endhead
% \textbf{\emph{What} intermediate representations did he create?} &
% Possible configurations for mounting the engine & Relationships of
% functions to each other; pseudo-code for how functions would work & A
% verbal description; narrated pseudo-code; hand-written pseudo-code;
% hand-written source code\tabularnewline
% \textbf{\emph{Where} do the intermediate representations live?} & In
% Lionel's head & On a chalkboard and, later, in the text editor & In
% Lionel's talk and gestures; on paper; in the text editor\tabularnewline
% \textbf{\emph{How} did he use those intermediate representations?} & To
% compare mounting configurations for ease and safety & To ``see the whole
% picture'' of a project & To ``understand what's goin on in my head''
% when structuring his code\tabularnewline
% \bottomrule
% \end{supertabular}
%
% \subsection{Rebecca's approach to
% programming}\label{rebeccas-approach-to-programming}
%
% In this section, I analyze data from a different student --- Rebecca. I
% offer this data as an example of what developing design expertise might
% look like. If Lionel's approach to programming uses high-level work
% (e.g., his pseudo-code) to cascade down the specifics of syntax, it
% seems reasonable to ask whether other students use pseudo-code in the
% same way. And, if they do not, it seems worth asking how uses of
% pseudo-code differ and what that difference might tell us about
% students' approaches to programming.
%
% In Rebecca's case, the data I'll present begins with her stuck on the
% third of four projects in the course: the ``iTunes'' project.\footnote{In
%   this project, students were given a series of text files ---
%   representing the user's music collection --- that contained formatted
%   albums and song titles. The task was to create a program to read in
%   the music information, create a database representation of it, and
%   allow the user to transact with that database. Typical transactions
%   might be asking for a listing of all the albums in a user's library or
%   creating a playlist by allowing the user to choose specific songs.} My
% analysis will expand on the following points:
%
% \begin{enumerate}
% \def\labelenumi{\arabic{enumi}.}
% \item
%   To understand Rebecca's struggle on the iTunes project and her
%   approach to programming during that struggle, we need to understand
%   some of her prior experiences in programming. In our fourth interview
%   Rebecca recalls a pivotal project experience from her Basic
%   Programming course. At the cost of hours of work, half her codebase,
%   and a good grade, Rebecca discovered that her sense of how her program
%   should work did not match what was expressible in C. I analyze
%   Rebecca's recollection because it sits at the intersection of
%   self-efficacy and disciplinary practice. Rebecca lost confidence and
%   felt ``demoralized'' because her attempts at expressing a procedure
%   did not match the constructs and underlying workings of the C
%   language. Moreover, the fallout from that experience persisted into
%   Intermediate Programming. By our fourth interview she had considered
%   dropping Intermediate Programming entirely because of her low grades.
%   ``My *biggest* problem'' in programming, she said, ``is I think I have
%   the logic, my logic just doesn't transfer to code. Or I don't know
%   *how* to transfer it correctly'' (Interview 4 of 5, April 6, 2012).
% \item
%   Rebecca has the capacity to make sense of program designs. As she
%   tries to understand why the instructor would prescribe an array of
%   pointers to store track titles in the music server, Rebecca makes
%   progress by reasoning about what she knows about pointers and arrays.
%   Like Lionel, Rebecca can use gestures, talk, and inscriptions (viz.,
%   pseudocode) to express how she might intend for her program to work.
%   Rebecca also has a grasp of the conceptual issues --- from a
%   computational and programming perspective --- relevant to the part of
%   her design she's struggling with. But, Because Rebecca's approach to
%   programming so strongly orients toward producing valid, working
%   syntax, it may be limiting Rebecca from seeing other ways in which she
%   can make progress on her design. In other words, Rebecca \emph{can} do
%   things like write pseudo-code to plan, but she doesn't tend toward
%   those behaviors when she's stuck. I'll offer an explanation based on
%   epistemological dynamics to explain why that's the case.
% \end{enumerate}
%
% In what follows, I first present point 1 to give readers a sense of how
% Rebecca's prior experiences may inform her approach to programming. I
% then outline the particular part of the iTunes project where she feels
% ``stuck,'' discussing point 2. Before I jump into that analysis, I
% \emph{strongly} urge readers who are not familiar with pointers or
% dynamic memory allocation to refresh themselves on the subject.
%
% \subsubsection{\texorpdfstring{\protect\hypertarget{ux5fToc247188558}{}{\protect\hypertarget{ux5fToc252445932}{}{}}Rebecca's
% struggles on the iTunes project make sense in light of some prior
% programming experiences she
% had}{Rebecca's struggles on the iTunes project make sense in light of some prior programming experiences she had}}\label{rebeccas-struggles-on-the-itunes-project-make-sense-in-light-of-some-prior-programming-experiences-she-had}
%
% At the time of this study Rebecca was a first-year electrical
% engineering major who had gone to high school in rural Maryland. Her
% university courses in Basic and Intermediate Programming were the first
% experiences she had doing any programming at all, let alone in the C
% language. Though she felt academically at ease in high school, a tough
% project in her first-semester Basic Programming course shook her
% confidence in programming. In the data presentation and analysis that
% follows, there are four sub-points I'd like to make about Rebecca's
% development in programming.
%
% \begin{enumerate}
% \def\labelenumi{\arabic{enumi}.}
% \item
%   Prior to the semester I worked with her, Rebecca had difficult
%   experience trying to understand arrays on a Basic Programming course
%   project.
% \item
%   That difficult experience had two pieces to it that I believe relate
%   to personal epistemology:
%
%   \begin{enumerate}
%   \def\labelenumii{\alph{enumii}.}
%   \item
%     Rebecca came to feel that her idea for a solution wasn't something
%     expressible in C.
%   \item
%     When she got help from a friend who was ``very smart in
%     programming,'' she felt like she typed the code he told her to into
%     her project without really coming to understand it.
%   \end{enumerate}
% \item
%   Reflecting on it in our first interview, Rebecca pinpoints the
%   experience of that array project as demoralizing and having a lasting
%   effect on her confidence in programming.
% \item
%   Epistemological issues continue to factor into Rebecca's confidence in
%   programming. By our fourth interview, Rebecca was getting low grades
%   in Intermediate Programming. She had strongly considered dropping the
%   course. Trying to articulate her ``biggest problem,'' Rebecca
%   explained that she felt like she understood the theory behind the code
%   examples in class but had difficulty transferring her logic into code.
% \end{enumerate}
%
% \paragraph{Rebecca had a difficult experience on a Basic Programming
% arrays
% project}\label{rebecca-had-a-difficult-experience-on-a-basic-programming-arrays-project}
%
% In our first interview, I asked Rebecca about how her experiences in
% Basic Programming compared to those she was having in Intermediate
% Programming. She explained that in that first semester, she ``kinda got
% lost on one part'' --- arrays --- ``that was key to the rest of the
% year.''
%
% Rebecca: {[}I{]}n {[}Beginning Programming{]} we had three projects, and
% the first one, uh, we did---I did discuss it with other students, and we
% all worked together kind of. But that one I felt much more like I had a
% grasp on everything I was doing. Whereas the second project, I would
% stare at the code and be like ``ohmygosh, I don't even understand what,
% like, it's asking me to do.'' Like, or how to make it do what it needed
% to do, which was to like store in numbers in arrays, change them, and
% add `em together and stuff. (Interview 1 of 5, February 10, 2012)
%
% I asked Rebecca if she felt she had a better grasp by the time she
% turned in the project.
%
% Rebecca: No. Cuz, there was a friend on my floor, who---very smart at
% programming, he's had like years of experience. And he gave me a lot of
% pointers and tips doing it, so I felt like I was just using him a lot
% instead of actually learning it, which was a downfall for me.
%
% Interviewer: Huh. So, how, I mean. I feel like sometimes you could---you
% might ask somebody for help, and then they show you something and then
% you might be like oh, you've learned it now /yes/, but it sounds like
% you're saying /no/
%
% Rebecca: Just, I don't, like, I don't blame him at all because he helped
% me, but, it was kind of more of a, he showed me how to do it, I was like
% \{mimes exaggerated typing, in an almost singsong voice\} *OK, sure,
% I'll type that code*, but I don't really, I never really, I don't know I
% never made the connection. (Interview 1 of 5, February 10, 2012).
%
% \paragraph{Rebecca's ``terrible'' array experience has epistemological
% components}\label{rebeccas-terrible-array-experience-has-epistemological-components}
%
% In our second interview a week later when we discussed debugging,
% Rebecca returned to that experience in her Basic Programming class:
%
% Rebecca: My array project, my---the second project we did last semester,
% that I had a big error. I had to delete like half the code I worked on.
% Like, just cuz, like I had said last week, I didn't do very well on
% arrays. So.
%
% Interviewer: Oh. Was it cuz of one thing, or was it like a repeated---
%
% Rebecca: It was kind like a general, like, what I was trying to do
% didn't work with C.
%
% Interviewer: Do you remember what it was you were trying to do?
%
% Rebecca: I don't know exactly, but uh, I just remember, like, whatever I
% was tryin to do, cuz I had my friend help me, he's like ``you can't do
% this this way,'' like, it just, C doesn't recognize whatever I was
% trying to do.
%
% Interviewer: Huh. So that was like, it sounds like it's almost kind of a
% case where your, your plain English of what should happen /yeah/ can't
% actually translate---it's not even that you're not sure how it's---
%
% Rebecca: It's like, it was impossible.
%
% Interviewer: It was never---
%
% Rebecca: It was never allowed.
%
% Interviewer: There is no word for that /yeah/ in this language.
%
% Rebecca: I \{laughs\}, that was back then, so. That was always fun.
%
% (Interview 2 of 5, February 17, 2012)
%
% Epistemologically, two patterns stand out in Rebecca's recollections.
% First, as Rebecca remembers it, one of her core problems was that her
% visions for how her program should accomplish things were ``impossible''
% and ``never allowed'' in C. Second, in order to get past that obstacle,
% she ultimately typed code a friend told her to without feeling like she
% understood it. As this analysis unfolds, I think these two points give
% us a crucial starting point to understand Rebecca's practice. The
% experience underscored a gulf between ideas Rebecca wanted to invoke and
% the syntax she needed to realize them in C. And, her way of dealing with
% that gulf was to make an end-run around it by taking and using code she
% didn't fully understand.\footnote{I say this without any intended
%   judgment (or pejorative connotations) about what Rebecca did. Given
%   the assessment structure of the course, where 90\% of a student's
%   grade came entirely from whether the program worked to spec, she may
%   have felt that getting a good grade by using code she didn't
%   understand was preferable to getting a bad grade with code that didn't
%   work.}
%
% It's worth noting that whether her friend's code worked to
% spec\footnote{Here, I'm using ``worked to spec'' as a shorthand for
%   ``worked in such a way that it passed all of the input/output tests
%   the instructor would run to determine 90\% of a student's grade.''} is
% immaterial to the two epistemological issues I raise. First, if the code
% she copied worked, Rebecca would get a good grade, but she wouldn't
% understand how her project functioned. And, if it didn't work, she would
% get a bad grade \emph{and} she wouldn't understand how her project
% functioned. Ultimately, no matter how the code performed, Rebecca would
% likely be in a tough spot if she ever had to maintain, revisit, or
% refactor the code she copied. Second, no matter how her copied-code
% project functioned, it represented \emph{someone else's} articulation of
% how a procedure should go. That is, the copied code was an artifact of
% someone else bridging the gulf between high-level idea and articulation
% in C. Rebecca still found difficulty traversing that gulf.
%
% \paragraph{Rebecca's array project experience had a lasting effect on
% her confidence in
% programming}\label{rebeccas-array-project-experience-had-a-lasting-effect-on-her-confidence-in-programming}
%
% As that interview drew to a close, I wrapped it up with what had become
% standard protocol: asking whether there was anything else she wanted to
% talk about.
%
% Rebecca: I think I'm good, I---just
%
% Interviewer: Anything else on your mind?
%
% Rebecca: Sorry I'm not the greatest programmer \{laughs\}
%
% Interviewer: No, hey hey hey. That's \{shakes head\}, first of all, I
% mean, um, how come you're apologizing?
%
% Rebecca: I dunno. It's a study, and I'm probably not the best test
% subject. (Interview 2 of 5, February 17, 2012)
%
% Rebecca's spontaneous apology spawned a discussion about why she feels
% diffident in programming. For her, that diffidence traced its way back
% Basic Programming. Feeling lost and stuck on that array project was more
% than just a tough experience; it was a turning point:
%
% Rebecca: When I got lost I think is when I started losing that
% confidence thing. And then I never really got it back, so.
%
% Interviewer: When was it you started to get lost?
%
% Rebecca: Halfway through last semester, like with the arrays in the
% second project. So\ldots{}. I guess it was like, demoralizing, kind of,
% and, like, like I don't want to sound cocky or arrogant or anything, but
% in high school I did very well in academics. And, so, I never really,
% like anything that I had to work for, I worked for for a short period of
% time, got it, and like I'd understand it, and if I didn't, like it was
% really easy for someone usually to explain it to me. And, I felt like I
% didn't really get that after I lost everything on that second project, I
% was like, I wasn't able to build myself back up out of that, which
% probably stems to why, not as confident in the class. (Interview 2 of 5,
% February 17, 2012)
%
% As I interpret how this array project experience may be playing a part
% in Rebecca's later approaches to programming, I stress that my analysis
% is about Rebecca's perceptions and recollections.\footnote{Due to
%   deliberate study limitations I don't have access to the actual code
%   from that project. To be as conservative as possible in our data
%   collection and respect the privacy of participants, we asked
%   participants to grant us access only to folders that would contain
%   their Intermediate Programming projects. Outside code/data, including
%   code from prior semesters, was excluded from the scope of our code
%   snapshot data collection.} It's possible that her ideas for how her
% project should work wouldn't have been implementable in any language,
% let alone in C. And, while I think that possibility is unlikely, I have
% to concede that I only know what Rebecca recalled; I wasn't present for
% the experiences she's describing. Nevertheless, the question for my
% research is not ``can we know what really happened in that experience?''
% Rather, my question is, ``how might Rebecca's \emph{image} of that
% experience --- with its specific epistemological facets --- be playing a
% part in her approach to programming now?''
%
% Part of the answer to that question may be in her quote above. She talks
% about losing her confidence and ``never really {[}getting{]} it back.''
% Later, she says ``I felt like I didn't really get that after I lost
% everything on that second project.'' Because of the peculiarities of
% speech, I had a difficult time parsing Rebecca's statement about losing
% everything on that second project. It wasn't clear to me whether there
% was a pause between ``get that'' and after, but the meaning of ``that,''
% and in turn the meaning of the sentences, depends on the antecedent of
% ``that.'' So what she said could have two consequentially different
% meanings:
%
% \begin{enumerate}
% \def\labelenumi{\arabic{enumi}.}
% \item
%   ``I feel like I didn't really get \textbf{that}. After I lost
%   everything I lost everything on that second project, I was like, I
%   wasn't able to build myself back up out of that.'' In this version,
%   ``that'' may mean the topic---arrays---or the project as a whole. I
%   might then interpret Rebecca's speech as saying this project was a
%   turning point because of the particular topic. Her being ``unable to
%   build myself back up out of that'' reflects a kind of debt that might
%   have piled up when later assessments depended cumulatively on that
%   early topic. Her failure to grasp arrays was thus cumulatively
%   punishing because arrays formed a core part of many projects that
%   followed.
% \item
%   ``I feel like I didn't really get \textbf{that} after I lost
%   everything on the second project. I was like, I wasn't able to build
%   myself back up out of that.'' In this version, ``that'' may mean her
%   ready insight into topics or her facility in understanding when
%   friends explain them to her. The distinction I draw in this second
%   interpretation is ``that'' refers to a kind of continuity in her
%   identity---someone for whom school always came easy. The array project
%   disturbed that continuity because it was an instance in which things
%   no longer came easy to her. And, the experience may have been so
%   troubling that she was never able to re-establish being able to easily
%   grasp things in programming. The result, then, is that she couldn't
%   build herself---in the sense of her continuous identity as one who
%   quickly grasps school topics in general and programming topics in
%   particular---back up after the array project.
% \end{enumerate}
%
% Distinguishing between these possible meanings of Rebecca's speech might
% seem pedantic. But, I think the bifurcation is important for theorizing
% about learning and instruction in computing. If Rebecca's difficulty was
% about the topic of arrays, a plausible narrative is that the difficulty
% of that concept---and the assessments that continued to depend on
% it---caused a run of bad grades, and the bad grades shook Rebecca's
% confidence. But, if Rebecca's difficulty comes from troubling her sense
% that she has a ready understanding of programming, the attack on her
% confidence is more complex. It's not just that she doesn't get arrays,
% it's that she feels she doesn't readily understand other new code she's
% seeing in the course---even code that doesn't strongly rely on arrays.
% In this view, the confidence Rebecca speaks of isn't just a product of
% getting good grades; it's part of a feedback loop in which grades,
% confidence, and a feeling of ready insight into programming are all
% components.
%
% That potential coupling of grades, confidence, and feelings about ready
% insight into code matters because it further pushes the discussion
% toward epistemological considerations. It is why, for example, I was
% paying such careful attention to Rebecca's recollection of copying down
% code her classmate told her. Instead of helping her confidence in
% programming by potentially boosting her grade, the copying may have
% diminished her confidence because she couldn't understand why someone
% else's code worked.
%
% \paragraph{Epistemological issues continued to interact with Rebecca's
% confidence in Intermediate
% Programming}\label{epistemological-issues-continued-to-interact-with-rebeccas-confidence-in-intermediate-programming}
%
% Rebecca started our fourth interview with a warning to me that she might
% ``rant'' about programming.
%
% Interviewer: What would you rant about? Go ahead.
%
% Rebecca: Ohh god. Um, I am not doing well \{laughs\} in the class right
% now.
%
% Interviewer: OK
%
% Rebecca: And it's not good, so. Yeah, like, we just had our last project
% due, which I did a lot better on than the first project
%
% Interviewer: Uh-huh
%
% Rebecca: Which is making me really happy. Um, but, everything about
% programming just upsets me, because there's the two guys, like I had
% mentioned before who, uh, help me
%
% Interviewer: Mmmhmm
%
% Rebecca: Uh, one of them, he---like I worked every day from the day we
% got this project on this project, and my project still didn't turn out
% perfect. He wrote his twice in one day because he lost the whole thing.
% And, his was perfect. And, I dunno---it just irks me *so* bad that that,
% it can be that much of a disparity between people and programming.
% (Interview 4 of 5, April 6, 2012)
%
% In her telling, Rebecca wasn't the only person upset by the relative
% ease with which this student programmed.
%
% Rebecca: we didn't say it to him, but some of us were like, it's, we
% just found it like, so, like aggravating. Like, cuz there's me and
% another girl who, on my floor who, she's uh, never had programming
% experience either---
%
% Interviewer: Uh-huh
%
% Rebecca: And we're just like it gets so aggravating that it comes easy,
% like to him, and that he does it well, but, that's just---he ju---he is
% very good at school. He's very smart. So. (Interview 4 of 5, April 6,
% 2012)
%
% Note here that Rebecca is connecting the idea that programming came easy
% to this student to the notion that he was ``very good at school.'' In
% this moment, the statement seems to suggest Rebecca has a \emph{fixed
% mindset} \{Dweck and colleagues\} about programming expertise. While it
% would be naïve to say Rebecca believed --- globally --- that programming
% came easily to those who were ``good at school,'' it's nonetheless true
% she saw a discrepancy between how easy programming was for her and her
% friend and how easy it was for this ``very smart'' classmate.
%
% As we kept talking about this discrepancy, Rebecca offered further
% evidence that struggling with programming troubled her academic
% identity. In high school, she ``didn't have to study a lot'' because ``a
% lot of things came relatively easy'' (Interview 4 of 5, April 6, 2012).
% But, her experiences in Intermediate Programming pushed her into a new
% position. As she said,
%
% now I'm on the other end and I'm realizing how aggravating it
% is\ldots{}. And, I don'---I guess it's just a shock, cuz I do not like
% not being good at things \{laughs\}. So...'' (Interview 4 of 5, April 6,
% 2012).
%
% Again, Rebecca's comments stress the frustration of feeling like
% programming comes easily to some people but not to her.\footnote{Or, at
%   least, at that moment, in that semester, it did not seem to be coming
%   easily to her.} And, they are charged with emotion. She speaks of
% being ``on the other end,'' connoting an idea of a structure whose poles
% have people who ``get'' programming on one end and those who don't on
% the other. It's ``aggravating'' to be on that end, and a ``shock'' to be
% there for the first time.
%
% I asked Rebecca to reflect on whether she could ever remember seeing
% classmates in the position she was now, where something seemed to come
% easily to Rebecca but not to them.
%
% Rebecca: Um. I mean, there was a---there's a friend of mine, who, from
% my high school, who, I, I mean just always got math. And, uh, she was
% struggling in math a lot.
%
% Interviewer: Mmmhmm
%
% Rebecca: But her attitude was, uh, ``I don't need math. I'm gonna be an
% English major.''
%
% Interviewer: OK
%
% Rebecca: So, she didn't even care about it
%
% Interviewer: Oh, OK
%
% Rebecca: Whereas, like, I was like, I'm an electrical engineer major, I
% actually need this. So. Like that's the only time I think I've ever felt
% like someone's had that feeling that I probably did.
%
% Interviewer: Right.
%
% Rebecca: And, I mean I'm sure there are people that I---uh, that I
% didn't know, or didn't realize they had that feeling. But, it's just,
% interesting to be on the other end. \{6 second pause\}
%
% Rebecca: Umm. Just. Ugh. I've like, debated withdrawing from the class.
% And, like, I've talked to three different advisors or whatever, and
% they're like ``don't do it. It's not---you can still bring your---you
% can bring your grade up. And even if you don't you have freshman
% forgiveness,\footnote{In the interview, Rebecca explained that Freshman
%   Forgiveness was a policy extended to students during their first 24
%   credit-hours at the university. If a student gets a grade she's
%   unhappy with, she can retake that course and her second final grade
%   will replace the first.} so.''
%
% Rebecca, unlike her friend, couldn't afford \emph{not} to care about
% Intermediate Programming; it was a required course for her electrical
% engineering major. But, the combination of feelings she had---consisting
% partly of a feeling that programming did not come easily to her---pushed
% her to think about withdrawing from the course anyway. And, Rebecca was
% not in the habit of withdrawing from classes. Intermediate Programming
% was the first and only class where she was experiencing this level of
% difficulty (Interview 4 of 5, April 6, 2012).
%
% It would be tempting to think Rebecca struggled in Intermediate
% Programming because she didn't work hard enough. But, by her own
% assertions she \emph{did} put a considerable amount of effort on
% projects. On Project 2, for instance, she said ``I worked every day from
% the day we got this project on this project, and my project still didn't
% turn out perfect'' (Interview 4 of 5, April 6, 2012). Figure 5 below
% shows Rebecca's activity over time from March 5, the day Project 2 was
% handed out, to March 28, the day it was due:
%
% \includegraphics[width=5.73333in,height=3.98889in]{media/image10.emf}
%
% \protect\hypertarget{ux5fToc252445971}{}{}Figure ‑ -- Rebecca's
% compilation activity over time for Project 2. Each compile a student
% initiates creates a commit, so long as there has been a change to the
% underlying code. The height of each bar maps to the number of commits
% recorded that day. The red line charts \emph{cumulative} commits over
% time. The red line is steepest in periods of frequent activity and
% shallowest in periods with little or no compile activity.
%
% From the data we can tell Rebecca was compiling Project 2 code almost
% every day between March 19 and March 26. Moreover, roughly two thirds of
% compilations for the entire project happened in the last 3 days leading
% up to the deadline. So, at the very least our data shows she was
% actively compiling her code for 8 out the last 10 days.
%
% Rebecca registered no commits between March 12 and March 19. There were
% also other days for which no commits were recorded. One possible cause
% for those zero values is our data collection system was faulty. A second
% possible explanation is that Rebecca was not working. But, a third
% explanation is that Rebecca was working but not compiling. This third
% explanation actually seems most plausible. It fits the pattern of
% Rebecca's work on project 1, where she wrote dozens of lines of code
% before ever compiling (Interview 2 of 5, February 17, 2012). It also
% accounts for Rebecca's lack of compilations despite her claim that she
% worked on Project 2 every day.
%
% So, we can be fairly certain Rebecca was spending time on
% projects.\footnote{The specific nature of \emph{what} Rebecca's work
%   looked like on Project 2 is the subject of the first study of this
%   dissertation, so I don't expand upon it substantially in this study.}
% Another possibility for why Rebecca was struggling might be that she
% wasn't studying. To address that conjecture, we have to step back to
% when I asked Rebecca whether other classes --- classes other than
% Intermediate Programming --- had ever pushed her close to withdrawing
% from them. She said no.
%
% Rebecca: the thing is, I *don't* know how to like, study for
% programming. Cuz like, I look at code and stuff, and I try, and I work
% every day on it, but like, even when---I dunno, it just, does not come
% naturally.
%
% Interviewer: So, what is it like when you study it? What do you usually
% try to do?
%
% Rebecca: Uhh, a lot of times, like, I'll read through the notes we took
% in class, like, cuz, what we're doing now---we have a project, uh, we're
% doing linked lists, and like, malloc'ing data, which is, uh, *you*
% storing it in, as you go. And, like, I get the theory and everything
% behind it, and, like it makes sense what it does, but, I just like try
% and look at his code examples and stuff, and try and replicate it, but I
% don't know what's happening right now, but it's not doing it correctly.
% I just, like, my *biggest* problem, I realized, in programming, is I
% think I have the logic, my logic just doesn't transfer to code. Or I
% don't know *how* to transfer it correctly.
%
% Interviewer: So, um. when you're spending time studying, um, do you
% spend most of it on your computer trying to actually program or---
%
% Rebecca: A lot of it is on my computer and a lot of it is also trying to
% just look at his code and notes in class. Uh, that's probably about
% split, 50/50. Uh, because, sometimes I'll just end up staring at my
% computer screen, like, ``what do I even type in to try?'' So, I try and
% go look at his examples, so.
%
% (Interview 4 of 5, April 6, 2012)
%
% There are two striking features of this exchange.\footnote{There are
%   actually three striking features. The one I left out above is that
%   Rebecca describes malloc'ing as \emph{you} storing in data. I think a
%   great deal of Rebecca's difficulty with malloc specifically hinges on
%   what she means by ``you'' storing it in.} The first is that Rebecca
% felt she didn't know how to study for programming. The second striking
% feature of this exchange is how Rebecca elaborated on what it might mean
% for programming to come naturally to her.
%
% Rebecca feels programming doesn't come naturally to her because she
% routinely finds herself unable to transfer her ``logic'' to code. As a
% follow-up to Rebecca's answer, I asked what ``logic'' meant to her in
% this case.
%
% Rebecca: Like, like how to get the---like, if I need to get a certain
% data, like I have a process of how I want to get it, I ju---and I'm
% like, ``Oh, you would just do this,'' but can the *computer* do that?
% Can I---and it's my---me being able to tell the computer to do that is
% where I get lost. (Interview 4 of 5, April 6, 2012)
%
% The tension Rebecca is describing is one I see as strongly
% epistemological. First, it's epistemological in the sense that it
% reflects a struggle to articulate knowledge of \emph{how to do
% something}. Papert (1980) and Abelson \& Sussman (1996) would describe
% such a struggle as a pertaining to \emph{procedural epistemology}: the
% struggle concerns a particular kind of knowledge work in which the goal
% is not to describe what is, but rather how to. The second sense in which
% it's an epistemological struggle is closer to how Hammer, Elby, and
% colleagues use the word epistemological: what does students' activity
% reveal about how they orient toward knowledge and knowing in a
% discipline (Hammer et al., 2005; Hammer \& Elby, 2002, 2003; Scherr \&
% Hammer, 2009). In this second sense, we can observe that Rebecca defines
% her difficulty in terms of the creation of proper code. We can also pose
% the question ``if Rebecca struggles to transfer logic to code, what
% directs her activity as she \emph{tries} to transfer logic to code?''
%
% Later in the interview, she compared the iTunes project --- on which she
% was stuck --- to Project 2.\footnote{Project 2 did not require students
%   to use malloc() to dynamically allocate memory. Project 3 did.} In so
% doing, she offered another telling idea about the relationship between
% her logic and ``syntax'':
%
% Rebecca: Like, cuz, I guess I feel like I did a lot better on the last
% project because it was so similar to stuff that I'm used to /Mmmhmm/ um,
% whereas all this is brand new and I still don't have any of the syntax
% down yet. /Mmmhmm/ Like, if-statements and while-loops, they make sense
% to me, whereas I'm still like, trying to grasp at this stuff.
%
% By ``this stuff,'' the context of the conversation suggests Rebecca was
% referring to dynamic memory allocation and malloc, which we had been
% discussing.
%
% It seems sensible to think if-statements and while-loops might ``make
% sense'' to Rebecca because their syntactical form aligns strongly with
% their computational operation.\footnote{To borrow terminology from
%   Sherin (2001) and rebut Saussure (1986) at the same time, the symbol
%   template for an if-statement is strongly connected to its conceptual
%   schema because we so often use everyday language in a manner
%   concordant with an if-statement's branched control flow. In this
%   instance, the relationship between signifier and signified is the
%   \emph{opposite} of arbitrary. The similarity between the syntax for an
%   if-statement and an understanding of what that statement does is by
%   design, because C itself was a designed language.} By that, I mean in
% principle the designers of C could have chosen \emph{any} symbol to
% represent branched control flow, but they chose a word --- if --- that
% invites connections to its meaning in everyday speech. If-statements in
% C split control flow based on a Boolean expression's value; while-loops
% allow continued iteration until a Boolean expression turns up false.
% Both structures have everyday analogues with similar conceptual
% properties. A parent might say, ``if you're staying out past 9, let me
% know,'' or, ``while we're gone, can you make sure Sadie doesn't eat
% anything from the garbage?'' If a child will be home by 8, there is no
% need to let that parent know. Similarly, it's reasonable to expect that
% once the parents return, the responsibility of keeping Sadie out of the
% garbage no longer falls solely to the child.\footnote{Some parents might
%   object that family members should \emph{always} be on the lookout for
%   Sadie rooting through the trash, that these parents were just
%   emphasizing the point. And, reasonable children might in general try
%   to heed that. But, what makes computers computers is their lack of
%   qualms about abandoning any and all responsibility to keep Sadie from
%   the trash once the parents return.} The point of this observation is
% that Rebecca may feel certain kinds of programming structures ``make
% sense'' not simply because they were covered in introductory
% programming, but because something about their syntactical form offers
% an affordance for thinking about how they work.
%
% \subsubsection{Rebecca has productive capacities for making progress on
% software design
% work}\label{rebecca-has-productive-capacities-for-making-progress-on-software-design-work}
%
% In our fourth of five interviews, Rebecca was having trouble making
% progress on Project 3, which the instructor described as an ``online
% music server.''\footnote{Students in my study colloquially referred to
%   this project as the ``iTunes project,'' so ``the iTunes project'' and
%   ``Project 3'' all refer to this project.} ``Like, I am so lost, like I
% don't even know where to start,'' she said. We discussed why she was
% stuck. Then, with my help, we began to work through what Rebecca
% identified as a core cause of her confusion. In brief, the assignment
% contained a diagram showing the data scheme students were required to
% use to store data about the music (Figure 8 below). Specifically, the
% scheme required an array of pointers --- each entry of which pointed to
% an array of characters --- as a mechanism for storing data about song
% titles for the music server. Rebecca's work to resolve her confusion
% offers evidence for her productive capacities in designing programs. In
% this section, I argue:
%
% \begin{enumerate}
% \def\labelenumi{\arabic{enumi}.}
% \item
%   Rebecca made sense of the array-of-pointers design\footnote{By
%     ``design scheme,'' here, I mean the instructor-provided diagram
%     indicating how students were to structure their data for Project 3.}
%   using talk and gesture. She was able to explain both why an array of
%   pointers \emph{should} exist in C as well as suggest candidate syntax
%   for what might create such an array.
% \item
%   When I proposed we pretend her candidate syntax worked, Rebecca
%   articulated through gesture, talk, and written pseudo-code how part of
%   her design would incorporate that syntax.
% \item
%   After the interview, Rebecca resumed work on the project and
%   incorporated a version of the design she developed in the interview
%   into her code.
% \end{enumerate}
%
% \paragraph{Rebecca made sense of the array-of-pointers design using talk
% and
% gesture}\label{rebecca-made-sense-of-the-array-of-pointers-design-using-talk-and-gesture}
%
% The figure below shows the instructor-provided diagram for how students
% should structure their data.
%
% \includegraphics[width=5.36667in,height=4.10000in]{media/image11.emf}
%
% \protect\hypertarget{ux5fToc252445972}{}{}Figure ‑ -- The instructor
% required students to use this data arrangement for storing information
% in the music server. The scheme uses an array of pointers to represent
% the track names of an album.
%
% Rebecca would have seen this diagram before because the project
% assignment was given out 11 days prior to our interview. But, the data I
% present is the first time I had any access to how she was thinking about
% it.
%
% At the time of the interview Rebecca felt stuck on handling issues with
% \emph{music albums} --- the top-left and top-right structures depicted
% on the diagram. The assignment required that ``all of the data
% structures described {[}in this diagram{]}'', including music albums and
% all of the data they referenced, ``must be allocated dynamically''
% (Instructor-provided Project 3 description). So, Rebecca and I started
% discussing what parts of that scheme she understood. Table 3 below
% presents what she said and what she did while beginning to explain her
% understanding.\footnote{}
%
% \protect\hypertarget{ux5fToc252445961}{}{}Table - Rebecca describes an
% array of structures using gestures (Interview 4 of 5, April 6, 2012)
%
% \begin{supertabular}[]{@{}ll@{}}
% \toprule
% What she says & What she does\tabularnewline
% \midrule
% \endhead
% Like, the---the thing that makes sense to me that I, uh, got right now,
% is that, to malloc the array, to \{bounds a wide space with her hands\}
% make the array a certain length, do that. &
% \includegraphics[width=2.03333in,height=2.72222in]{media/image12.png}\tabularnewline
% And then I know that each of the \{chops out three invisible boxes in
% the air\} I guess nu---characters---I---uh, I'll call `em characters. &
% \includegraphics[width=1.37778in,height=3.66667in]{media/image13.emf}\tabularnewline
% Like, each of the spots in the array holds a structure and each of the
% structures holds three things. & (gestures obscured by the computer
% )\tabularnewline
% \bottomrule
% \end{supertabular}
%
% Rebecca's gestures create a kind of virtual object that looks much like
% the music albums array in the diagram. As she chops out spaces for each
% of the ``characters,'' those spaces are in series and fit --- more or
% less --- within the larger area she bounded with her when she said ``to
% malloc the array.'' The spatial subdivision is key because it offers
% supporting evidence that in some way, Rebecca might have been thinking
% about the album structures themselves as being \emph{contained} in or by
% the array. And, that spatial containment metaphor resurfaced later when
% we discussed the array of pointers, which I'll get to in a moment.
%
% As Rebecca explained her understanding to me, I took to writing down
% what I understood her to be saying on paper:
%
% Interviewer: OK. So you have, so here's what you just said. You said
% \{writes\} ``malloc the array'' um, ``each element has a structure,''
% and then ``each structure has'' you said---
%
% Rebecca: Has three parts
%
% Interviewer: OK. Has three parts. OK.
%
% Rebecca: Yeah. So, and the first part I can do `cuz it's an integer
% \{laughs\}. The first part is just the number of tracks on the album.
% So---
%
% Interviewer: Oh, oh. So, of the three parts {[}{[}I'm sorry
%
% Rebecca: Yes{]}{]}
%
% Interviewer: I {[}{[}\{untillegible\}
%
% Rebecca: Oh yeah, sorry{]}{]}
%
% Interviewer: So you're saying, um, and then---and then this is number of
% tracks, which is an integer, right? /yeah/ OK (Interview 4 of 5, April
% 6, 2012)
%
% Figure 9 below shows what I wrote as we worked:
%
% \includegraphics[width=5.62222in,height=1.48889in]{media/image14.emf}
%
% \protect\hypertarget{ux5fToc252445973}{}{}Figure ‑ -- I write down what
% I understand as Rebecca explains the overall data structure of albums to
% me
%
% It wasn't the first time --- either in my series of interviews with
% Rebecca or in that interview in particular --- that I tried to write
% down what she said. But, I highlight it here because it became part of
% my activity as we worked together in this episode. Rebecca would
% articulate something, often through talk and gesture, and I would try to
% write down and offer back to her what I understood her to be explaining.
%
% As we continued, we landed on trying to understand the ``tracks''
% portion of the album data structure. Storing tracks was where Rebecca
% felt she was getting ``lost'':
%
% \protect\hypertarget{ux5fToc252445962}{}{}Table -- Rebecca says gets
% lost on the array of pointers pointing to track names (Interview 4 of 5,
% April 6, 2012)
%
% \begin{supertabular}[]{@{}ll@{}}
% \toprule
% What she says & What she does\tabularnewline
% \midrule
% \endhead
% Rebecca: And, the next one is where I get lost, on the tracks part.
% Because it's a pointer \{2 sec pause\} to \{hands make a vertical
% cylinder in front of her; note that the array is represented on the
% assignment as a narrow, vertical rectangle\}--- &
% \includegraphics[width=3.40000in,height=2.62222in]{media/image15.emf}\tabularnewline
% it looks like another array that's
% \textbar{}\textbar{}pointing\textbar{}\textbar{} \textbar{}\{left hand
% crosses from her left to right, at chest height, index finger extended
% in the direction of motion; this is the same direction (from Rebecca's
% perspective) in which the pointer array on paper points to each track
% name\}\textbar{} to the names. &
% \includegraphics[width=1.98889in,height=3.38889in]{media/image16.emf}\tabularnewline
% \bottomrule
% \end{supertabular}
%
% Rebecca's gestures for the array of pointers again spatially mimic the
% depiction of the array on the page: a vertical cylinder. The reason I
% note the orientation of her array gestures is that for practical
% purposes arrays as represented by bits in C have no orientation in space
% that the programmer could know, much less care about. An array cannot be
% parallel to the ground, nor can it be oriented toward the
% ceiling.\footnote{The computational hardware representing the array does
%   have an orientation in space, but that's clearly not what is at issue
%   here. The whole point of the array is to be an abstraction away from
%   the soldered transistors, capacitive plates, flipped switches, vacuum
%   tubes, tinker toys, or other physical means that store the state of
%   the array.} But, the assignment \emph{depicted} the album array as
% horizontal, and so too did Rebecca's gestures for it. The assignment
% \emph{depicted} the pointer array as vertical, and so too did Rebecca's
% gestures for it. In other words, the observation that Rebecca's gestures
% align with the assignments pictures might suggest that she is doing more
% than thinking about arrays in the abstract. Rather, some elements of the
% representational features of the page have made their way into her
% activity and, arguably, her cognition.\footnote{One could argue that
%   gesture is possibly just undirected flailing, and that the correlation
%   between Rebecca's gestures and the page is weak at best --- the sort
%   of thing you'd expect to happen some percentage of the time anyway
%   under the assumption that the pictures on the page have nothing to do
%   with how she's thinking about it. I concede that it's \emph{possible}
%   the correlation isn't structurally meaningful. But, I think the
%   correlation is meaningful given how many other times her gestures
%   align with canonical written representations of computational
%   structures and processes.}
%
% Rebecca continued on to try thinking through why an array of pointers
% was part of the design.
%
% Rebecca: But, I've always been confused as to why you---I guess, I
% always---I was always like, when we learned pointers, I was like ``why
% do you need pointers when you could just call it the name? Why do you
% need two names for it?'' But, I think, what---at least what I'm seeing
% here maybe is like, this is just another array /Mmmhmm/ because you
% can't put more than one character in each element of the array, so that
% is just an array of pointers that point to strings. And the strings are
% the names.
%
% Rebecca: So, that would make sense. So you would---the tracks would
% point to the array, and so you could do the pointer of that \{1.5 sec
% pause\} The pointer to the array, an array---the array of like 1 would
% be a pointer to track 1's name.
%
% Interviewer: Mmmhmm
%
% Rebecca: So. OK, so---see that part makes sense now. I just---no idea
% how I would ever access that. Or, store it, I guess, is the better word.
% (Interview 4 of 5, April 6, 2012)
%
% What ``confused'' Rebecca has an understandable origin. Most of the work
% she would have been doing until that point would have used named
% variables. For example:
%
% int age;
%
% float weight;
%
% char *name;
%
% If age, weight, and name are already-named pieces of data, why would
% something ever need to refer to them indirectly? It would be as if we
% had to call me ``The second-born son of the second-born daughter of
% Henry'' for some reason. I already have a name. Why do I need another
% name that uses the names and relationships of my forebears to refer to
% me when historically my given name has perfectly fine for all
% referential purposes?
%
% Rebecca's way of finding sense in pointers was to observe that you can't
% put names --- each of which is defined as an array of characters ---
% into an array because ``you can't put more than one character in an
% array.'' To verify I was understanding her idea correctly, I tried to
% restate it:
%
% Interviewer: OK, so you were saying the reason you---when you're tryin
% to convince yourself like---you originally you were thinking ``why don't
% you just put all the names {[}{[}here
%
% Rebecca: Yeah{]}{]}
%
% Interviewer: And you were {[}{[}saying the reason is
%
% Rebecca: saying, no{]}{]}
%
% Interviewer: because==
%
% Rebecca: ==That makes sense now. Yeah, you can't put more than one
% character in an element. And, it's a whole name of a song, so you would
% need to point---that element would just \{gesture obscured by my
% laptop\} be a pointer to a name. OK. (Interview 4 of 5, April 6, 2012)
%
% The latching and overlapping turns of talk above offer strong evidence
% that Rebecca's explanation aligned with my interpretation. And again,
% notions of \emph{containment} come into play. Rebecca says, ``you can't
% \emph{put} more than one character \emph{in} an element'' in part
% because you're talking about ``a \emph{whole name} of a song'' (my
% emphasis added here). If Rebecca has a capacity for thinking about the
% canonical concept of type-restrictions on arrays, it's manifesting here
% as a part-physical metaphor, evoked through both talk (the quote above)
% and gesture (Table 4), in the service of making sense of a design
% decision.
%
% As she continued, Rebecca discerned that because the input data
% contained the number of tracks on each album, she could use that number
% to define how long the pointer array needed to be. But, something was
% still troubling her.
%
% Rebecca: So. \{4 sec pause\} That would \{6 sec pause\}
%
% Interviewer: What're you thinkin?
%
% Rebecca: Just that, um, I'm just tryna think, cuz like that makes sense,
% but I don't know how to get to that. Cuz, how do you make a pointer
% point to a random array that you just made? Uh, I guess \{4 sec pause\}
%
% Rebecca: Uh, p---the pointer syntax I need---I would need to go back and
% look at them because now you have an array, each array---my issue---cuz
% like now that makes sense. Like, it's just how do I get to that point of
% making that work?
%
% Interviewer: {[}{[}So
%
% Rebecca: And this is what{]}{]} I mean by my logic /OK/ I get the logic
% and the theory behind it but I don't know how to actually put it into C.
%
% Interviewer: OK, so in other words, what makes
% \textbar{}\textbar{}sense\textbar{}\textbar{} \textbar{}\{air
% quotes\}\textbar{} to you know /Mmmhmm/ that I guess didn't before is
% that *that* \{gestures to the instructor's structural diagram on
% screen\} is a---that's an understandable arrangement /yes/ for the data
%
% Rebecca: Yes.
%
% Interviewer: But then, how do I {[}{[}make it
%
% Rebecca: make it{]}{]} Yeah \{nods\}. (Interview 4 of 5, April 6, 2012)
%
% When Rebecca said ``I don't know how to get to that,'' I take ``how to
% get to that'' to mean \emph{how to write the code for that procedure}.
% We had just agreed that design-wise, she could use a number provided in
% the input data to define the length of the pointer of array. But, she
% was stuck because of her difficulty translating the ``logic'' we
% developed into code.
%
% From the outside, what Rebecca and I had done together was something I
% think crucially important to the design of a program. We worked to
% understand how someone else chose to structure their data.\footnote{That
%   the ``someone else'' was in this case the instructor is important for
%   larger power dynamic considerations.} We tried to decide for ourselves
% whether and how those structures made sense. Along the way we created
% ephemeral objects of that sense-making in the form of talk and gesture
% as well as the durable artifacts of my inscriptions. But, none of that
% work resulted in C code.
%
% For Rebecca, the lack of certainty about \emph{how to make} our ideas in
% C code was a marked concern. And, again, that concern makes sense. We
% had just convinced ourselves that we could understand \emph{why} an
% array-of-pointers. We even figured out \emph{how} we could use the input
% data to provide the length of the arrays we would need to make. But we
% had not laid out how to declare the array \emph{and} preserve a
% reference to it (``how do you make a pointer point to a random array
% that you just made?). Without that piece --- a piece Rebecca was unsure
% of and felt she would need to look up --- our solution clearly wouldn't
% work.
%
% Absent that piece, Rebecca tried thinking of possible ways to fill in
% the missing syntax.
%
% Rebecca: Cuz, I'm like, like I'm trying to think right now of things
% that I would try /Mmmhmm/ but I don't know what I would try first
% because, all I'm thinking right now is that I know tracks is going to be
% a pointer. /Mmmhmm/ So you make that a pointer.
%
% Interviewer: OK.
%
% Rebecca: But I'm not sure where it would point to, because you don't
% have that array made yet. So I \{3 sec pause\} so you'd have to, I guess
% you could make the array, and then make it point to the array, but \{3
% sec pause\} Cuz you have to make an array of pointers.
%
% Interviewer: Mmmhmm
%
% Rebecca: And I don't know how to do that. Because I've made arrays of
% characters and integers and stuff before /OK/ but never pointers, so.
% (Interview 4 of 5, April 6, 2012)
%
% Rebecca was seeing a temporal problem: how could she reference an array
% that didn't exist yet?\footnote{As a crude analogy, imagine trying to
%   write your own will. If you think forward in time, you might later
%   have children to whom you'd like to bequeath your assets. But how can
%   you enumerate those children in the will now, before you've had them?
%   You need your will to refer to something that does not exist at the
%   time you write the will. Extrapolating forward and rather ludicrously,
%   we could imagine you are always capable of producing children. So, up
%   until the day of your death, there is always the seeming risk that
%   your will might need to refer to a beneficiary who does not yet exist.
%   Recognizing this subtle problem as Rebecca did is an act I consider to
%   be sense-making, even though she as-yet still did not have a solution.}
%
% Rebecca started exploring ways to solve that problem of getting tracks
% to point to the array of pointers.
%
% Rebecca: But I'm not sure where it (tracks) would point to, because you
% don't have that array made yet. So I \{3 sec pause\} so you'd have to, I
% guess you could make the array, and then make it point to the array, but
% \{3 sec pause\} Cuz you have to make an array of pointers.
%
% Interviewer: Mmmhmm
%
% Rebecca: And I don't know how to do that. Because I've made arrays of
% characters and integers and stuff before /OK/ but never pointers, so.
% (Interview 4 of 5, April 6, 2012)
%
% Rebecca landed on a potential solution: make the array first, then make
% tracks point to the array. But, with that solution came another problem:
% how do you make an array of pointers?
%
% \paragraph{Rebecca could propose candidate syntax and build pseudocode
% around
% it.}\label{rebecca-could-propose-candidate-syntax-and-build-pseudocode-around-it.}
%
% Rebecca came up with an idea to create an array of pointers, but she
% wasn't sure if it would work.
%
% Rebecca: if I were to d---when I declared the array, if I were to---I
% dunno if you even can, like, whenever we---we do ints, and then it would
% be like star-p /Mmmhmm/ and then star-p would be the pointer /ok/ I
% don't know if you can do like int star-p
% \textbar{}\textbar{}bracket-bracket\textbar{}\textbar{}
% \textbar{}\{makes square brackets with hands\}\textbar{} and that would
% be an array or not.
%
% Interviewer: {[}{[}OK
%
% Rebecca: But I don't know{]}{]} if you can do that.
%
% Notice what's happening here. Rebecca is taking patterns that she knows
% work:
%
% int age; // creates an integer variable called age
%
% int ages{[}35{]}; // creates an array called ages,
%
% // 35 integers long
%
% int *p; // creates a pointer-to-an-integer;
%
% // said pointer is called p
%
% and considering a pattern that might work:
%
% int *p{[}35{]}; // Rebecca is not exactly sure what this will do
%
% Notice further that in this exchange, Rebecca wondered ``if you can do
% that.'' Determining the referent of \emph{that} was, again,
% consequential. One possibility was Rebecca meant ``int *p{[}35{]}''
% might not be the proper syntax for creating an array of 35
% pointers-to-ints. If so, Rebecca was struggling with her aforementioned
% problem of not being able to translate her logic to code. Another
% possibility was Rebecca meant ``I don't know if there exists \emph{any}
% syntax in C to create an array of pointers.'' So, I followed up on this
% latter interpretation. We both agreed it \emph{should} be possible to
% create an array of pointers, it was indeed just a question of whether
% that syntax accomplished it (Interview 4 of 5, April 6, 2012). So, that
% left the former interpretation: it looked like Rebecca was unsure how to
% ``transfer her logic to code.''
%
% We both agreed creating an array of pointers should be possible. So, I
% asked Rebecca to suppose her candidate syntax worked. She readily
% agreed.
%
% Interviewer: OK. So then, um, what if we just pretended for a minute,
% that that, like==
%
% Rebecca: ==That that works==
%
% Interviewer: ==That that worked==
%
% Rebecca: ==OK==
%
% Interviewer: ==OK. So then, um. \{begins writing\} So then you might
% write like to make an array of \{stops writing\} actually what would we
% call this? This is==
%
% Rebecca: ==array of pointers, I guess==
%
% Interviewer: ==OK. \{under breath\}
% \textbar{}\textbar{}pointers\textbar{}\textbar{} \textbar{}\{writes
% ``pointers''\}\textbar{}. It'd be, well. So. The---the one we had was
% like \textbar{}\textbar{}int, star p, um, it would be some
% number\textbar{}\textbar{} \textbar{}\{writing\}\textbar{}, um, and I
% guess that'd be it in order to just declare /Yes/ that right? OK.
%
% In this exchange note every turn boundary is latched. Once I bid for
% pretending, Rebecca accedes. When I'm unsure what to call the array,
% Rebecca finishes my sentence. Rebecca rapidly approves the final product
% before I even finish my last sentence. Figure 10 below is the syntax I
% wrote in the exchange.
%
% \includegraphics[width=5.12500in,height=1.09792in]{media/image17.emf}
%
% If that syntax created the array-of-pointers, the next challenge for
% storing track names was how to access elements in array-of-pointers.
% Figure 11 presents the pseudo-code Rebecca ultimately wrote to scan in
% and store track names.
%
% \includegraphics[width=5.36667in,height=1.30000in]{media/image18.emf}
%
% \protect\hypertarget{ux5fToc252445974}{}{}Figure ‑ -- Rebecca's
% pseudo-code for scanning in and storing track names
%
% The next section will show the rich talk and gestures that accompanied
% Rebecca's generation of this pseudo-code. I argue activity in the form
% of gesture and talk constituted and sustained Rebecca's in-the-moment
% approach to programming.
%
% \paragraph{Rebecca's talk and gestures constituted and sustained her
% in-the-moment approach to
% programming}\label{rebeccas-talk-and-gestures-constituted-and-sustained-her-in-the-moment-approach-to-programming}
%
% Throughout this section I detail the speech and gestures that accompany
% Rebecca's pseudocode production. Along both modalities I saw evidence of
% a marked shift in Rebecca's orientation. It was a shift from being
% hunched, silent, thinking over the keyboard to being animated, open, and
% gestural toward me. It was also a shift from seeming diffident,
% uncertain, and hedging in speech to seeming confident, assured, and able
% to dispatch my questions. Taken in their totality across the episode,
% these subtleties constitute evidence of an approach that strongly
% differs from those Rebecca had when discussing her struggles in the
% course (section 3.5.1).
%
% First, Rebecca suggested she could write the pseudo-code as we talked.
%
% \protect\hypertarget{ux5fToc252445963}{}{}Table -- Rebecca starts
% talking out pseudo-code and asks for the pen to write it
%
% \begin{supertabular}[]{@{}ll@{}}
% \toprule
% What she said & What she did\tabularnewline
% \midrule
% \endhead
% Uh, well, what I was think---like if I wanted to like,
% \textbar{}\textbar{}save it\textbar{}\textbar{} \textbar{}\{pinches
% right hand, thumb and index finger a few inches apart\}\textbar{} or
% whatever, /Yeah/ &
% \includegraphics[width=2.68889in,height=3.23333in]{media/image19.png}\tabularnewline
% I could make a while loop, \textbar{}\textbar{}scan in
% the\textbar{}\textbar{} \textbar{}\{drags hand across the
% table\}\textbar{} ---scan in the data from the album, /OK/ \{interviewer
% begins writing\} uh, which is, {[}{[}I can write it if you want &
% \includegraphics[width=2.66667in,height=3.23333in]{media/image20.png}\tabularnewline
% \bottomrule
% \end{supertabular}
%
% Rebecca then took the pen and started describing her thinking while she
% wrote.
%
% \protect\hypertarget{ux5fToc252445964}{}{}Table -- Rebecca's
% verbal/gestural overview of looping through the input track titles
% (Interview 4 of 5, April 6, 2012)
%
% \begin{supertabular}[]{@{}lll@{}}
% \toprule
% Panel \# & What she said & What she did\tabularnewline
% \midrule
% \endhead
% 1 & And then my thinking at least, is you should be able to, um, say
% that
%
% ``star p of i'' /mmhmm/ equals, uh, the title, and then you just do i++,
% so then it'll \textbar{}\textbar{}move to the next
% one\textbar{}\textbar{} \textbar{}\{makes looping gesture with left
% hand\}\textbar{} /OK/ &
% \includegraphics[width=2.03333in,height=2.33333in]{media/image21.png}\tabularnewline
% 2 & and you just keep \textbar{}\textbar{}saving each of the
% pointers\textbar{}\textbar{} \textbar{}\{left hand makes horizontal
% chops in the air, like rungs down a ladder\}\textbar{} &
% \includegraphics[width=3.40000in,height=1.92222in]{media/image22.emf}\tabularnewline
% 3 & in the array \textbar{}\textbar{}to a title\textbar{}\textbar{}
% \textbar{}\{right hand makes pinching motion, left-hand points to
% it\}\textbar{} &
% \includegraphics[width=2.18889in,height=2.24444in]{media/image23.png}\tabularnewline
% 4 & \textbar{}\textbar{}And\textbar{}\textbar{} \textbar{}\{right hand
% makes a large loop counterclockwise (from Rebecca's
% perspective)\}\textbar{} &
% \includegraphics[width=3.13333in,height=1.54444in]{media/image24.emf}\tabularnewline
% 5 & \textbar{}\textbar{}you just increment by 1\textbar{}\textbar{}
% \textbar{}\{right hand makes a cycloid to her right, in a plane parallel
% to the back wall, like counting dots on a line\}\textbar{} until you
% reach the end of file. &
% \includegraphics[width=1.06667in,height=3.62222in]{media/image25.emf}\tabularnewline
% \bottomrule
% \end{supertabular}
%
% ``Like, that would make sense to me,'' Rebecca said. I asked whether the
% while loop would get a fresh line of text when it runs again.
%
% \begin{supertabular}[]{@{}lll@{}}
% \toprule
% Panel \# & What she said & What she did\tabularnewline
% \midrule
% \endhead
% 1 & Yeah, because{]}{]} uh, \textbar{}\textbar{}what the while loop
% does\textbar{}\textbar{} \textbar{}\{pinches right thumb and
% forefinger\}\textbar{} &
% \includegraphics[width=1.74444in,height=1.95556in]{media/image26.png}\tabularnewline
% 2 & \textbar{}\textbar{}is it reads in the line\textbar{}\textbar{}
% \textbar{}\{right hand, palm down, chest height, slides out to the
% right\}\textbar{} &
% \includegraphics[width=1.76667in,height=2.83333in]{media/image27.emf}\tabularnewline
% 3 & and then \textbar{}\textbar{}once it reaches\textbar{}\textbar{}
% \textbar{}\{right hand rises, lowers in a chop\}\textbar{} &
% \includegraphics[width=1.78889in,height=2.11111in]{media/image28.png}\tabularnewline
% 4 & the end of \textbar{}\textbar{}character it'll\textbar{}\textbar{}
% \textbar{}\{right hand rises, swoops down in a crescent, index finger
% extended, looping back up\}\textbar{} &
% \includegraphics[width=2.74444in,height=1.36667in]{media/image29.emf}\tabularnewline
% 5 & go back down---it'll \textbar{}\textbar{}do the while
% loop\textbar{}\textbar{} \textbar{}\{right hand moves back down toward
% table\}\textbar{} \textbar{}\textbar{}and then\textbar{}\textbar{}
% \textbar{}\{right hand moves back up above shoulder-height,
% palm-down\}\textbar{} &
% \includegraphics[width=2.63333in,height=1.55556in]{media/image30.emf}\tabularnewline
% 6 & \textbar{}\textbar{}go back\textbar{}\textbar{} \textbar{}\{right
% hand creates a loop down and back up\}\textbar{} down to the next
% thing---line, &
% \includegraphics[width=3.38889in,height=1.31111in]{media/image31.emf}\tabularnewline
% 7 & it'll \textbar{}\textbar{}read in the line\textbar{}\textbar{}
% \textbar{}\{right hand scans rightward at shoulder height\}\textbar{} &
% \includegraphics[width=1.61111in,height=2.47778in]{media/image32.emf}\tabularnewline
% 8 & \textbar{}\textbar{}until it\textbar{}\textbar{} \textbar{}\{pinches
% right thumb and forefinger, swipes right hand back to the left, palm
% down\}\textbar{} reaches the end of file &
% \includegraphics[width=3.38889in,height=1.45556in]{media/image33.emf}\tabularnewline
% \bottomrule
% \end{supertabular}
%
% In panel 4, notice that Rebecca's gesture of swooping back actually
% \emph{precedes} her saying ``go back down.'' In other words, her body
% articulates the idea of cycling back before she actually speaks it. A
% similar argument could be made for panel 8, where Rebecca's gesture for
% a process hitting the bottom (hand swooping and hitting an inflection
% point) precedes her saying ``reaches the end of file''
%
% I tried to sum up my understanding of Rebecca's description:
%
% Interviewer: So as you step through this loop /mmhmm/ \{points to i++\}
% i keeps going up by one==
%
% Rebecca: ==Yes. (Interview 4 of 5, April 6, 2012)
%
% As I thought about that, Rebecca continued explaining and gesturing:
%
% \protect\hypertarget{ux5fToc252445965}{}{}Table -- Rebecca concludes her
% visual and gestural explanation for scanning in track titles (Interview
% 4 of 5, April 6, 2012)
%
% \begin{supertabular}[]{@{}lll@{}}
% \toprule
% Panel \# & What she said & What she did\tabularnewline
% \midrule
% \endhead
% 1 & And the \textbar{}\textbar{}lines keep going
% down\textbar{}\textbar{} \textbar{}\{left hand horizontally chops the
% air, creating ``ladder rungs''\}\textbar{} & (Gesture partially obscured
% by computer )\tabularnewline
% 2 & so, \textbar{}\textbar{}that way,\textbar{}\textbar{}
% \textbar{}\{left hand raises up to her head, palm down and parallel to
% the table, and it is replaced by right hand, index finger extended and
% pointing to her left\}\textbar{} &
% \includegraphics[width=2.10000in,height=2.60000in]{media/image34.emf}\tabularnewline
% 3 & \textbar{}\textbar{}the first line\textbar{}\textbar{}
% \textbar{}\{right hand scans across to her right\}\textbar{} &
% \includegraphics[width=1.57778in,height=3.50000in]{media/image35.emf}\tabularnewline
% 4 & is going to \textbar{}\textbar{}be\textbar{}\textbar{}
% \textbar{}\{right hand rotates to become pinched thumb and forefinger,
% palm out\}\textbar{} , \textbar{}\textbar{}uh the
% p---\textbar{}\textbar{} \textbar{}\{right-hand wiggles\}\textbar{} &
% \includegraphics[width=3.25556in,height=1.62222in]{media/image36.emf}\tabularnewline
% 5 & element zero /OK/ uh \textbar{}\textbar{}the
% second\textbar{}\textbar{} \textbar{}\{hand swoops slightly up, then
% curves down and locks in in a position below the prior one\}\textbar{}
% one'll be element one &
% \includegraphics[width=3.25556in,height=1.55556in]{media/image37.emf}\tabularnewline
% \bottomrule
% \end{supertabular}
%
% \subsection{Conclusion}\label{conclusion-1}
%
% In pulling together concluding ideas, I revisit the points I established
% at the end of section 3.1.
%
% \subsubsection{Students' early-stage design activity reveals patterns
% outside the explanatory scope of (mis)conceptions
% accounts}\label{students-early-stage-design-activity-reveals-patterns-outside-the-explanatory-scope-of-misconceptions-accounts}
%
% Lionel's strategy of working at a whiteboard, keeping himself at a
% top-level of planning, and copying pseudo-code into a computer isn't
% explained by appealing to ``concepts'' in computer science, which in
% many research accounts are simply mappings of content in computer
% science curricula. I demonstrate this gap in Appendix 4, where the
% ``conceptual features'' of Lionel's code reveal little (if anything)
% about the resources involved in Lionel's design process.
%
% Rebecca, in a similar fashion, has lots of productive knowledge for
% thinking about arrays (section 3.5.2), but at the time of the interview
% none of her code reflected that knowledge. Moreover, Rebecca was able to
% sense-make (Danielak et al., in press) about an array-of-pointers
% design, where her sense-making was again a complex activity that would
% be poorly accounted for in CSEd frameworks that appeal to
% misconceptions. In Rebecca's case, the situation is particularly
% paradoxical if we consider conceptual knowledge about a topic to naively
% be something students have or don't have.
%
% Suppose Rebecca had the requisite conceptual knowledge for thinking
% about an array of pointers. It's a puzzle, then, to explain solely via
% conceptual knowledge why she was so frustrated with programming and felt
% stuck when she began Interview 4. Why hadn't the knowledge she
% possessed---relevant to solving the problem at hand---manifested already
% some way? Why was she stuck?
%
% The alternative is to assume that as of Interview 4 she didn't have the
% conceptual knowledge for thinking about the problem at hand. But, that
% doesn't make sense either. If Rebecca didn't know how to think about an
% array of pointers, why was she able to do it so well during the
% interview? Both possibilities---that either Rebecca had or did not have
% conceptual knowledge about an array of pointers---lead to fairly
% non-sensical conclusions under naïve have/don't have assumptions of
% conceptual knowledge. Yet, again, the majority of conceptual knowledge
% research in CSEd today is silent on the issue of degree when it comes to
% conceptual knowledge: students either have a mental model that matches
% canonical function or they don't.
%
% \subsubsection{Rebecca had epistemological resources to support
% expert-like practices, but she framed those practices
% differently}\label{rebecca-had-epistemological-resources-to-support-expert-like-practices-but-she-framed-those-practices-differently}
%
% Consider these summative statements made by Lionel and Rebecca on how
% they use pseudo-code in their work.
%
% \protect\hypertarget{ux5fToc252445966}{}{}Table -- Comparing Lionel's
% and Rebecca's views toward pseudo-code
%
% \begin{supertabular}[]{@{}ll@{}}
% \toprule
% Lionel & Rebecca\tabularnewline
% \midrule
% \endhead
% ``on my computer I'd, you know I'd write out the pseudocode, I'd
% ac---I'd literally write out the pseudocode, even though obviously it
% wouldn't compile and actually work.'' (Interview 1 of 1, October 17,
% 2011) & ``So, I can't just type in ``if the white piece reaches in,''
% I---the, the C language, I guess, putting it in those, their terms,
% their terminology into programming language.'' (Interview 2 of 5,
% February 17, 2012)\tabularnewline
% \bottomrule
% \end{supertabular}
%
% As illustrative examples, they capture a fundamental difference in
% practice between Lionel and Rebecca. Lionel worked in pseudo-code on his
% computer, even going so far as to copy it into his source-code files
% despite the fact that it ``obviously wouldn't compile and actually
% work.'' Rebecca also wrote pseudo-code, and she even wrote it into her
% source code files. But, her local sense of the activity of writing
% pseudo-code was different. It came to represent a discontinuity with the
% final code she was trying to write because there was nonetheless a gap
% between writing in English and writing C in ``their terms.''
%
% These diverging stances take the same practice---in this case,
% pseudo-coding---and situate it within a different kind of
% epistemological coherence. For Lionel, the coherence associated with
% pseudo-coding is productive and optimistic. Copying non-working
% pseudo-code into the computer is, in a colloquial sense, all part of the
% plan. It's a legitimate step toward creating a final working program
% and, crucially, it reflects his own understanding of what's supposed to
% happen in his code. For Rebecca, by contrast, writing pseudo-code is a
% fallback. Pseudo-code is what she writes ``if I know what I want to put
% underneath of it,'' by which she means if she's not sure how to flesh
% out a loop or other control structure. Rather than being part of a
% planful coherence, the practice of pseudo-coding for Rebecca gets
% triggered as part of a stopgap coherence; a contingent measure at times
% associated with a kind of diffidence.
%
% Crucially, Rebecca has intellectual resources for pseudo-coding.
% Moreover, her practice of writing pseudo-code to flesh out a loop is the
% plausible beginning of ``method stubbing,'' the process by which one
% might defer implementing the guts of a function and instead create a
% simple stopgap. For example, the function below will always print 28
% Fahrenheit, which is fairly useless as far as temperature-getting
% functions go:
%
% getOutsideTemperature \textless{}- function(sensor) \{
%
% print(``28F'')
%
% \}
%
% But, the advantage of having \emph{something} in the function body, no
% matter how trivial, is enormous. Now, because
% \textbf{getOutsideTemperature} is defined and takes arguments, other
% functions can safely call it. It also offers a diagnostic output---by
% printing ``\textbf{28F}'' out---that we can rely on if we ever need to
% make sure the function was called. At the time of creating
% \textbf{getOutsideTemperature}, a programmer may have no idea how the
% temperature will be obtained but will nonetheless need to be able to
% write code that depends on it getting the temperature. Stubbing is, in
% such cases, a highly productive thing for a programmer to be able to do.
% But, stubbing is as much a strict practice as it is an epistemological
% move, because it represents a decision that an entity-to-be-known can
% be, for the moment, underspecified so it can be incorporated into a
% system before its behavior is defined.
%
% Rebecca is already capable of simple stubbing. It's part of an
% epistemological coherence that's triggered when she doesn't know how to
% flesh out a control structure. So, from a constructivist standpoint
% (Smith et al., 1993), she has knowledge of something that can be further
% refined into software engineering knowledge. Indeed, much of the
% surrounding data from Rebecca suggests that she would have been served
% well if an instructional intervention had taught her some simple ways to
% stub out functions so she could call them even if she hadn't defined
% their bodies yet. To my knowledge, she was never exposed to stubbing
% formally, which is unfortunate considering how much it might have led to
% improving her confidence in programming.
%
% Lastly, consider Rebecca's episode in reasoning about an array of
% pointers (section 3.5.2). As I show, Rebecca clearly had resources for
% reasoning about the sensibility of an array-of-pointers design and for
% predicting array-of-pointers syntax by extrapolating from patterns she
% had seen before. And, when I explicitly suggested that we suppose a
% candidate syntax works, Rebecca was able to fluidly articulate the
% procedure she would build around that syntax (section 3.5.2.3). Rebecca
% knew things, but her knowledge was deployed in such a highly contextual
% way as to be sensitive to what kind of knowledge-activity she thought
% she was supposed to be doing. It's thus sensible to model that set of
% phenomena from an epistemological standpoint, rather than a strictly
% conceptual one. Rebecca had certain kinds of knowledge, but part of my
% in-interview intervention was to establish a frame where \emph{supposing
% syntax worked} was a valid part of programming. Once we mutually
% negotiated that frame there was a strong microcoherence of Rebecca
% stably explaining her code. But, crucially, my intervention wasn't about
% introducing conceptual content but rather establishing a kind of
% knowledge game---supposing---that one could play as part of programming
% design.
%
% \subsubsection{Students displayed a diversity of approaches to
% programming in the
% moment}\label{students-displayed-a-diversity-of-approaches-to-programming-in-the-moment}
%
% The diversity of programming approaches I saw students take is far
% greater than just the space spanned by Lionel and Rebecca. Within and
% across my data, I continually observed students saying and doing things
% that not only distinguished them from each other, but constituted
% phenomena I haven't seen described in prior CSEd research. I saw
% students employ clever tests for debugging, create their own makeshift
% debugger, and discuss at length whether certain programming constructs
% represented natural ways of thinking. For illustrative purposes, I'll
% discuss three such examples. While I don't analyze them at length, I
% include them here as proof of principle of the kinds of phenomena we're
% not currently capturing that nonetheless have a strong effect on
% students' programming approaches. Moreover, these ``in the wild''
% (Hutchins, 1995a) practices are further evidence of the kinds of
% knowledge that could be further refined into student expertise (Smith et
% al., 1993). So, not only do our research accounts miss these kinds
% phenomena, by missing them they preclude research and practice from
% building off them.
%
% \paragraph{Isaac used a thoughtful debugging strategy that code
% snapshots alone could never
% capture}\label{isaac-used-a-thoughtful-debugging-strategy-that-code-snapshots-alone-could-never-capture}
%
% When Isaac was debugging an error in his checkers game code, he made use
% of a clever test to discover he had reversed array index subscripts. He
% had become uncertain about which of the two subscripts corresponded to
% what he thought of as the x-coordinate on the board and which controlled
% the y-coordinate. So, he took a piece whose x-y position he knew and
% instructed his program to print it as a ``7'' on the board. Then, he
% told it to increment the first index of that known piece by 1, leave the
% second index alone and print the resulting piece on the board as ``8''.
% Finally, he told his program to increment the second index of the known
% piece by 1, leave the first piece alone, and print the resulting piece
% on the board as ``9''. By inspecting the visual output, Isaac could tell
% which subscript controlled which position because he knew where 8 and 9
% were relative to the unchanged 7. When he reflected on his work in the
% interview, he explained that what was tricky was realizing that the
% coordinate system for the checkers board didn't work the way standard
% coordinate systems in math worked; in the checkers program the indices
% were reversed.
%
% Because Isaac had been set up as a code-snapshot participant, we have a
% snapshot record of his work for that portion of the interview. But, the
% only snapshot-visible changes are the reversal of the indices and the
% accompanying test cases. The narrative surrounding Isaac's particular
% debugging text, including his reflection that coordinate systems in
% programming are ``trickier'' than those in math, is entirely invisible
% to the snapshot record. If I hadn't interviewed him I never would have
% captured it.
%
% \paragraph{Dana created her own debugging
% environment}\label{dana-created-her-own-debugging-environment}
%
% Dana was working on her checkers program and needed to debug some errant
% behavior. In class and in discussion students had been exposed to the
% GNU Debugger (GDB), so in the interview I asked Dana if she'd like to
% use it. She said she didn't feel comfortable using GDB. Instead, using
% my MacBook, Dana decided to create three separate instances of a
% terminal. In one instance, she had her checkers program source code
% open. In the second, she had the source code for a set of test inputs
% she was developing to try to pinpoint the error. The third window was
% dedicated to compiling and running the code. Since the program was
% designed to print an ASCII (text) representation of the board at each
% turn, the dedicated program window was Dana's visual output for which
% pieces were where at any state in the game.
%
% The important finding from this episode is how Dana deliberately
% structured her environment to support her activity. At the lowest level,
% she could have tried to do everything in a single terminal window. But,
% if she did that she'd have to constantly switch contexts. Because
% students in the course used a terminal version of emacs, she had no easy
% way to simultaneously view her code while compiling it. GDB would have
% been in many respects the right tool for that problem: it would let her
% set breakpoints, step through iterations of code, and inspect variable
% values as she did so. But Dana chose not to use GDB.
%
% Instead, Dana exploited the fact that Terminal.app on Mac OS X 10.7 can
% have multiple instances of itself open at the same time on the same
% screen. With three terminals running, she could persistently have her
% source code visible and executing at the same time. With one glance
% pattern she could move from the board's visual output to the test to the
% program source and back again. Collectively, the field of displays
% enabled a kind of programming that was substantially different---in
% terms of its representational affordances---from using GDB or a single
% terminal instance. Moreover, a snapshot-history-only research approach
% would have never had access to how or why Dana (re)configured her
% development environment.
%
% \paragraph{Toby said recursion was the ``hardest programming way to
% think''}\label{toby-said-recursion-was-the-hardest-programming-way-to-think}
%
% When I gave Toby a code sample task in an interview, he immediately
% commented on the fact that it used recursion. Recursion was, as he said,
% ``the hardest programming way to think.'' At one point his exact
% phrasing, quoting the film \emph{Ice Age}, was that ``recursion is bad
% juju.'' The task in question was adapted from an example in Abelson and
% Sussman (1996) that approximates the square root of a number using an
% elegant recursive approach.
%
% Toby's first explanation for why recursion was bad was that it's a
% completely unnatural way to think. He cited examples from cooking by
% saying, in effect, ``Nobody ever stirs by saying stir once, and if it's
% not stirred, stir once. They say stir 100 times.'' He cited simple
% operations---including incrementing a number---where using recursion to
% push and pop a stack seemed needlessly complex. One of his most damning
% indictments of recursion was that his class introduced it as a solution
% to generate numbers in the Fibbonacci sequence. Of course recursion
% works for examples that are mathematically recursive, he reasoned, but
% outside of those obviously contrived examples it was bad joo joo when
% compared to understandable, sensible iteration.
%
% The fuller account of Toby's resistance to recursion---and slight change
% of heart after an instructional moment---is beyond the scope of what I
% can present here. Suffice it to say that Toby's feelings toward
% recursion---rooted in a sense of it being an unnatural way to
% think---strongly directed his approach to programming. The approach was
% so strong, in fact, that during our first interview he wasn't fully able
% to produce working code examples of why recursion was absurd despite his
% vehement conviction that it was. In other words, he maintained that
% recursion was bad and inefficient even though every time he tried to
% demonstrate its inferiority he made mistakes in his code and his
% examples didn't work.
%
% \subsubsection{Dynamic epistemological models can offer a lens for
% reforming assessment and
% instruction.}\label{dynamic-epistemological-models-can-offer-a-lens-for-reforming-assessment-and-instruction.}
%
% Historically, a recognition of in-pieces cognitive dynamics in science
% education (diSessa, 1993; Hammer, 1994) has led to more direct research
% on how those frameworks inform instruction (Hammer \& Elby, 2003;
% Hammer, 1996; Louca et al., 2004). What I describe here is research
% still at the formative end of generating models to explain cognition.
% And, because the focus of the research was explicitly on learning and
% not instruction, I advise prudence in trying to draw certain kinds of
% specific recommendations from the work I present.
%
% Those restrictions being said, we can think carefully about what this
% work means for instruction. For practical purposes, let's refine
% terminology so we can talk more specifically and precisely about
% components of teaching. Below are my working definitions for teaching
% components:
%
% \begin{itemize}
% \item
%   How should instructors change the way they act in the classroom
%   (\emph{instruction})?
% \item
%   How should this research inform the set of intended learning outcomes
%   instructors prepare (\emph{curriculum})?
% \item
%   How should we change the way we measure or otherwise go about
%   ascertaining what students know (\emph{assessment})?
% \end{itemize}
%
% The clearest implications of my work are in instruction and assessment.
%
% From an instruction perspective, it may help teachers to know that
% students can enact different epistemological stances as part of their
% programming practice. As stated, that finding is in principle not new
% (cf., Gaspar \& Langevin, 2007). What is new, I think, is the
% characterization that these stances evidence epistemological resources
% that can be tapped for learning (cf., Hammer \& Elby, 2003). That
% sense-making about program design choices or computational concepts is a
% worthwhile activity, for example, is a message instructors can send to
% students that resonates with larger findings in engineering education
% research (Danielak et al., in press). Moreover, instructors can be more
% sensitive to what kinds of knowledge-activity students think they're
% doing, particularly when making epistemological moves like
% \emph{supposing} might help students work through design problems.
%
% My findings also speak to assessment. In the course I studied, there
% were few (if any) assessment instruments that would have revealed the
% kinds of knowledge I document Lionel and Rebecca having in this study.
% At no point in my observational data of the course were students asked
% to explain a design choice or reason about competing solutions to handle
% a problem. At no point were they encouraged to deeply sense-make about
% the conceptual content of the course, particularly the traditionally
% tricky topics of pointers and pointer arithmetic. At no point in my data
% did the instructor model the kind of sense-making Rebecca did in her
% interviews, a fact which she noted at one point as distinguishing the
% course from Basic Programming, its first-semester counterpart. Because
% the course never explicitly asked students to sense-make, reason about
% design choices, or explain designs, it never used information about what
% knowledge students had about those things to inform its instruction. It
% couldn't. Moreover, because the course never explicitly asked students
% to sense-make, reason about design choices, or explain designs, I was
% left to conclude such things were not prioritized learning outcomes of
% the course. To be clear, all courses establish a kind of focus by
% deciding what won't be covered, and I don't fault the instructor at all
% for running a course where design knowledge wasn't an enacted learning
% outcome. But, I'm also left to wonder: why wasn't it? Given that even
% beginning students can think about software design, and given that
% engineering as a discipline fundamentally involves design, shouldn't it
% have been?
%
% \section{Conclusion}\label{conclusion-2}
%
% If there is one overarching finding from this dissertation, it's that
% students clearly have resources for thinking about designing programs
% and a diversity of approaches to programming in the moment. Below, I
% explain the kind of diversity I saw in programming approaches. Then, I
% conclude with a discussion of what my research might mean for
% assessment.
%
% \subsection{We should think carefully about what students' programming
% design knowledge means for
% assessment}\label{we-should-think-carefully-about-what-students-programming-design-knowledge-means-for-assessment}
%
% My research helps us document and model the kinds of knowledge students
% have. In so doing, it points out kinds of information that the course's
% assessments were apt to miss:
%
% \begin{enumerate}
% \def\labelenumi{\arabic{enumi}.}
% \item
%   How students frame or otherwise approach the task of programming in
%   the moment
% \item
%   The role of different kinds of prior experience in stabilizing (or
%   potentially destabilizing) certain kinds of frames
% \item
%   In-the-moment practices---including talking out solutions, sketching
%   out debugging strategies, and writing out pseudo-code---that display
%   students' competence and sense-making
% \item
%   The code history that traces how designs evolve, including how
%   students start projects and what parts of their designs become
%   dead-ends
% \end{enumerate}
%
% Together, those four points are relevant for formative assessment (Black
% \& Wiliam, 1998) in introductory programming. (1) and (2) point us
% toward what students think they're doing when they're programming.
% Analogous work from science education tells us that students' sense of
% the kind of knowledge activity they're enacting matters for learning and
% assessment (Russ et al., 2008). For example, knowing early on that
% students are blindly copying code or randomly trying syntax gives
% instructors a chance to intervene. But, my research suggests
% intervention can't just be about stopping a bad behavior: knowing that
% copy-paste behavior is happening is different from knowing \emph{why}
% it's happening.
%
% Formative assessment has to be about diagnosing causes, not just
% identifying symptoms. Otherwise, we run the risk of ignoring or even
% harming students' productive knowledge. Take the example of copy-paste
% behavior, as documented by Gaspar and Langevin (Gaspar \& Langevin,
% 2007). One reason we may see copy-paste behavior, as Study 1
% demonstrates, is that students might see situations as new instances of
% already-solved problems. In professional practice, seeing old solved
% problems in new situations can be productive. Such insights can, for
% example, direct engineers to use a pre-built library of functions
% instead of building their own. But, another related reason for
% copy-paste may be efficiency. For Rebecca, copying and pasting code was
% very fast; it required only a few quick keystrokes. In the short run,
% Rebecca's choice to copy the code was a faster, less demanding, more
% trustworthy route to go on than was abstracting the code to a function.
% When we consider students trying to see common problems in new scenarios
% and solve such problems efficiently, ``copy-paste'' becomes a symptom
% rather than a root cause. Epistemological frameworks, then, can inform
% assessments by offering explanations that aim at the root cause of
% certain behaviors.
%
% Points 3 and 4 complement knowledge analysis by drawing focus to
% artifacts, practices, and history. As data from Lionel shows (Study 2),
% a crucial part of his design process involves artifacts and activities
% that were only distally knowable to the assessments he got in class.
% Lionel's instructor had no direct access to:
%
% \begin{itemize}
% \item
%   Lionel's whiteboard
% \item
%   the hours Lionel may have spent working out designs in chalk
% \item
%   how Lionel might have talked out design features
% \item
%   how Lionel's initial ``pseudo-code'' evolved into his final design.
% \end{itemize}
%
% In Rebecca's case, commented-out code in her final project submission
% were perhaps the only clues at all that she tried abstracting some
% flight day-checking procedures into functions (Study 1). Without
% Rebecca's history, the instructor would have had no reliable way of
% knowing:
%
% \begin{itemize}
% \item
%   Rebecca began her flights database project by copying code from an
%   earlier project
% \item
%   Rebecca's original solution for day-checking evolved from a seven-fold
%   conditional structure, one for each day
% \item
%   Rebecca may have tried abstracting day-checking procedures, even
%   creating a function called check\_days.
% \end{itemize}
%
% To sum up: traditional assessments can tell us what students finally
% produce but not how they produce it, when they produce it, why they
% produce it, or what the production process was. In the summative
% assessment structure of the course, final student products were the only
% products submitted to the instructor. My research shows that what's
% happening in the interstices---before typing, between compiles, away
% from the computer---can be captured and, in principle, analyzed and
% acted upon by instructors.
%
% \subsection{What We Might Change About Classroom
% Practice}\label{what-we-might-change-about-classroom-practice}
%
% What follows is my speculation about how we might specifically change
% classroom practices in light of my research findings. I move from
% recommendations I think are most strongly supported by my data to
% recommendations that align with my findings but are more expansive, and
% thus less strongly supported. As my recommendations broaden, I try to
% offer not only an instructional recommendation but a concomitant
% question for research.
%
% \subsubsection{Instructors could look beyond content to understand
% student
% difficulties}\label{instructors-could-look-beyond-content-to-understand-student-difficulties}
%
% I think first and foremost, instructors have to have a willingness to
% recognize that the difficulties they think they're seeing in students
% are difficulties they may be viewing through a lens of content. For
% example, after seeing a student struggle an instructor might say ``the
% student doesn't know assignment statements.'' But, those difficulties
% almost certainly have a deeper explanation. I say ``almost certainly,''
% where what I mean is that there are a number of patterns we've
% identified as common novice errors, but most research stops before
% asking why that's a common novice error. With rare exceptions (Fleury,
% 1991, 2000), the computing education community doesn't encourage asking
% the question \emph{why might a student be doing this?} and \emph{what
% might this tell me about the way a student is making sense of this?} By
% contrast, my research suggests those orientations are ripe, low-hanging
% fruit for instructors to take on. I think one of the first things as
% instructor could ask is, \emph{why would a student be thinking that this
% is the appropriate thing to be doing?} And that holds true whether the
% thing in question is writing the statement this way, or interpreting the
% code this way, or enacting this kind of programming activity this way.
%
% Moreover, we know from research on metacognition (Schoenfeld, 1987,
% 1992) that thinking beyond content opens the palette of kinds of
% interventions an instructor can make. Schoenfeld (1987), for example,
% came to encourage metacognition in his classroom by doggedly asking
% student groups what they were doing, why they were doing it, and how
% they hoped it would lead them to a solution. Students ultimately came to
% internalize such strategies and spent far less time floundering along
% solution paths that weren't ultimately productive. Might the same idea
% be true in introductory programming courses? To find out, we could begin
% researching in earnest how often and in what ways instructors model
% metacognition for their students. Currently, I would argue, we don't
% know whether and how instructors do so.
%
% \subsubsection{Instructors could use code history to inform
% interventions}\label{instructors-could-use-code-history-to-inform-interventions}
%
% When, for example, a student comes to office hours with a problem on a
% project, an instructor needs to quickly come to grips with the state of
% the students' project, the logic of their design, where they're stuck,
% why they're stuck, and what might best help them. That's no small task.
% But, given my analysis I have strong reason to believe that having code
% history available to instructor could change both the nature of coming
% to terms with a student's project and the conversation with a student
% that results. If an instructor can see the evolution of a student's
% code, at the very least they could see where the student started, how
% the code was growing, and where the student was working most recently.
% Even if the instructor had never before seen the code (or its history)
% until that moment in office hours, having both available changes the
% kind of view the instructor can get of the code and the specifics of the
% intervention that might result. In Rebecca's case, for example, it might
% have been an opportunity to explore why her check\_days abstracted
% function was failing on her flights database project.
%
% Having code-snapshot capabilities also changes the kind of research we
% can do. First and foremost, a result of this dissertation was to create
% a freely-available, lightweight, open-source framework for capturing and
% visualizing students' code histories.\footnote{https://github.com/briandk/gitvisualizations}
% So, the most basic kind of study would involve deploying that framework
% from the instructional side of a course and exploring what happens, as,
% for example, Hurd (2013) has done. One could ask questions of how
% instructors build code snapshotting into their course, how such
% information could or did inform assessment, and how such information
% could or did change the nature of instructional interactions with
% students. Was it for the better? If so, how do we know? If not, how do
% we know?
%
% \subsubsection{Instructors could establish a norm of asking why a design
% choice makes
% sense}\label{instructors-could-establish-a-norm-of-asking-why-a-design-choice-makes-sense}
%
% The most far-reaching implication of the research I present here is that
% instructors should establish a norm in their classes where anyone, at
% any point, for any piece of code, is allowed to ask \emph{why does this
% make sense as a design choice? Why is that an obvious choice to make?
% How does that work?} Consider a parallel example from mathematics
% education. In Don Saari's calculus class at UC Irvine, he ``invokes the
% principle of what he calls ``WGAD''---``Who gives a damn?'' (Bain, 2004,
% pp. 38--39). Bain explains:
%
% At the beginning of his courses, he tells his students that they are
% free to ask him the question on any day during the course, at any moment
% in class. He will stop and explain to his students why the material
% under consideration at that moment---however abstruse and minuscule a
% piece of the big picture it may be---is important, and how it relates to
% the larger questions and issues of the course. (Bain, 2004, p. 39)
%
% What I'm suggesting is even broader than that, because it's not just a
% question students can ask of professors; it's a question \emph{anyone}
% can ask of \emph{anyone.} I see it as the programming and design
% extension of a sociomathematical norm (Yackel \& Cobb, 1996), and I
% think it could lead to collaborative sense-making. That is, I'm asking
% for an accepted cultural practice that is also itself a design practice,
% whereby an instructor can help create a safe, stable space for a
% community to be reflective and critical about design choices. I also
% think it's an idea that leads to others, such as letting, if not
% requiring, students review one another's code. When students are forced
% to reckon with someone else's code and understand their design
% decisions, they're also forced to justify their own decisions. And
% establishing ``why that design choice?'' or ``how does that make
% sense?'' as a norm provides reciprocal opportunities for students, not
% just instructors, to improve the code of others.
%
% The research questions that come out of such an idea would include:
%
% \begin{itemize}
% \item
%   How can an instructor satisfactorily establish norms about design in a
%   classroom?
% \item
%   What does it look like when students collaboratively sense-make about
%   a program's design choices?
% \item
%   What does it look like when students are asked to reflect on their own
%   design choices?
% \item
%   How might introducing a code review component into a course change the
%   way students approach design? How might it improve students'
%   conceptual understanding? How might it improve the quality of the code
%   students produce?
% \end{itemize}
%
% \subsection{Final Remarks}\label{final-remarks}
%
% How students design programs matters for learning and instruction in
% engineering. It matters because finished code reflects what students
% know about design, whether or not instructors capture such information.
% It matters because students have resources for learning about and
% engaging in design; whether or not curriculum, instruction, and
% assessment choose to tap into those resources. It matters because design
% should be an intellectual thread that runs through all engineering
% courses. That thread shouldn't stop when we introduce students to
% programming.
%
% \section{Appendix 1 -- Transcript
% conventions}\label{appendix-1-transcript-conventions}
%
% \begin{itemize}
% \item
%   Turns are not numbered, but they are blank-line-delimited
% \item
%   Short interjected speech that does not interrupt a speaker's turn is
%   bounded /by slashes/
% \item
%   Matching double brackets show the {[}{[}onset and termination{]}{]} of
%   overlapping talk across turns.
% \item
%   *emphasized speech* is bounded by asterisks
% \item
%   Parenthetical clarifications by the analyst appear in parentheses. (He
%   considered but rejected square brackets.)
% \item
%   matching double equals signs mark turn boundaries== ==with minimal or
%   no audible silence (also known as latching turns)
% \item
%   Where gestures don't overlap speech they are in-lined by curly braces
%   when they happen \{folds arms, having made his point\}. All gestures
%   enacted \emph{by} a speaker appear within that speaker's turn unless
%   otherwise noted \{smiles, self-satisfied at having made this important
%   clarification\} \{audience scoffs\}.
% \item
%   When gestures happen \emph{during} speech, the speech is presented
%   first and bounded by double pipes. Gestures that happen simultaneously
%   with such speech are bounded by curly braces, nested within double
%   pipes, and immediately follow the speech they overlap. For example:
% \end{itemize}
%
% Throwing chainsaws \textbar{}\textbar{}*up*\textbar{}\textbar{}
% \textbar{}\{throws chainsaw\}\textbar{} is easy. Just be careful when
% they come \textbar{}\textbar{}*down*\textbar{}\textbar{}
% \textbar{}\{catches chainsaw for a punctuated finish\}\textbar{}.
%
% \section{Appendix 2 -- Visual conventions for
% gestures}\label{appendix-2-visual-conventions-for-gestures}
%
% A challenge of this dissertation was trying to use still-frames to
% convey motion and animation. Where I could, I annotated pictures with
% arrows to show trajectories of movement:
%
% \includegraphics[width=5.62222in,height=0.87778in]{media/image38.emf}
%
% \begin{supertabular}[]{@{}ll@{}}
% \toprule
% \textbf{Description} & \textbf{Example}\tabularnewline
% \midrule
% \endhead
% \textbf{Blue} arrows project forward to show action that will occur but
% hasn't yet in the frame you're looking at. The origin is where a hand
% \emph{is}; the terminus is where it \emph{will be} shortly. &
% \includegraphics[width=1.67778in,height=1.58889in]{media/image39.png}\tabularnewline
% \textbf{Pink} arrows show action that already occurred before the frame
% you're looking at. The origin is where a hand \emph{was}; the terminus
% is where it \emph{is}. &
% \includegraphics[width=1.65556in,height=2.37778in]{media/image20.png}\tabularnewline
% \bottomrule
% \end{supertabular}
%
% \section{\texorpdfstring{Appendix 3 -- Transcript of Rebecca's
% pseudo-code episode \emph{without} gesture
% codes}{Appendix 3 -- Transcript of Rebecca's pseudo-code episode without gesture codes}}\label{appendix-3-transcript-of-rebeccas-pseudo-code-episode-without-gesture-codes}
%
% Interviewer: OK. So then, um, what if we just pretended for a minute,
% that that, like==
%
% Rebecca: ==That that works=
%
% Interviewer: ==That that worked==
%
% Rebecca: ==OK==
%
% Interviewer: ==OK. So then, um. So then you might write like to make an
% array of \{stops writing\} actually what would we call this? This is==
%
% Rebecca: ==array of pointers, I guess==
%
% Interviewer: ==OK. Pointers. It'd be, well. So. The---the one we had was
% like int, star p, um, it would be some number, um, and I guess that'd be
% it in order to just declare /Yes/ that right? OK. And then you're
% saying, how would you like access an element of it?
%
% Rebecca: Uh, well, what I was think---like if I wanted to like, save it
% or whatever, /Yeah/ I could make a while loop, scan in the---scan in the
% data from the album, /OK/ uh, which is, I can write it {[}if you want
%
% Interviewer: Sure, go ahead{]}{]}
%
% Rebecca: Um, so, it was==
%
% Interviewer: {[}{[}Oh, sorry, just have to{]}
%
% Rebecca: {[}Oh{]}
%
% Interviewer: It's /Oh, OK/ one of those annoying ticks, but it works
% best cause of the camera if, ah, /Ah, OK/ if that's pointing away from
% your hand
%
% Rebecca: {[}OK{]}
%
% Interviewer: {[}That's fine{]}
%
% Rebecca: Um, there was album, and then, so there was, uh, percent d, and
% percent s, and then give those names, just, I'll just call it number,
% and title. /Mmmhmm/ And so, at least, my---until it does not equal EOF,
% and then my thinking at least, is you should be able to, um, say that
%
% Rebecca: ``star p of i'' /mmhmm/ equals, uh, the title, and then you
% just do i++, so then it'll move to the next one /OK/ and you just keep
% saving each of the pointers in the array to a title /Mmmhmm/ And you
% just increment by 1 until you reach the end of file.
%
% Interviewer: OK==
%
% Rebecca: ==Like *that* would make sense to me.
%
% Interviewer: So then, like, what would go in p of one would be the first
% title we read in==
%
% Rebecca: ==Yes,
%
% Interviewer: uh, would, when the while loop runs again, does it get a
% fresh line {[}{[}line from that
%
% Rebecca: Yeah, because{]}{]} uh, what the while loop does is it reads in
% the line and then once it reaches the end of character it'll go back
% down---it'll do the while loop and then go back down to the next
% thing---line, it'll read in the line until it reaches the end of file
%
% Interviewer: So as you step through this loop /mmhmm/ i keeps going up
% by one==
%
% Rebecca: ==Yes.
%
% Interviewer: Uh
%
% Rebecca: And the lines keep going down so, that way, the first line is
% going to be, uh the p---element zero /OK/ uh the second one'll be
% element one (Interview 4 of 5, April 6, 2012)
%
% \section{Appendix 4 -- Conceptual knowledge frameworks in computing
% don't tell us much about how Lionel structured an in-interview
% program}\label{appendix-4-conceptual-knowledge-frameworks-in-computing-dont-tell-us-much-about-how-lionel-structured-an-in-interview-program}
%
% In this section, I explore Lionel's work in solving an in-interview
% programming task. First, I reproduce the prompt Lionel was given. Next,
% I show the final source code of his solution to the problem, written in
% C. In the analysis that follows, I move outward to explore the context
% of that code's production.
%
% My choice to start with the code first is deliberate. I'm trying to
% mimic what the instructor of a typical programming course might see: the
% final submitted form of a student's code. This approach is admittedly a
% bit backwards, since the data I show after I present Lionel's submission
% is all about what happened \emph{before} the code reached its final
% form. But, I think this approach useful because it invites us to explore
% an artifact first, then raise questions about the conditions of its
% production. In that sense, my presentation has the anthropological feel
% of understanding a found object, which I think is a faithful way of
% looking at what many instructors (not to mention researchers, and of
% course active software developers) face in their day-to-day work.
%
% \subsection{\texorpdfstring{\protect\hypertarget{ux5fToc247188555}{}{\protect\hypertarget{ux5fToc252445950}{}{}}Analyzing
% Lionel's Solution From a Conceptual Knowledge
% View}{Analyzing Lionel's Solution From a Conceptual Knowledge View}}\label{analyzing-lionels-solution-from-a-conceptual-knowledge-view}
%
% To infer conceptual knowledge, I'll use Elliott Tew's (2010) conceptual
% categories for procedural programming in first-semester computer science
% (CS1) courses. My reasons for doing so are:
%
% \begin{enumerate}
% \def\labelenumi{\arabic{enumi}.}
% \item
%   Elliott Tew (2010) sought concepts that were language-agnostic. That
%   is, the concepts Elliott Tew (2010) identified are not idiomatic to
%   just one language. Rather, they are general ideas implemented in
%   nearly every major procedural language taught to undergraduates (Java,
%   Python, C-based languages, Scheme).
% \item
%   Elliott Tew's (2010, pp. 23--27) selection process was extensive. She
%   began with curricular documents, moved to canonical texts, then
%   ultimately conducted a thematic analysis on emerging categories to
%   produce her final list of concepts.
% \item
%   To date, Elliott Tew's (2010) identification of CS1 concepts is the
%   only work I know of that has been carried forward to create,
%   administer, and validate concept inventories in CS1 (Elliott Tew \&
%   Guzdial, 2010, 2011)
% \end{enumerate}
%
% Figure 2 below presents an overview of the conceptual features I
% identified in Lionel's code. Even in a short, relatively simple program
% Lionel exhibits four of the ten concept categories Elliott Tew (2010)
% identifies. Moreover, he uses each of them ``correctly,'' so to speak,
% in that not only does his program compile, it also properly finds the
% range of the numbers it's given. It's also worth noting that Lionel's
% code displays consistent, helpful use of whitespace (indentation) as
% well as informative comments. While these attributes aren't conceptual,
% per se, they are the sorts of features an instructor might consider in
% assigning a grade to this program.
%
% \includegraphics[width=5.60000in,height=5.67778in]{media/image40.jpeg}
%
% \protect\hypertarget{ux5fToc252445975}{}{}Figure ‑ -- Highlighting the
% conceptual features of Lionel's code given Elliott Tew's (2010) outline
% of CS 1 concepts
%
% \section{Appendix 5 -- Neverly-Asked Questions
% (NAQs)}\label{appendix-5-neverly-asked-questions-naqs}
%
% \subsubsection{Neverly Asked Questions (NAQs) about my conceptual
% analysis of Lionel's
% code}\label{neverly-asked-questions-naqs-about-my-conceptual-analysis-of-lionels-code}
%
% \emph{Didn't Elliott Tew (2010) define those concepts as a way of
% creating a concept inventory?}
%
% Yes. In her work, I believe each concept was represented by at least one
% multiple-choice question (MCQ) with carefully-chosen distractors. In
% each MCQ a student would see a snippet of pseudo-code and be asked to
% choose from a list of responses. The structure was explicitly modeled
% after Force Concept Inventory (FCI) work in physics education.
%
% \emph{So, if her intent (and research) was about making a concept
% \textbf{inventory}, aren't you misusing her ideas of concepts?}
%
% What makes you say that?
%
% \emph{Well, first off, she used categories of concept to create concrete
% instances of code, which she tests students on}\ldots{}
%
% That's my understanding, yep.
%
% \emph{But you used a concrete instance of code --- Lionel's code --- and
% inferred the existence of concepts in it. So, that's not the same thing
% as what Elliott Tew (2010) did.}
%
% I agree I'm not doing what she did, but I don't think that observation
% necessarily undermines the usefulness of what I \emph{am} doing. If the
% concepts she identifies really are general concepts, then it would seem
% silly to think we couldn't find concrete instances of code that
% exemplify them. I think I might rephrase your objection: ``how can I
% trust that you properly identified concepts in Lionel's code?'' My
% answer to that is, again, yes, I have no coding scheme from Elliott Tew
% to apply. But, the code Lionel wrote is there and my candidates for
% assigning concepts are there: my results are open to inspection and
% challenge.
%
% \emph{OK. Fine. But what ``conceptual knowledge analysis'' is there to
% speak of? You just made a slide and identified which parts of his code
% might correspond to which concepts.}
%
% What I did was assign concepts to canonically-correct --- that is,
% syntactically valid parts of Lionel's code that also work properly to
% make the program produce the right results --- sections of Lionel's
% code. I think your objection is about my inferential basis: ``why does
% the presence of those code chunks --- and my tagging certain chunks as
% aligning with Elliott Tew's (2010) concepts --- imply that Lionel has
% those concepts in his head?''
%
% \emph{I accept your amendment. So, what basis do you have for saying
% ``this concept is in Lionel's mind?''}
%
% I feel like I'm on the same playing field as the concept inventory
% folks, though in some ways I'm setting a modified standard for validity.
% Elliott Tew (2010) says this about validity:
%
% validity is the evidence that assures us that questions about a
% particular concept are indeed measuring that concept. For instance, a
% question about arrays should require a student to have knowledge about
% arrays, but should not require knowledge about another concept, such as
% recursion. In addition, it is important that the question cannot be
% answered correctly without knowledge of arrays. (Elliott Tew, 2010, p.
% 10)
%
% I don't know that I share her assumptions about validity. But, what
% specifically might be at issue here is ``does the presence of a for-loop
% in code imply the student has knowledge of definite loops?'' My answer
% is, ``if not that, then what?''
%
% \emph{But couldn't I argue that the for-loop could be there because he
% copied and pasted it from the internet? That he wouldn't have had to
% know anything about definite loops to do that?}
%
% Yes, you could. My rejoinder to that is in section 3.3.3.
%
% \subsubsection{Neverly-asked questions about Lionel's verbal pseudo-code
% description}\label{neverly-asked-questions-about-lionels-verbal-pseudo-code-description}
%
% Why do you so pedantically analyze such a short, small snippet of
% conversation?
%
% Because by Lionel's own assertion what was said constitutes his ``main
% concept'' (line 12) for the program. That is, by his own account he has
% just expressed the entire top-level design for the program. But, he's
% done it all all without writing a single additional line of code.
%
% So?
%
% If anything, this episode suggests that Lionel can and does have ways of
% expressing a process or procedure that don't rely on writing code. More
% importantly, I think, is that he views it as a legitimate activity to
% express ideas at a level \emph{above} the particularities of syntax and
% implementation.
%
% What do you mean ``views it as a legitimate activity?''
%
% I think here I'm appealing to my interpretation of Goffman's (1974)
% \emph{frames}: a participant's understanding of ``what is it'' that's
% going on in a social situation. Lionel takes up my bid for him to
% explain what he's thinking, but he does so first by explaining his
% ``general concept'' for the program.
%
% Right. But, I mean, he's just doing what you \textbf{asked} him to do. I
% don't understand how any of this relates to Goffman or frames. That just
% seems like a needlessly complicated way to explain he did what you
% asked.
%
% OK, let's step back for a second. Suppose you've just started going to
% group therapy. One of the things group therapy environments often stress
% is that you should try to talk about how \emph{you feel}, which can be a
% hard transition for people. Group therapy participants may be used to
% saying ``you always do {[}X{]}'' when in a fight with someone. And, if
% prompted by a group leader to discuss how they feel about troublesome
% instance, a participant may adopt that kind of language. ``My partner
% always yells at me if I'm out late,'' for example.
%
% Group therapy can encourage inward reflection in a way that reshapes how
% participants talk about themselves and their feelings. So, after some
% time in group therapy our hypothetical participant may change the way
% they orient toward questions. If asked ``how are you feeling about the
% fight you and your partner had?'' our hypothetical participant may now
% say ``I feel like I don't have the freedom to see the people I want to
% see.'' The focus in speech shifts from the \emph{other}---in this case a
% partner---to the \emph{self}. Moreover, the substance of the speech is
% about feelings, rather than behavior.
%
% I think these reshapings of speech are neither trivial nor incidental.
% If we had transcripts of our hypothetical person's early sessions in
% group therapy and compared them to those of later sessions, I think we
% would not be surprised to see a marked change. When asked the same
% question---in this case ``how are you feeling about {[}X{]},'' the
% structure and character of our participant's response changes after
% therapy.
%
% My point here is that an everyday sense of how participants' speech
% patterns change in therapy primes our intuition for thinking more
% broadly about social phenomena. A cynical view of our participant would
% still have to acknowledge that, observably, the structure of their
% response to the question ``how does that make you feel?'' changes after
% therapy. I think a sociolinguistic stance would be more generous: our
% participant is \emph{orienting} differently toward the question.
% Furthermore, I take the following as evidence of such an orientation:
%
% \begin{itemize}
% \item
%   A change in sentence focus from the other to the self
% \item
%   Introspection into the source of a feeling
% \item
%   Acknowledgment of personal responsibility for feeling those feelings
% \end{itemize}
%
% I think we as analysts can acknowledge that after therapy, our
% participant frames the event of being asked, ``how does that make you
% feel?'' differently.
%
% OK. So, you're saying that if we look at the change in how someone
% answers the question ``how does that make you feel?'' we can find
% evidence of the superstructure---a frame---that directs their sense of
% appropriate ways to answer that question?
%
% Yes.
%
% So how does this get back to your claims about Lionel?
%
% I think it's tempting to take this position that, ``it's
% inevitable/totally expected that he would just verbalize the top-level
% description of his procedure! I mean, what else could he possibly do?''
%
% I mean, yeah. Kind of.
%
% Right, but I think actually that assumption isn't in line with a classic
% frame analysis approach. Even if the assumption is right, frame analysis
% changes the explanation from ``it's inevitable'' to ``hmm, I guess
% almost everyone I've ever seen brings or enacts the same structure of
% expectations.''
%
% So, my argument is, if you ask someone to describe to you how they're
% thinking about their program, they're gonna do that. What else
% \textbf{would} they do?
%
% Right. And I don't think we disagree that they'll describe their
% thinking. I think where we disagree is that everyone will describe in
% the same way. I'm saying \emph{if} they describe their program to you,
% \emph{that} they're able to express it, what they choose to verbalize,
% what they choose to leave out, how their gestures accompany the
% explanations, and their ultimate criteria for having satisfied the task
% of describing for themselves are all driven underneath (or from above?)
% by structures of expectations. And even if \emph{one thousand} out of
% one thousand people do the same kind of thing Lionel does, that doesn't
% undermine the idea that people have and bring structures of expectations
% to a task. What I think it does is show that one thousand out of one
% thousand people share the same sets of expectations about what it is
% that's going on when they're asked to verbalize their thinking on a
% programming protocol.
%
% So, what's the point of this whole section on Lionel's verbal
% description, then?
%
% I think the point is that when I ask Lionel to verbalize his thinking,
% he outlines through talk what the ``main concept'' of his program is. I
% focus on \emph{that} he decides to describe it out in words, \emph{how}
% he describes it, \emph{what} he leaves out\emph{, how} he repairs, and
% how all of that reflects structures of expectations about the activity
% and what he's being asked to do. Ultimately, the argument is that
% Lionel's choice to take me up on that bid and how he does so
% reflects---to me, the analyst---that he views \emph{verbal description
% of a high-level program} as a legitimate activity and part of his
% approach to programming.
%
% \section{References}\label{references}
%
% Abelson, H., \& Sussman, G. J. (1996). \emph{Structure and
% Interpretation of Computer Programs} (2nd ed.). Cambridge, Mass: MIT
% Press.
%
% Adelson, B., \& Soloway, E. (1985). The Role of Domain Experience in
% Software Design. \emph{IEEE Transactions on Software Engineering},
% \emph{SE-11}(11), 1351 -- 1360. doi:10.1109/TSE.1985.231883
%
% Archer, L. B. (1979). Whatever became of design methodology.
% \emph{Design Studies}, \emph{1}(1), 17--18.
%
% Bain, K. (2004). \emph{What the best college teachers do}. Cambridge,
% Mass: Harvard University Press.
%
% Baker, A., \& van der Hoek, A. (2010). Ideas, subjects, and cycles as
% lenses for understanding the software design process. \emph{Design
% Studies}, \emph{31}(6), 590--613. doi:10.1016/j.destud.2010.09.008
%
% Ball, L. J., Onarheim, B., \& Christensen, B. T. (2010). Design
% requirements, epistemic uncertainty and~solution development strategies
% in software design. \emph{Design Studies}, \emph{31}(6), 567--589.
% doi:10.1016/j.destud.2010.09.003
%
% Bayman, P., \& Mayer, R. E. (1983). A diagnosis of beginning
% programmers' misconceptions of BASIC programming statements.
% \emph{Communications of the ACM}, \emph{26}(9), 677--679.
% doi:http://doi.acm.org/10.1145/358172.358408
%
% Black, P., \& Wiliam, D. (1998). Inside the Black Box: Raising Standards
% Through Classroom Assessment. \emph{Phi Delta Kappan}, \emph{80}(2),
% 139--44.
%
% Boaler, J. (1998). Open and closed mathematics: Student experiences and
% understandings. \emph{Journal for Research in Mathematics Education},
% \emph{29}(1), 41--62.
%
% Boaler, J. (2000). Mathematics from another world: Traditional
% communities and the alienation of learners. \emph{The Journal of
% Mathematical Behavior}, \emph{18}(4), 379--397.
% doi:10.1016/S0732-3123(00)00026-2
%
% Boaler, J. (2002). The development of disciplinary relationships:
% Knowledge, practice and identity in mathematics classrooms. \emph{For
% the Learning of Mathematics}, \emph{22}(1), 42--47.
%
% Boaler, J., \& Greeno, J. G. (2000). Identity, agency, and knowing in
% mathematical worlds. In J. Boaler (Ed.), \emph{Multiple perspectives on
% mathematics teaching and learning} (pp. 171--200). Westport, CT: Ablex
% Pub.
%
% Bonar, J., \& Soloway, E. (1983). Uncovering principles of novice
% programming. \emph{Proceedings of the 10th ACM SIGACT-SIGPLAN Symposium
% on Principles of Programming Languages}, 10--13.
% doi:10.1145/567067.567069
%
% Bonar, J., \& Soloway, E. (1985). Preprogramming Knowledge: A Major
% Source of Misconceptions in Novice Programmers. \emph{Human-Computer
% Interaction}, \emph{1}(2), 133.
%
% Bucciarelli, L. L. (1994). \emph{Designing engineers}. Cambridge, Mass:
% MIT Press.
%
% Clancy, M. (2004). Misconceptions and Attitudes that Interfere with
% Learning to Program. In S. Fincher \& M. Petre (Eds.), \emph{Computer
% Science Education Research} (pp. 85--100). London, UK: RoutledgeFalmer.
%
% Danielak, B. A., Gupta, A., \& Elby, A. (in press). The Marginalized
% Identities of Sense-Makers: Reframing Engineering Student Retention.
% \emph{Journal of Engineering Education}.
%
% Danielsiek, H., Paul, W., \& Vahrenhold, J. (2012). Detecting and
% Understanding Students' Misconceptions Related to Algorithms and Data
% Structures. In \emph{Proceedings of the 43rd ACM Technical Symposium on
% Computer Science Education} (pp. 21--26). New York, NY, USA: ACM.
% doi:10.1145/2157136.2157148
%
% diSessa, A. A. (1986). Models of Computation. In D. A. Norman \& S. W.
% Draper (Eds.), \emph{User centered system design: new perspectives on
% human-computer interaction} (pp. 201--218). Hillsdale, N.J: L. Erlbaum
% Associates.
%
% diSessa, A. A. (1993). Toward an Epistemology of Physics.
% \emph{Cognition and Instruction}, \emph{10}(2/3), 105--225.
%
% diSessa, A. A. (2002). Why ``Conceptual Ecology'' is a good idea. In M.
% Limón \& L. Mason (Eds.), \emph{Reconsidering conceptual change: issues
% in theory and practice} (pp. 29--60). Dordrecht ; Boston: Kluwer
% Academic Publishers.
%
% diSessa, A. A., \& Sherin, B. L. (1998). What changes in conceptual
% change? \emph{International Journal of Science Education},
% \emph{20}(10), 1155--1191. doi:10.1080/0950069980201002
%
% Duckworth, E. R. (2006). \emph{``The having of wonderful ideas'' and
% other essays on teaching and learning} (3rd ed.). New York: Teachers
% College Press.
%
% Elby, A., \& Hammer, D. (2010). Epistemological resources and framing: A
% cognitive framework for helping teachers interpret and respond to their
% students' epistemologies. In L. D. Bendixen \& F. C. Feucht (Eds.),
% \emph{Personal epistemology in the classroom: theory, research, and
% implications for practice} (pp. 409--434). Cambridge, UK ; New York:
% Cambridge University Press.
%
% Elliott Tew, A. (2010). \emph{Assessing fundamental introductory
% computing concept knowledge in a language independent manner} (Ph.D.).
% Georgia Institute of Technology, Ann Arbor. Retrieved from ProQuest
% Dissertations \& Theses Full Text. (873212789)
%
% Elliott Tew, A., \& Guzdial, M. (2010). Developing a validated
% assessment of fundamental CS1 concepts. \emph{Proceedings of the 41st
% ACM Technical Symposium on Computer Science Education}, 97--101.
% doi:10.1145/1734263.1734297
%
% Elliott Tew, A., \& Guzdial, M. (2011). The FCS1: a language independent
% assessment of CS1 knowledge. In \emph{Proceedings of the 42nd ACM
% technical symposium on Computer science education} (pp. 111--116). New
% York, NY, USA: ACM. doi:10.1145/1953163.1953200
%
% Erickson, F. (1986). Qualitative methods in research on teaching. In M.
% C. Wittrock (Ed.), \emph{Handbook of research on teaching} (3rd ed., pp.
% 119--161). New York : London: Macmillan ; Collier Macmillan.
%
% Eynde, P., \& Hannula, M. (2006). The Case Study of Frank.
% \emph{Educational Studies in Mathematics}, \emph{63}(2), 123--129.
% doi:10.1007/s10649-006-9030-8
%
% Fleury, A. E. (1991). Parameter passing: the rules the students
% construct. In \emph{Proceedings of the twenty-second SIGCSE technical
% symposium on Computer science education} (pp. 283--286). New York, NY,
% USA: ACM. doi:10.1145/107004.107066
%
% Fleury, A. E. (1993). Student Beliefs about Pascal Programming.
% \emph{Journal of Educational Computing Research}, \emph{9}(3), 355--371.
% doi:10.2190/VECR-P8T6-GB10-MXJ5
%
% Fleury, A. E. (2000). Programming in Java: student-constructed rules.
% \emph{SIGCSE Bull.}, \emph{32}(1), 197--201. doi:10.1145/331795.331854
%
% Gainsburg, J. (2006). The mathematical modeling of structural engineers.
% \emph{Mathematical Thinking \& Learning}, \emph{8}(1), 3--36.
% doi:10.1207/s15327833mtl0801\_2
%
% Gal-Ezer, J., \& Zur, E. (2004). The efficiency of
% algorithms-\/-misconceptions. \emph{Computers \& Education},
% \emph{42}(3), 215--226.
%
% Gaspar, A., \& Langevin, S. (2007). Restoring ``coding with intention''
% in introductory programming courses. \emph{Proceedings of the 8th ACM
% SIGITE Conference on Information Technology Education}, 91--98.
% doi:10.1145/1324302.1324323
%
% Ginsburg, H. P. (1997). \emph{Entering the child's mind: the clinical
% interview in psychological research and practice}. Cambridge ; New York:
% Cambridge University Press.
%
% Ginsburg, H. P., \& Opper, S. (1988). \emph{Piaget's Theory of
% Intellectual Development} (3rd ed.). Englewood Cliffs, N.J:
% Prentice-Hall. Retrieved from http://lccn.loc.gov/87017353
%
% Goffman, E. (1974). \emph{Frame Analysis: An Essay on the Organization
% of Experience}. New York: Harper \& Row.
%
% Goldin-Meadow, S. (2003). \emph{Hearing gesture: how our hands help us
% think}. Cambridge, Mass: Belknap Press of Harvard University Press.
%
% Goodwin, C. (2000). Action and embodiment within situated human
% interaction. \emph{Journal of Pragmatics}, \emph{32}(10), 1489--1522.
% doi:10.1016/S0378-2166(99)00096-X
%
% Gupta, A., Hammer, D., \& Redish, E. F. (2010). The Case for Dynamic
% Models of Learners' Ontologies in Physics. \emph{Journal of the Learning
% Sciences}, \emph{19}(3), 285. doi:10.1080/10508406.2010.491751
%
% Hall, R. (1999). Following mathematical practices in design-oriented
% work. In C. Hoyles, C. Morgan, \& G. Woodhouse (Eds.), \emph{Rethinking
% the Mathematics Curriculum} (pp. 29--47). London: Falmer Press.
%
% Hall, R., \& Nemirovsky, R. (2012). Introduction to the Special Issue:
% Modalities of Body Engagement in Mathematical Activity and Learning.
% \emph{Journal of the Learning Sciences}, \emph{21}(2), 207--215.
% doi:10.1080/10508406.2011.611447
%
% Hall, R., \& Stevens, R. (1995). Making space: A comparison of
% mathematical work in school and professional design practices. In S. L.
% Star (Ed.), \emph{The cultures of computing} (pp. 118--145). Oxford, UK:
% Blackwell Publisher.
%
% Hall, R., Stevens, R., \& Torralba, T. (2002). Disrupting
% representational infrastructure in conversations across disciplines.
% \emph{Mind, Culture \& Activity}, \emph{9}(3), 179--210.
%
% Hall, R., Wright, K., \& Wieckert, K. (2007). Interactive and Historical
% Processes of Distributing Statistical Concepts Through Work
% Organization. \emph{Mind, Culture \& Activity}, \emph{14}(1/2),
% 103--127. doi:10.1080/10749030701307770
%
% Hammer, D. (1989). Two approaches to learning physics. \emph{The Physics
% Teacher}, \emph{27}(9), 664--670. doi:10.1119/1.2342910
%
% Hammer, D. (1994). Epistemological beliefs in introductory physics.
% \emph{Cognition and Instruction}, \emph{12}(2), 151--183.
% doi:10.2307/3233679
%
% Hammer, D. (1996). Misconceptions or P-Prims: How May Alternative
% Perspectives of Cognitive Structure Influence Instructional Perceptions
% and Intentions? \emph{Journal of the Learning Sciences}, \emph{5}(2),
% 97--127. doi:10.1207/s15327809jls0502\_1
%
% Hammer, D., \& Elby, A. (2002). On the form of a personal epistemology.
% In B. K. Hofer \& P. R. Pintrich (Eds.), \emph{Personal epistemology:
% The psychology of beliefs about knowledge and knowing} (pp. 169--190).
% Mahwah, N.J: L. Erlbaum Associates.
%
% Hammer, D., \& Elby, A. (2003). Tapping epistemological resources for
% learning physics. \emph{The Journal of the Learning Sciences},
% \emph{12}(1), 53--90. doi:10.2307/1466634
%
% Hammer, D., Elby, A., Scherr, R. E., \& Redish, E. F. (2005). Resources,
% framing, and transfer. In J. P. Mestre (Ed.), \emph{Transfer of learning
% from a modern multidisciplinary perspective}. Greenwich, CT: IAP.
%
% Hannula, M., Evans, J., Philippou, G., \& Zan, R. (2004). Affect in
% Mathematics Education--Exploring Theoretical Frameworks. Research Forum.
% \emph{International Group for the Psychology of Mathematics Education},
% 30.
%
% Henderson, K. (1999). \emph{On line and on paper: visual
% representations, visual culture, and computer graphics in design
% engineering}. Cambridge, Mass: MIT Press.
%
% Herman, G. L., Kaczmarczyk, L., Loui, M. C., \& Zilles, C. (2008). Proof
% by incomplete enumeration and other logical misconceptions.
% \emph{Proceeding of the Fourth International Workshop on Computing
% Education Research}, 59--70. doi:10.1145/1404520.1404527
%
% Hofer, B. K., \& Pintrich, P. R. (1997). The development of
% epistemological theories: Beliefs about knowledge and knowing and their
% relation to learning. \emph{Review of Educational Research},
% \emph{67}(1), 88--140. doi:10.3102/00346543067001088
%
% Holland, S., Griffiths, R., \& Woodman, M. (1997). Avoiding object
% misconceptions. In \emph{Proceedings of the twenty-eighth SIGCSE
% technical symposium on Computer science education} (pp. 131--134). New
% York, NY, USA: ACM. doi:10.1145/268084.268132
%
% Hurd, A. (2013, March). \emph{Assessment in an Introduction to
% Programming Course}. Presented at the 16th Annual Course Technology
% Conference, San Diego, CA, USA. Retrieved from
% http://www.slideshare.net/CengageLearning/andrew-hurd-assessment-in-an-intro-to-programming-course
%
% Hutchins, E. (1995a). \emph{Cognition in the Wild}. Cambridge, Mass: MIT
% Press.
%
% Hutchins, E. (1995b). How a cockpit remembers its speeds.
% \emph{Cognitive Science}, \emph{19}(3), 265--288.
% doi:10.1016/0364-0213(95)90020-9
%
% Izsák, A. (2004). Students' Coordination of Knowledge When Learning to
% Model Physical Situations. \emph{Cognition \& Instruction},
% \emph{22}(1), 81--128.
%
% Jackson, M. (2010). Representing structure in a software system design.
% \emph{Design Studies}, \emph{31}(6), 545--566.
% doi:10.1016/j.destud.2010.09.002
%
% Jadud, M. C. (2006). Methods and tools for exploring novice compilation
% behaviour. In \emph{Proceedings of the 2006 international workshop on
% Computing education research - ICER '06} (p. 73). Canterbury, United
% Kingdom. doi:10.1145/1151588.1151600
%
% Jordan, B., \& Henderson, A. (1995). Interaction analysis: Foundations
% and practice. \emph{Journal of the Learning Sciences}, \emph{4}(1), 39.
% doi:10.1207/s15327809jls0401\_2
%
% Joy, M., Sinclair, J., Sun, S., Sitthiworachart, J., \& López-González,
% J. (2009). Categorising computer science education research.
% \emph{Education and Information Technologies}, \emph{14}(2), 105--126.
% doi:10.1007/s10639-008-9078-4
%
% Kaczmarczyk, L. C., Petrick, E. R., East, J. P., \& Herman, G. L.
% (2010). Identifying student misconceptions of programming.
% \emph{Proceedings of the 41st ACM Technical Symposium on Computer
% Science Education}, 107--111. doi:10.1145/1734263.1734299
%
% Kaiser, D. (2005). \emph{Drawing theories apart: the dispersion of
% Feynman diagrams in postwar physics}. Chicago: University of Chicago
% Press.
%
% Keller, E. F. (1983). \emph{A feeling for the organism: the life and
% work of Barbara McClintock}. San Francisco: W.H. Freeman.
%
% Kolikant, Y. B.-D., \& Mussai, M. (2008). ``So my program doesn't run!''
% Definition, origins, and practical expressions of students'
% (mis)conceptions of correctness. \emph{Computer Science Education},
% \emph{18}(2), 135.
%
% Kölling, M., \& Utting, I. (2012). Building an open, large-scale
% research data repository of initial programming student behaviour. In
% \emph{Proceedings of the 43rd ACM technical symposium on Computer
% Science Education} (pp. 323--324). New York, NY, USA: ACM.
% doi:10.1145/2157136.2157234
%
% Latour, B. (1987). \emph{Science in action: how to follow scientists and
% engineers through society}. Cambridge, Mass: Harvard University Press.
%
% Latour, B. (1990). Drawing things together. In M. Lynch \& S. Woolgar
% (Eds.), \emph{Representation in Scientific Practice} (1st MIT Press ed.,
% pp. 19--68). Cambridge, Mass: MIT Press.
%
% Lehrer, R., Schauble, L., Carpenter, S., \& Penner, D. (2000). The
% interrrelated development of inscriptions and conceptual understanding.
% In P. Cobb, E. Yackel, \& K. McClain (Eds.), \emph{Symbolizing and
% Communicating in Mathematics Classrooms: Perspectives on Discourse,
% Tools, and Instructional Design} (pp. 325--360). Mahwah, N.J: Lawrence
% Erlbaum Associates.
%
% Lising, L., \& Elby, A. (2005). The impact of epistemology on learning:
% A case study from introductory physics. \emph{American Journal of
% Physics}, \emph{73}(4), 372. doi:10.1119/1.1848115
%
% Louca, L., Elby, A., Hammer, D., \& Kagey, T. (2004). Epistemological
% Resources: Applying a New Epistemological Framework to Science
% Instruction. \emph{Educational Psychologist}, \emph{39}(1), 57--68.
%
% Malmi, L., Sheard, J., Simon, Bednarik, R., Helminen, J., Korhonen, A.,
% \ldots{} Taherkhani, A. (2010). Characterizing research in computing
% education: a preliminary analysis of the literature. In
% \emph{Proceedings of the Sixth international workshop on Computing
% education research} (pp. 3--12). New York, NY, USA: ACM.
% doi:10.1145/1839594.1839597
%
% Martin, R. C. (2009). \emph{Clean code: a handbook of agile software
% craftsmanship}. Upper Saddle River, NJ: Prentice Hall.
%
% Mayer, R. E. (1981). The Psychology of How Novices Learn Computer
% Programming. \emph{ACM Computing Surveys (CSUR)}, \emph{13}, 121--141.
% doi:10.1145/356835.356841
%
% Minsky, M. L. (1986). \emph{The Society of Mind}. New York: Simon and
% Schuster.
%
% Nasir, N. S., \& Cooks, J. (2009). Becoming a Hurdler: How Learning
% Settings Afford Identities. \emph{Anthropology \& Education Quarterly},
% \emph{40}(1), 41--61. doi:10.1111/j.1548-1492.2009.01027.x
%
% Nasir, N. S., \& Hand, V. (2008). From the court to the classroom:
% Opportunities for engagement, learning, and identity in basketball and
% classroom mathematics. \emph{Journal of the Learning Sciences},
% \emph{17}(2), 143--179. doi:10.1080/10508400801986108
%
% Nasir, N. S., \& Hand, V. M. (2006). Exploring Sociocultural
% Perspectives on Race, Culture, and Learning. \emph{Review of Educational
% Research}, \emph{76}(4), 449--475.
%
% Nemirovsky, R., Rasmussen, C., Sweeney, G., \& Wawro, M. (2012). When
% the Classroom Floor Becomes the Complex Plane: Addition and
% Multiplication as Ways of Bodily Navigation. \emph{Journal of the
% Learning Sciences}, \emph{21}(2), 287--323.
% doi:10.1080/10508406.2011.611445
%
% Ochs, E., Gonzales, P., \& Jacoby, S. (1996). ``When I come down I'm in
% the domain state'': grammar and graphic representation in the
% interpretive activity of physicists. In \emph{Interaction and Grammar}
% (pp. 328--369). Cambridge: Cambridge University Press.
%
% Papert, S. (1980). \emph{Mindstorms: Children, Computers, and Powerful
% Ideas}. New York: Basic Books. Retrieved from
% http://lccn.loc.gov/79005200
%
% Parsons, J., \& Saunders, C. (2004). Cognitive heuristics in software
% engineering: Applying and extending anchoring and adjustment to artifact
% reuse. \emph{IEEE Transactions on Software Engineering}, \emph{30}(12),
% 873 -- 888. doi:10.1109/TSE.2004.94
%
% Patitsas, E., Craig, M., \& Easterbrook, S. (2013). On the Countably
% Many Misconceptions About \#Hashtables (Abstract Only). In
% \emph{Proceeding of the 44th ACM Technical Symposium on Computer Science
% Education} (pp. 739--739). New York, NY, USA: ACM.
% doi:10.1145/2445196.2445443
%
% Paul, W., \& Vahrenhold, J. (2013). Hunting High and Low: Instruments to
% Detect Misconceptions Related to Algorithms and Data Structures. In
% \emph{Proceeding of the 44th ACM Technical Symposium on Computer Science
% Education} (pp. 29--34). New York, NY, USA: ACM.
% doi:10.1145/2445196.2445212
%
% Pea, R. D. (1986). Language-independent conceptual`` bugs'' in novice
% programming. \emph{Journal of Educational Computing Research},
% \emph{2}(1), 25--36.
%
% Pea, R. D., Soloway, E., \& Spohrer, J. C. (1987). The Buggy Path to the
% Development of Programming Expertise. \emph{Focus on Learning Problems
% in Mathematics}, \emph{9}(1), 5--30.
%
% Petre, M., van der Hoek, A., \& Baker, A. (2010). Editorial.
% \emph{Design Studies}, \emph{31}(6), 533--544.
% doi:10.1016/j.destud.2010.09.001
%
% Rodrigo, M. M. T., \& Baker, R. S. J. d. (2009). Coarse-grained
% detection of student frustration in an introductory programming course.
% In \emph{Proceedings of the fifth international workshop on Computing
% education research workshop} (pp. 75--80). New York, NY, USA: ACM.
% doi:10.1145/1584322.1584332
%
% Rodrigo, M. M. T., Tabanao, E., Lahoz, M. B. ., \& Jadud, M. C. (2009).
% Analyzing Online Protocols to Characterize Novice Java Programmers.
% \emph{Philippine Journal of Science}, \emph{138}(2), 177--190.
%
% Rooksby, J. (2010). ``Just try to do it at the whiteboard'':
% Researcher-participant interaction and issues of generalisation. In
% \emph{Proceedings of the Studying Professional Software Design (SPSD)
% Conference}. San Diego, CA, USA.
%
% Rooksby, J., \& Ikeya, N. (2012). Collaboration in Formative Design:
% Working Together at a Whiteboard. \emph{IEEE Software}, \emph{29}(1), 56
% --60. doi:10.1109/MS.2011.123
%
% Rosenberg, S., Hammer, D., \& Phelan, J. (2006). Multiple
% Epistemological Coherences in an Eighth-Grade Discussion of the Rock
% Cycle. \emph{Journal of the Learning Sciences}, \emph{15}(2), 261--292.
% doi:10.1207/s15327809jls1502\_4
%
% Russ, R. S., Coffey, J. E., Hammer, D., \& Hutchison, P. (2008). Making
% classroom assessment more accountable to scientific reasoning: A case
% for attending to mechanistic thinking. \emph{Science Education},
% \emph{93}(5), 875--891. doi:10.1002/sce.20320
%
% Saussure, F. de. (1986). \emph{Course in general linguistics}. LaSalle,
% Ill: Open Court.
%
% Scherr, R. E., \& Hammer, D. (2009). Student Behavior and
% Epistemological Framing: Examples from Collaborative Active-Learning
% Activities in Physics. \emph{Cognition \& Instruction}, \emph{27}(2),
% 147--174. doi:10.1080/07370000902797379
%
% Schoenfeld, A. H. (1987). What's all the fuss about metacognition? In
% \emph{Cognitive Science and Mathematics Education} (pp. 189--215).
% Lawrence Erlbaum Associates.
%
% Schoenfeld, A. H. (1988). When good teaching leads to bad results: The
% disasters of ``well-taught'' mathematics courses. \emph{Educational
% Psychologist}, \emph{23}(2), 145--166. doi:10.1207/s15326985ep2302\_5
%
% Schoenfeld, A. H. (1991). On mathematics as sense-making: An informal
% attack on the unfortunate divorce of formal and informal mathematics. In
% \emph{Informal reasoning and education} (pp. 311--343).
%
% Schoenfeld, A. H. (1992). Learning to Think Mathematically: Problem
% Solving, Metacognition, and Sense-Making in Mathematics. In D. Grouws
% (Ed.), \emph{Handbook for research on mathematics teaching and learning}
% (pp. 334--370). New York: MacMillan.
%
% Schwartz, D. L., Chase, C. C., \& Bransford, J. D. (2012). Resisting
% Overzealous Transfer: Coordinating Previously Successful Routines With
% Needs for New Learning. \emph{Educational Psychologist}, \emph{47}(3),
% 204--214. doi:10.1080/00461520.2012.696317
%
% Seppälä, O., Malmi, L., \& Korhonen, A. (2006). Observations on student
% misconceptions---A case study of the Build -- Heap Algorithm.
% \emph{Computer Science Education}, \emph{16}(3), 241--255.
% doi:10.1080/08993400600913523
%
% Sherin, B. L. (2001). How students understand physics equations.
% \emph{Cognition and Instruction}, \emph{19}(4), 479--541.
%
% Smith, J. P., diSessa, A. A., \& Roschelle, J. (1993). Misconceptions
% Reconceived: A Constructivist Analysis of Knowledge in Transition.
% \emph{The Journal of the Learning Sciences}, \emph{3}(2), 115--163.
%
% Soloway, E. (1986). Learning to program = learning to construct
% mechanisms and explanations. \emph{Communications of the ACM},
% \emph{29}, 850--858. doi:10.1145/6592.6594
%
% Soloway, E., Bonar, J., \& Ehrlich, K. (1983). Cognitive Strategies and
% Looping Constructs: An Empirical Study. \emph{Commun. ACM},
% \emph{26}(11), 853--860. doi:10.1145/182.358436
%
% Soloway, E., \& Spohrer, J. C. (Eds.). (1989). \emph{Studying the Novice
% Programmer}. Hillsdale, N.J: L. Erlbaum Associates.
%
% Spacco, J., Hovemeyer, D., Pugh, W., Hollingsworth, J., Padua-Perez, N.,
% \& Emad, F. (2006). Experiences with Marmoset: Designing and Using an
% Advanced Submission and Testing System for Programming Courses. In
% \emph{ITiCSE '06: Proceedings of the 11th annual conference on
% Innovation and technology in computer science education}. ACM Press.
%
% Spacco, J., Pugh, W., Ayewah, N., \& Hovemeyer, D. (2006). The Marmoset
% project: an automated snapshot, submission, and testing system. In
% \emph{Companion to the 21st ACM SIGPLAN symposium on Object-oriented
% programming systems, languages, and applications} (pp. 669--670). New
% York, NY, USA: ACM. doi:10.1145/1176617.1176665
%
% Spacco, J., Strecker, J., Hovemeyer, D., \& Pugh, W. (2005). Software
% Repository Mining with Marmoset: An Automated Programming Project
% Snapshot and Testing System. In \emph{Proceedings of the Mining Software
% Repositories Workshop (MSR 2005)}. St. Louis, Missouri, USA.
%
% Spohrer, J. C., \& Soloway, E. (1986). Alternatives to construct-based
% programming misconceptions. \emph{ACM SIGCHI Bulletin}, \emph{17},
% 183--191. doi:10.1145/22339.22369
%
% Stevens, R., \& Hall, R. (1998). Disciplined perception: Learning to see
% in technoscience. In \emph{Talking Mathematics in School: Studies of
% Teaching and Learning} (pp. 107--150). Cambridge, U.K: Cambridge
% University Press.
%
% Stieff, M. (2007). Mental rotation and diagrammatic reasoning in
% science. \emph{Learning and Instruction}, \emph{17}(2), 219--234.
% doi:10.1016/j.learninstruc.2007.01.012
%
% Tabanao, E. S., Rodrigo, M. M. T., \& Jadud, M. C. (2011). Predicting
% at-risk novice Java programmers through the analysis of online
% protocols. In \emph{Proceedings of the seventh international workshop on
% Computing education research} (pp. 85--92). New York, NY, USA: ACM.
% doi:10.1145/2016911.2016930
%
% Tannen, D. (Ed.). (1993). \emph{Framing in Discourse}. New York: Oxford
% University Press.
%
% Trakhtenbrot, M. (2013). Students Misconceptions in Analysis of
% Algorithmic and Computational Complexity of Problems. In
% \emph{Proceedings of the 18th ACM Conference on Innovation and
% Technology in Computer Science Education} (pp. 353--354). New York, NY,
% USA: ACM. doi:10.1145/2462476.2465604
%
% Traweek, S. (1988). \emph{Beamtimes and lifetimes: the world of high
% energy physicists}. Cambridge, Mass: Harvard University Press.
%
% Turkle, S., \& Papert, S. (1991). Epistemological Pluralism and the
% Reevaluation of the Concrete. In I. Harel, S. Papert, \& Massachusetts
% Institute of Technology (Eds.), \emph{Constructionism: research reports
% and essays, 1985-1990}. Norwood, N.J: Ablex Pub. Corp.
%
% Valentine, D. W. (2004). CS educational research: a meta-analysis of
% SIGCSE technical symposium proceedings. \emph{SIGCSE Bull.},
% \emph{36}(1), 255--259. doi:10.1145/1028174.971391
%
% Van de Sande, C. C., \& Greeno, J. G. (2012). Achieving Alignment of
% Perspectival Framings in Problem-Solving Discourse. \emph{Journal of the
% Learning Sciences}, \emph{21}(1), 1--44.
% doi:10.1080/10508406.2011.639000
%
% VanLehn, K. (1990). \emph{Mind Bugs: The Origins of Procedural
% Misconceptions}. Cambridge, Mass: MIT Press.
%
% Wagner, J. F. (2006). Transfer in Pieces. \emph{Cognition \&
% Instruction}, \emph{24}(1), 1--71. doi:10.1207/s1532690xci2401\_1
%
% Wortham, S. (2006). \emph{Learning Identity: The Joint Emergence of
% Social Identification and Academic Learning}. Cambridge: Cambridge
% University Press.
%
% Yackel, E., \& Cobb, P. (1996). Sociomathematical Norms, Argumentation,
% and Autonomy in Mathematics. \emph{Journal for Research in Mathematics
% Education}, \emph{27}(4), 458--477. doi:10.2307/749877
%
% Yin, R. K. (2009). \emph{Case study research: design and methods} (4th
% ed.). Los Angeles, Calif: Sage Publications.
